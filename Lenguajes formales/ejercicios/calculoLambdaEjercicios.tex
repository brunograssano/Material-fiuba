\subsection{Calculo Lambda}
\subsubsection*{Ejercicio 1}
Escribir las siguientes expresiones con el menor número de paréntesis posible:

\begin{enumerate}[label=\alph*)]
    \item  ($\lambda$x.($\lambda$y.($\lambda$z.( (x z) (y z) ) ) ) )
    
    $\lambda$x. ($\lambda$y.($\lambda$z.( (x z) (y z) ) ) ) - Los externos se omiten (1)
    
    $\lambda$x.$\lambda$y. ($\lambda$z.( (x z) (y z) ) ) - Cuerpo de las abstracciones hasta el final o paréntesis (3)
    
    $\lambda$x.$\lambda$y.$\lambda$z. ( (x z) (y z) ) - Cuerpo de las abstracciones hasta el final o paréntesis (3)
    
     $\lambda$x.$\lambda$y.$\lambda$z. (x z) (y z) - Cuerpo de las abstracciones hasta el final o paréntesis (3)
    
    $\lambda$x.$\lambda$y.$\lambda$z. x z (y z) - Aplicaciones se asocian a la izquierda (2)? % la pag da este
    
    $\lambda$x. y.  z. (x z) (y z) - Opcional, se contraen las abstracciones lambda (4)
    
    \item ( ( (a b) (c d) ) ( (e f) (g h) ) )
    
    ( (a b) (c d) ) ( (e f) (g h) ) - Los externos se omiten (1)
    
   (a b) (c d) ( (e f) (g h) )  - Aplicaciones a la izquierda (2)
    
    a b (c d) ( (e f) (g h) )  - Aplicaciones a la izquierda (2)
    
    a b (c d) ( e f (g h) )  - Aplicaciones a la izquierda (2) % resultado dado por la pag
    
    \item ( $\lambda$x. ( ($\lambda$y. (y x)) ($\lambda$v. v) z) u) ($\lambda$w.w) 
    
    ( $\lambda$x. ( ($\lambda$y. y x ) ($\lambda$v. v) z) u) ($\lambda$w.w) - Aplicaciones a la izquierda (2)
    
    ( $\lambda$x. ($\lambda$y. y x ) ($\lambda$v. v) z u) ($\lambda$w.w) - Aplicaciones a la izquierda (2)

\end{enumerate}

\subsubsection*{Ejercicio 2}
Restaurar todos los paréntesis descartados en las siguientes expresiones:

\begin{enumerate}[label=\alph*)]
\item  x x x x

( x x x x ) - Los externos se omiten (1)

( ( x x x ) x ) - Aplicaciones se asocian a la izquierda (2)

( ( ( x x ) x ) x ) - Aplicaciones se asocian a la izquierda (2)

\item $\lambda$x. x $\lambda$y. y

( $\lambda$x. x $\lambda$y. y ) - Los externos se omiten (1)

( $\lambda$x. ( x $\lambda$y. y ) ) - Cuerpo de las abstracciones hasta el final o paréntesis (3)

( $\lambda$x. ( x ( $\lambda$y. y ) ) ) - Aplicaciones se asocian a la izquierda (2) %?

\item $\lambda$x.( x $\lambda$y.y x x ) x

 ( $\lambda$x.( x $\lambda$y.y x x ) x ) - Los externos se omiten (1)

 ( ($\lambda$x.( x $\lambda$y.y x x  ) ) x ) - Aplicaciones se asocian a la izquierda (2)

 ( ($\lambda$x.(  ( x $\lambda$y.y x ) x ) ) x ) - Aplicaciones se asocian a la izquierda (2)

 ( ($\lambda$x.(  ( x $\lambda$y.y x ) x ) ) x ) - Aplicaciones se asocian a la izquierda (2)

 ( ($\lambda$x.(  (  (x $\lambda$y.y ) x ) x ) ) x ) - Aplicaciones se asocian a la izquierda (2)

 ( ($\lambda$x.(  (  (x ($\lambda$y.y) ) x ) x ) ) x ) - Aplicaciones se asocian a la izquierda (2)

\end{enumerate}


\subsubsection*{Ejercicio 3}
Para las siguientes expresiones lambda:

\begin{enumerate}[label=\alph*)]
\item Identificar las ocurrencias de variables libres y ligadas. 
\end{enumerate}

\begin{enumerate}
    \item  ( $\lambda$x.( ( $\lambda$y.y ) x ) ) z  -  x ligada, y ligada, z libre %1
    
	( $\lambda$x.( ( $\lambda$y.y ) x ) ) z       - Beta con z (orden normal)

     ( $\lambda$y.y ) z ) - Beta con z

      			 z
    		%%%%%%%%% aplicativo 
     ( $\lambda$x.( ( $\lambda$y.y ) x ) ) z 
    
	( $\lambda$x.x  ) z       - Beta con x (orden aplicativo)

     z  - Beta con z
    
    
    
    \item ( $\lambda$x.$\lambda$y.x y ) ( z y )  -  x ligada, y ligada (operador), y libre (operando) , z libre % 2
    
( $\lambda$x.$\lambda$y.x y ) ( z y ) - Beta con x y alfa con la y (si no estaría ligada) (igual con orden normal y aplicativo)
    
$\lambda$u. ( z y )  u 

    \item ( $\lambda$x.$\lambda$y.x ) x y -  x ligada en la funcion, x e y libres % 3
    

    \item ( $\lambda$x.( ( $\lambda$z.z x ) ( $\lambda$x.x ) ) ) y - x ligada en ambos casos, y libre, z ligada % 4
        
    \item ( $\lambda$x.( ( $\lambda$y.x y ) z ) ) ( $\lambda$x.x y ) %5
    
     x ligada en ambos casos, y ligada (operador de la abstracción interna de la función), y libre (operando), z libre
    

    \item ( ( $\lambda$y.( $\lambda$x.( ( $\lambda$x.$\lambda$y.x ) x ) ) y ) M ) N %6
    
     ( $\lambda$y.( $\lambda$x.( ( $\lambda$x.$\lambda$y.x ) x ) ) y ) M  N
    
    
     - x interna ligada, x externa ligada, y ligada
    
    
    \item ( $\lambda$x.$\lambda$y.$\lambda$x.x y z ) ($\lambda$x.$\lambda$y.y) M N - %7
    
    x interno ligado en la funcion
    y ligado en la funcion y en el parametro
    x libre en el parametro
    x externo libre en la funcion
    z libre
    
    
    \item ( ( $\lambda$x.( $\lambda$y.$\lambda$z.z ) x ) ( ( $\lambda$x.x x x ) ( $\lambda$x.x x x ) ) ) x %8
    
      ( $\lambda$x.( $\lambda$y.$\lambda$z.z ) x ) ( ( $\lambda$x.x x x ) ( $\lambda$x.x x x ) )  x
    
    aplicación
    en el parámetro mas externo: x libre
	en la función:
    tengo otra aplicación    
    aplicación izquierda 
    
    x ligada, y libre, z ligada
    aplicación derecha, sus dos abstracciones con x están ligadas (cada x a su lambda)
  
\end{enumerate}
 
\section*{Notas de las practicas}

\begin{itemize}
\item Jamas puedo hacer un cambio de variable en una variable libre.
\item Si no tengo variables libres en el parámetro no es necesario un cambio de variable
\item No hacer cambios de variables de mas, es para ver si entendimos el concepto.
\item Orden normal, si el parámetro se puede llegar a reducir, lo meto sin reducirlo. 
\item Orden aplicativo (lazy evaluation), el parámetro lo puedo reducir? Si, lo reduzco - Puedo reducir el cuerpo de mi función? Si, reduzco cuerpo y parámetro y vuelvo a escribir.
\item No ver los ordenes como \textit{la mas interna etc}, usar lo mencionado arriba para evitar equivocaciones.
\item Siempre empezar identificando variables libres y ligadas, es para ver donde hago el remplazo.
\item Si ambas formas llegan a un resultado, tiene que ser el mismo.
\item Evaluo de izquierda a derecha.
\item Si hay mas de un paréntesis a la izquierda, el resto esta de mas. Se borran.
\item Solo se puede hacer cambio de variable a las variables ligadas al lambda. Nunca cambiar lo que esta entrando.
\end{itemize}


