\subsection{APL}
\textit{Notas tomadas de las practicas.}

\begin{itemize}
\item Sustantivos todos los parámetros que puede tener una función.
\item Todos los verbos llevan uno o dos parámetros como mucho.
\item Adverbio es un operador de orden superior, actúa sobre un verbo. Lo modifica. Ejemplo la barra.
\item Conjunción es un operador de orden superior que actúa sobre dos verbos generando un verbo derivado. Ejemplo el producto interno.
\item APL se lee de derecha a izquierda.
\item Los parámetros son inmutables, nunca se modifican los resultados de los parámetros.
\item El barra hace la comprensión de todas las columnas, inserta la función entre medio de todas las columnas (en realidad estoy comprimiendo las filas)
\item Con un vector es lo mismo poner la barra y la barra cruzada, tenemos un solo elemento.
\end{itemize}







