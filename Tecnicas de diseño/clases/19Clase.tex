\section{Docker}
\begin{itemize}
\item Docker es una herramienta que automiza el despliegue de aplicaciones dentro de contenedores
\end{itemize}


\subsection*{Contexto}
\subsubsection*{Linux Namespaces}
\begin{itemize}
\item Agrupar y controlar acceso a recursos del sistema operativo. Separa estos recursos para distintos grupos de procesos.
\item Es de muy bajo nivel
\end{itemize}


\subsubsection*{Infrastructure as Code}
\begin{itemize}
\item Desplegar entornos automáticamente a partir de una descripción del entorno deseado
\item Se especifica que queremos tener y el gestor se encarga de hacerlo.
\item Surge el Problema de imagen dorada donde se siguen agregando modificaciones sobre un estado inicial que no se sabe o no se puede reconstruir.
\end{itemize}



\subsection*{Docker}
\begin{itemize}
\item Docker aparece para arreglar esto.
\item Container: un grupo aislado de procesos
\item Image: Un template usado para crear contenedores
\item Dockerfile: un archivo con instrucciones para construir una imagen, el dockerfile que genero la imagen(no garantiza generar una imagen idéntica)
\item Docker nos brinda consistencia, cualquier imagen se obtiene e inicia de la misma manera.
\item Nos brinda replicabilidad, los contenedores basados en una misma imagen son inicialmente iguales.
\item Aislamiento, se controla la interacción entre la aplicación en un contenedor y el mundo exterior
\item Los contenedores son livianos, se dedican a un único propósito, mantienen el estado mínimo y se pueden (re)iniciar rápidamente
\end{itemize}



\subsubsection*{Dockerfile}
\begin{itemize}
\item Describe la serie de pasos para construir una imagen
\item Cada una empieza referenciando a otra imagen generalmente que se referencia a si misma.
\item Maneja también el volumen del área de datos, las networks para interfaz de red interna, y los vínculos con el exterior (mounts, ports)
\item Hay ciertas aplicaciones que no pueden operar dentro de un contenedor por cuestiones de rendimiento o permisos.
\item No es una maquina virtual.
\end{itemize}


\subsubsection*{Orquestadores}

\begin{itemize}
\item Coordinan el uso de múltiples contenedores para implementar una única aplicación.
\item Permite tener tolerancia a fallos, balancear la carga, distribuir, y configurar contenedores heterogéneos.
\end{itemize}


% api nos devuelve un codigo y otro servicio lo valida
% se puede integrar alguna api de servicio de email tambien
% podemos simplificarlo
% paquetes de envio SMTP
% biblioteca mailgun? que consiste en hacer un POST a donde dicen
% forma sencilla es imprimir en consola o devolverlo a modo de respuesta
% JWT para autenticacion, generar token


% elegir imagen particular de node y mongo en docker