\section{Practica: Ejercicio AFA y Taller I}
% libro ] Fundamentals of Database Systems,Elmasri, S. Navathe

\subsubsection*{Conceptos}
\begin{itemize}
\item Cardinalidad: primer termino mínimo, segundo máximo en la cardinalidad (de ambos lados)
\item Conjunto de entidad
\item Entidad
\item Atributos (claves, otros)
\item Semántica, el significado de la interrelación
\item Interrelación
\item Conjunto de interrelación
\end{itemize}


\subsubsection*{Notas}

\begin{itemize}
\item La cardinalidad máxima tiene que estar si o si, se describe con esta siempre. Para enriquecer se agrega la cardinalidad mínima
\item Una entidad débil puede ser a su vez una entidad fuerte de otra.
\item No puedo tener un 0 como cardinalidad mínima en una entidad fuerte por la dependencia existencial.
\item Prestar atención a los sustantivos y a los verbos. Para ver cuales son las entidades mas visibles, se puede contar cuantas veces aparece un sustantivo.
\item En el diagrama los sustantivos los ponemos en singular, cuando lo pasamos a la tabla los ponemos en plural generalmente.
\item En las relaciones ternarias usamos solo las cardinalidades máximas.
\item Por mas que se tenga la misma cardinalidad como mínima y máxima, se ponen ambas (1,1)
\item Siempre tienen que estar los identificadores únicos.
\item Tratar de no usar abreviaturas.
\item Lo que no entra en el diagrama, se anota aparte (en el diccionario de datos)
\end{itemize}



\subsubsection*{AFA}
\begin{itemize}
\item Estamos modelando los torneos solamente. No irse de la narrativa de estos.
\item Equipo y encuentro es una relación particular, una binaria que tratamos como ternaria. 
\item En este caso usamos 2 binarias para evitar esa relación particular
\end{itemize}


\begin{figure}[!htb]
    \centering
    \includegraphics[width=0.8\textwidth]{img/EjercicioAFA.PNG}
\end{figure}

\subsubsection*{Taller}

\begin{itemize}
\item Usamos schemas para ordenar las tablas lógicamente. (tiene otros motivos de seguridad y demás también)
\item Las primeras lineas usarlas para documentación. Nombre del archivo, resumen del script, fecha creado y modificado, y autor. (medio innecesario los últimos tres si se usa git)
\end{itemize}


\begin{minted}
[
frame=leftline,
framesep=5mm,
baselinestretch=1.2,
]
{sql}
DROP TABLE IF EXISTS paradas -- Buena practica

CREATE TABLE paradas (
	cod_parada integer NOT NULL,
    longitud numeric NOT NULL,
    latitud numeric NOT NULL,
    tipo_parada varchar,
    calle varchar,
    altura integer,
    entre1 varchar,
    entre 2 varchar
);

ALTER TABLE paradas -- Se cambia el dueño de la tabla
	OWNER TO postgres;
\end{minted}

\begin{itemize}
\item Al hacer consultas siempre tratar de poner un limit, en caso de tener tablas grandes. \textit{Ejemplo:} \mint{sql}|SELECT ... limit 10;|
\item Al trabajar con $csv$ tener cuidado con los '.' y ',' cuando tenemos numeros. Revisar tambien los simbolos que no sean estandar \textit{por ejemplo la 'ñ'}.
\item Para exportar los archivos, irían a un lugar publico. \textit{En Windows: C:$\backslash$Users$\backslash$Public}
\end{itemize}


\textit{Articulo recomendado: \href{https://martinfowler.com/articles/evodb.html}{Evolutionary Database Design - Martin Fowler}}


\newpage