\section{Seguridad}

Seguridad de la información: Es el conjunto de procedimientos y medidas para proteger a los componentes de los sistemas de información

\medskip
Implica:

\begin{itemize}
\item \textbf{Confidencialidad}: La información no sea ofrecida a individuos que no están autorizados a verla.
\item \textbf{Integridad}: Asegurar la correctitud durante el ciclo de vida de la información. Por ejemplo alterar las notas
\item \textbf{Disponibilidad}: Asegurar que la información este disponible cuando las personas autorizadas la requieran.
\item \textbf{No repudio}: Alguien que accedió a la información no pueda negar haberlo hecho. Que quede un registro del acceso.
\end{itemize}

\medskip
Hay que cubrir la información en distintos niveles (Personas, aplicaciones, red, OS, SGBD, archivos). Ahora nos focalizamos en lo que puede hacer el SGBD.

\subsection*{Control de acceso basado en roles (RBAC)}
Esta basado en definir roles para las distintas actividades y funciones desarrolladas por los miembros de una organización, con el objetivo de regular el acceso de los usuarios a los recursos disponibles

\medskip
Los elementos/entidades son:
\begin{itemize}
\item Usuarios: son las personas
\item Roles:  son conjuntos de funciones y responsabilidades
\item Objetos: son aquello a ser protegido
\item Operaciones: son las acciones que pueden realizarse sobre objetos
\item Permisos: Acciones concedidas o revocadas a un usuario o rol sobre un objeto determinado.
\end{itemize}

\newpage
\begin{figure}[!htb]
    \centering
    \includegraphics[width=0.8\textwidth]{img/ModeloRBAC.png}
\end{figure}


Soporta tres principios:

\begin{itemize}
\item Criterio del menor privilegio posible, si un usuario no va a realizar una operación, no debería de tener los permisos para realizarla.
\item División de responsabilidades, nadie debe de tener suficiente privilegios para usar el sistema en beneficio propio. Necesita de roles excluyentes. \textit{Por ejemplo, Que el pago de una factura a un proveedor requiera la aprobación de un empleado de Pagos y la carga de los datos de los proveedores deba realizarla un empleado de Compras.}
\item Abstracción de datos: los permisos son abstractos, dependen del objeto en cuestión.
\end{itemize}

\newpage
\subsection*{Autenticación y permisos en SQL}

En Postgres el esquema queda de la siguiente forma:

\begin{figure}[!htb]
    \centering
    \includegraphics[width=0.8\textwidth]{img/ModeloRBACPostgres.png}
\end{figure}


\begin{itemize}
\item Postgres permite definir políticas a nivel de fila. Solo puede haber algunas filas según mis permisos al hacer un SELECT.
\item Solo veo hasta donde tengo permisos al ver la tabla.
\item Solo el dueño puede crear o borrar la tabla. Se puede cambiar el dueño. Lo mismo pasa con las databases y los esquemas. DROP y ALTER no son privilegios que pueden darse.
\item El dueño de un objeto tiene todos los privilegios sobre el objeto, y puede extenderlos a los usuarios que desee.
\item Los superusuarios tienen todos los privilegios sobre todos los objetos, y son irrevocables.
\end{itemize}

\subsection*{SQL Injection}
Son una de las falla de seguridad más frecuentes en sistemas de bases de datos relacionales. Se manipula a nivel aplicación las variables de entrada de una consulta SQL. Sus consecuencias pueden ser graves.


\medskip
La solución es hacer todo lo que se pueda para evitarlo.

\begin{itemize}
\item Una solución es usar una función de escape sobre los parámetros.
\item Otra es aprovechar los mecanismos de control de acceso. Desde afuera se conectan con algún usuario especifico al que controlamos los permisos.
\item Validar los parámetros, castear cada dato al tipo correspondiente antes de anexarlo a la consulta.
\item Usar consultas parametrizadas.
\end{itemize}

\newpage