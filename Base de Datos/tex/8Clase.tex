\section{Practica: Taller SQL}
\begin{itemize}
    \item Es un lenguaje declarativo, le digo que quiero y de alguna forma lo hace.
    \item ¿Cual forma de hacer algo es mas performante? No es fácil saberlo, generalmente el motor se puede dar cuenta cual hacer.
    \item Las cadenas de caracteres con comillas simples. Las comillas dobles las interpreta como que queremos acceder a una columna.
    \item Operadores lógicos AND, OR, NOT
    \item Evitar usar SELECT * ... , ya que no queda definida la consulta en ese momento.
    \item Definir forma de almacenar datos, esto es buena practica, pero trae problemas ante alguna equivocación.
    \item Con los operadores de conjuntos por defecto no hay duplicados.
    \item No usar NATURAL JOIN en aplicativos para evitar problemas a futuro, ya que la consulta no queda totalmente definida. Si el esquema de la tabla cambia, nos enteramos cuando se ejecuta.
    \item El resultado de una consulta se puede usar como tabla en la entrada de otra consulta (subconsulta)
    \item Cuando agrupo puedo devolver alguna de las columnas agrupadas, o resultados que surgen de alguna función de agregación.
\end{itemize}

\subsection*{1 - Devuelva todos los datos de las notas que no sean de la materia 75.1}

\begin{itemize}
\item Con que difiera alguno de ellos ya nos alcanza. Fijarse con De Morgan. Plantearlo con \mint{sql}|AND| es un error común, ya que obligamos a que sea distinto si o si, en este caso hay que negar el resultado si queremos el \mint{sql}|AND|.
\end{itemize}


\begin{minted}
[
frame=leftline,
framesep=5mm,
baselinestretch=1.2,
]
{sql} 
SELECT * FROM notas WHERE codigo <> 75 OR numero <> 1
\end{minted}

\subsection*{2 Devuelva para cada materia dos columnas: una llamada “código” que contenga una concatenación del código de departamento, un punto y el numero de materia, con el formato “XX.YY” (ambos valores con dos dígitos, agregando ceros a la izquierda en caso de ser necesario) y otra con el nombre de la materia.}

\begin{minted}
[
frame=leftline,
framesep=5mm,
baselinestretch=1.2,
]
{sql} 
SELECT to_char(codigo,'fm00') || '.' || to_char(numero,'fm00') AS "codigo",nombre
FROM materias
\end{minted}

\begin{itemize}
\item Con el doble pipe lo estamos concatenando con un punto en el medio.
\item Con el AS renombramos la columna inicial.
\item Documentación del \href{https://www.postgresql.org/docs/current/functions-formatting.html}{to\_char}.
\end{itemize}



\subsection*{3 - Para cada nota registrada, devuelva el padrón, código de departamento, número de materia, fecha y nota expresada como un valor entre 1 y 100.}

\begin{minted}
[
frame=leftline,
framesep=5mm,
baselinestretch=1.2,
]
{sql} 
SELECT padron,codigo,numero,fecha,nota*10 AS "nota"
FROM notas
\end{minted}

\subsection*{4 -  Ídem al anterior pero mostrando los resultados paginados en páginas de 5 resultados cada una, devolviendo la segunda página}

\begin{minted}
[
frame=leftline,
framesep=5mm,
baselinestretch=1.2,
]
{sql} 
SELECT padron,codigo,numero,fecha,nota*10 AS "nota"
FROM notas
OFFSET 5 ROWS FETCH FIRST 5 ROWS ONLY
\end{minted}


\begin{itemize}
\item Esta \mint{sql}|LIMIT 5| después del offset también.
\item Si no le digo algún orden, no se que me va a devolver. Si queremos garantizarnos los resultados, hay que tomarse la costumbre de ordenar de la forma que no deje ambigüedad en las filas. En este caso agregando antes del \mint{sql}|FETCH| un \mint{sql}|ORDER BY padron| - Así evitamos tener un orden indeterminado.
\end{itemize}

\begin{minted}
[
frame=leftline,
framesep=5mm,
baselinestretch=1.2,
]
{sql} 
SELECT padron,codigo,numero,fecha,nota*10 AS "nota"
FROM notas
ORDER BY padron,codigo,numero,fecha
OFFSET 5 ROWS FETCH FIRST 5 ROWS ONLY;
\end{minted}

\subsection*{5 - Ejecute una consulta SQL que devuelva el padrón y nombre de los alumnos cuyo apellido es “Molina”}
\begin{minted}
[
frame=leftline,
framesep=5mm,
baselinestretch=1.2,
]
{sql} 
SELECT padron,nombre
FROM alumnos
WHERE apellido = 'Molina';
\end{minted}

\begin{itemize}
\item No es lo mismo que: (el siguiente estaría mal)
\end{itemize}

\begin{minted}
[
frame=leftline,
framesep=5mm,
baselinestretch=1.2,
]
{sql} 
SELECT padron,nombre
FROM alumnos
WHERE apellido = 'molina';
\end{minted}


\begin{itemize}
\item Postgres es una base de datos CASE SENSITIVE. Por lo que no devolvería nada en este caso. \mint{sql}|LIKE| es case sensitive también, se puede usar \mint{sql}|ILIKE| (no es estándar y no aprovecha índices)
\end{itemize}

\begin{minted}
[
frame=leftline,
framesep=5mm,
baselinestretch=1.2,
]
{sql} 
SELECT padron,nombre
FROM alumnos
WHERE apellido ILIKE 'Molina';
\end{minted}


\begin{itemize}
\item Otra alternativa es llevándolo a una forma común con Upper o Lower.
\end{itemize}

\begin{minted}
[
frame=leftline,
framesep=5mm,
baselinestretch=1.2,
]
{sql} 
SELECT padron,nombre
FROM alumnos
WHERE Upper(apellido) = 'MOLINA';
\end{minted}


\subsection*{6 - Obtener el padrón de los alumnos que ingresaron a la facultad en el año 2010}

\begin{minted}
[
frame=leftline,
framesep=5mm,
baselinestretch=1.2,
]
{sql} 
SELECT padron
FROM alumnos
WHERE fecha_ingreso>='2010-01-01' AND fecha_ingreso<='2010-12-31';
\end{minted}

Con otra sintaxis:
\begin{minted}
[
frame=leftline,
framesep=5mm,
baselinestretch=1.2,
]
{sql} 
SELECT padron
FROM alumnos
WHERE fecha_ingreso BETWEEN '2010-01-01' AND '2010-12-31';
\end{minted}

\subsection*{7 - Obtener la mejor nota registrada en la materia 75.15}

\begin{minted}
[
frame=leftline,
framesep=5mm,
baselinestretch=1.2,
]
{sql} 
SELECT MAX(nota)
FROM notas
WHERE codigo=75 AND numero=15;
\end{minted}

\subsection*{8 - Obtener el promedio de notas de las materias del departamento de código 75}

\begin{minted}
[
frame=leftline,
framesep=5mm,
baselinestretch=1.2,
]
{sql} 
SELECT AVG(nota)
FROM notas
WHERE codigo=75;
\end{minted}


\subsection*{9 - Obtener el promedio de nota de aprobación de las materias del departamento de código 75.}

\begin{minted}
[
frame=leftline,
framesep=5mm,
baselinestretch=1.2,
]
{sql} 
SELECT AVG(nota)
FROM notas
WHERE codigo=75 and nota>=4;
\end{minted}

\subsection*{10 - Obtener la cantidad de alumnos que tienen al menos una nota}

\begin{minted}
[
frame=leftline,
framesep=5mm,
baselinestretch=1.2,
]
{sql} 
SELECT COUNT(DISTINCT padron)
FROM notas
\end{minted}


\begin{itemize}
\item \mint{sql}|COUNT| solo cuenta los distintos de nulos, si queremos que sean distintas lo hacemos con \mint{sql}|DISTINCT|.
\end{itemize}


\subsection*{11 - Devolver los padrones de los alumnos que no registran nota en materias}


\begin{itemize}
\item Estamos realizando la operación de conjuntos: A - B
\end{itemize}

\begin{minted}
[
frame=leftline,
framesep=5mm,
baselinestretch=1.2,
]
{sql} 
(SELECT padron
FROM alumnos)
EXCEPT
(SELECT padron
FROM notas)
\end{minted}

\begin{minted}
[
frame=leftline,
framesep=5mm,
baselinestretch=1.2,
]
{sql} 
SELECT padron
FROM alumnos
WHERE padron
NOT IN
(SELECT DISTINC(padron) FROM notas)
\end{minted}

\subsection*{12 - Con el objetivo de traducir a otro idioma los nombres de materias y departamentos, devolver en una única consulta los nombres de todas las materias y de todos los departamentos}

\begin{minted}
[
frame=leftline,
framesep=5mm,
baselinestretch=1.2,
]
{sql} 
(SELECT nombre
FROM materias)
UNION
(SELECT nombre
FROM departamentos)
\end{minted}

\subsection*{13 - Devolver para cada materia su nombre y el nombre del departamento}


\begin{itemize}
\item Con \mint{sql}|JOIN|
\end{itemize}
\begin{minted}
[
frame=leftline,
framesep=5mm,
baselinestretch=1.2,
]
{sql} 
SELECT m.nombre, d.nombre
FROM materias AS m, departamentos AS d
WHERE m.codigo=d.codigo 
\end{minted}


\begin{itemize}
\item Otras forma para hacerlo:
\end{itemize}
\begin{minted}
[
frame=leftline,
framesep=5mm,
baselinestretch=1.2,
]
{sql} 
SELECT m.nombre, d.nombre
FROM materias AS m INNER JOIN  departamentos AS d
ON m.codigo=d.codigo 
\end{minted}

\begin{minted}
[
frame=leftline,
framesep=5mm,
baselinestretch=1.2,
]
{sql} 
SELECT m.nombre, d.nombre
FROM materias AS m INNER JOIN departamentos AS d
USING (codigo) 
\end{minted}


\subsection*{14 - Para cada 10 registrado, devuelva el padrón y nombre del alumno y el nombre de la materia correspondientes a dicha nota}

\begin{minted}
[
frame=leftline,
framesep=5mm,
baselinestretch=1.2,
]
{sql} 
SELECT a.padron,a.nombre,m.nombre
FROM alumnos AS a, notas AS n, materias AS m
WHERE n.nota=10 AND a.padron=n.padron
AND n.codigo=m.codigo AND n.numero=m.numero;
\end{minted}

\begin{minted}
[
frame=leftline,
framesep=5mm,
baselinestretch=1.2,
]
{sql} 
SELECT a.padron,a.nombre,m.nombre
FROM notas AS n
INNER JOIN materias AS m ON m.numero=n.numero AND m.codigo=m.codigo
INNER JOIN alumnos AS a ON a.padron=n.padron
WHERE n.nota=10;
\end{minted}

\begin{minted}
[
frame=leftline,
framesep=5mm,
baselinestretch=1.2,
]
{sql} 
SELECT n.padron,a.nombre AS nombre_alumno,m.nombre AS nombre_materia
FROM (notas n INNER JOIN alumnos a USING (padron)) INNER JOIN materias m 
USING (codigo,numero)
WHERE n.nota = 10
\end{minted}

\subsection*{15 - Listar para cada carrera su nombre y el padrón de los alumnos que estén anotados en ella. Incluir también las carreras sin alumnos inscriptos}

\begin{minted}
[
frame=leftline,
framesep=5mm,
baselinestretch=1.2,
]
{sql} 
SELECT c.nombre,i.padron
FROM carreras AS c LEFT OUTER JOIN inscripto_en AS i
USING (codigo)
\end{minted}

\begin{minted}
[
frame=leftline,
framesep=5mm,
baselinestretch=1.2,
]
{sql} 
SELECT carreras.nombre,alumnos.nombre,alumnos.apellido
FROM carreras
LEFT OUTER JOIN inscripto_en ON inscripto_en.codigo = carreras.codigo
LEFT OUTER JOIN alumnos ON inscripto_en.padron = alumnos.padron
\end{minted}

\subsection*{16 - Ídem punto anterior pero teniendo en cuenta únicamente alumnos con padrón mayor a 75000}

% REVISAR
\begin{minted}
[
frame=leftline,
framesep=5mm,
baselinestretch=1.2,
]
{sql} 
SELECT carreras.nombre,alumnos.nombre,alumnos.apellido
FROM carreras
LEFT OUTER JOIN inscripto_en ON inscripto_en.codigo = carreras.codigo
LEFT OUTER JOIN alumnos ON inscripto_en.padron = alumnos.padron
WHERE alumnos.padron > 75000
\end{minted}

\begin{minted}
[
frame=leftline,
framesep=5mm,
baselinestretch=1.2,
]
{sql} 
SELECT c.nombre,i.padron
FROM carreras AS c LEFT OUTER JOIN inscripto_en AS i
USING (codigo)
WHERE padron > 75000
\end{minted}

\begin{minted}
[
frame=leftline,
framesep=5mm,
baselinestretch=1.2,
]
{sql} 
SELECT c.nombre, ie.padron
FROM taller_4.CARRERAS c 
LEFT OUTER JOIN taller_4.inscripto_en ie ON (ie.codigo = c.codigo AND ie.padron > 75000);
\end{minted}

\subsection*{17 - Listar el padrón de aquellos alumnos que tengan m´as de una nota en la materia 75.15}


\begin{itemize}
\item De esta forma no se tiene la mejor performance, ya que se esta revisando fila por fila con el valor de la consulta mayor.
\end{itemize}

\begin{minted}
[
frame=leftline,
framesep=5mm,
baselinestretch=1.2,
]
{sql} 
SELECT a.padron
FROM alumnos as a
WHERE 
(SELECT count(n.nota)>1 FROM notas AS n
WHERE n.padron=a.padron AND n.codigo=75 AND n.numero=15)
\end{minted}

\begin{itemize}
\item Otra opción poniendo el >1 afuera de la subconsulta.
\end{itemize}
\begin{minted}
[
frame=leftline,
framesep=5mm,
baselinestretch=1.2,
]
{sql} 
SELECT a.padron
FROM alumnos as a
WHERE 
(SELECT count(n.nota) FROM notas AS n
WHERE n.padron=a.padron AND n.codigo=75 AND n.numero=15) >1 
\end{minted}

\begin{minted}
[
frame=leftline,
framesep=5mm,
baselinestretch=1.2,
]
{sql} 
SELECT DISTINCT n1.padron
FROM notas n1, notas n2
WHERE n1.codigo = 75 AND n2.codigo = 75
AND n1.numero = 15 AND n2.numero = 15
AND n1.padron = n2.padron
AND n1.fecha <> n2.fecha
\end{minted}

\subsection*{18 - Obtenga el padrón y nombre de los alumnos que aprobaron la materia 71.14 y no aprobaron la materia 75.15}

\begin{minted}
[
frame=leftline,
framesep=5mm,
baselinestretch=1.2,
]
{sql} 
SELECT A.PADRON, A.NOMBRE 
FROM ALUMNOS A
WHERE EXISTS(SELECT 1 FROM NOTAS N WHERE N.NOTA >= 4
            AND N.CODIGO = 71 AND N.NUMERO = 14 
            AND N.PADRON = A.PADRON) 
AND NOT EXISTS(SELECT 1 FROM NOTAS N WHERE N.NOTA >= 4 
            AND N.CODIGO = 71 AND N.NUMERO = 15 
            AND N.PADRON = A.PADRON);
\end{minted}
\begin{itemize}
\item Se fija que exista un aprobado en la materia, y que no exista un aprobado en la otra.
\end{itemize}


Otra forma:
\begin{minted}
[
frame=leftline,
framesep=5mm,
baselinestretch=1.2,
]
{sql} 
SELECT A.PADRON, A.NOMBRE 
FROM ALUMNOS A
WHERE A.PADRON IN (SELECT N.PADRON FROM NOTAS N WHERE N.NOTA >= 4 
			 AND N.CODIGO = 71 AND N.NUMERO = 14 ) 
AND A.PADRON NOT IN (SELECT N.PADRON FROM NOTAS N 
				WHERE N.NOTA >= 4 
				AND N.CODIGO = 71 AND N.NUMERO = 15 );
\end{minted}

\subsection*{19 - Obtener, sin repeticiones, todos los pares de padrones de alumnos tales que ambos alumnos rindieron la misma materia el mismo día. Devuelva también la fecha y el código y numero de la materia}

\begin{minted}
[
frame=leftline,
framesep=5mm,
baselinestretch=1.2,
]
{sql} 
SELECT DISTINCT N.PADRON PADRON, NOTA.PADRON PADRON2,
N.FECHA, N.CODIGO, N.NUMERO FROM TALLER_4.NOTAS N
INNER JOIN
TALLER_4.NOTAS AS NOTA
ON NOTA.PADRON < N.PADRON
AND NOTA.FECHA = N.FECHA 
AND N.CODIGO = NOTA.CODIGO 
AND N.NUMERO = NOTA.NUMERO;
\end{minted}


\subsection*{20 - Para cada departamento, devuelva su código, nombre, la cantidad de materias que tiene y la cantidad total de notas registradas en materias del departamento. Ordene por la cantidad de materias descendente}

\begin{minted}
[
frame=leftline,
framesep=5mm,
baselinestretch=1.2,
]
{sql} 
SELECT d.codigo, d.nombre, COUNT(DISTINCT n.numero) AS "cant_mat",
COUNT(*) AS "cant_not"
FROM notas n INNER JOIN materias m USING (codigo, numero) 
INNER JOIN departamentos d USING (codigo)
GROUP BY d.codigo, d.nombre
ORDER BY COUNT(DISTINCT n.numero) DESC
\end{minted}

\begin{minted}
[
frame=leftline,
framesep=5mm,
baselinestretch=1.2,
]
{sql} 
WITH cant AS (
	SELECT codigo, COUNT(numero) AS cant FROM materias GROUP BY codigo
), notasD AS (
	SELECT codigo, COUNT(notas) AS cant FROM notas GROUP BY codigo
)
SELECT d.codigo,d.nombre, c.cant, n.cant
FROM departamentos d, cant c, notasD n
WHERE d.codigo = c.codigo AND d.codigo=n.codigo
ORDER BY c.cant DESC

\end{minted}

\subsection*{21 - Para cada carrera devuelva su nombre y la cantidad de alumnos inscriptos. Incluya las carreras sin alumnos}

\begin{minted}
[
frame=leftline,
framesep=5mm,
baselinestretch=1.2,
]
{sql} 
SELECT c.nombre, COUNT(i.padron)
FROM carreras c LEFT OUTER JOIN  inscripto_en i USING (codigo)
GROUP BY c.codigo, c.nombre
\end{minted}

Es importante contar un atributo que no tenga nulos, así lo cuentan.

\subsection*{22 - Para cada alumno con al menos tres notas, devuelva su padrón, nombre, promedio de notas y mejor nota registrada}

\begin{minted}
[
frame=leftline,
framesep=5mm,
baselinestretch=1.2,
]
{sql} 
SELECT a.padron, a.nombre, AVG(n.nota) AS promedio, MAX(n.nota) AS mejor_nota
FROM alumnos AS a, notas AS n
WHERE a.padron = n.padron
GROUP BY a.padron, a.nombre
HAVING COUNT(*) >= 3
\end{minted}

\begin{itemize}
\item Condiciones que se aplican grupo a grupo no van en el WHERE, ya que viene en una etapa después.
\item Puedo evaluar cosas que después no devuelvo en el HAVING.
\end{itemize}


\subsection*{23 - Obtener el código y numero de la o las materias con mayor cantidad de notas registradas.}

\begin{minted}
[
frame=leftline,
framesep=5mm,
baselinestretch=1.2,
]
{sql} 
SELECT n.codigo, n.numero, COUNT(*)
FROM notas n
GROUP BY n.codigo, n.numero
HAVING COUNT(*) >= ALL (
    SELECT COUNT(*)
    FROM notas n
    GROUP BY n.codigo, n.numero
)
\end{minted}

\begin{minted}
[
frame=leftline,
framesep=5mm,
baselinestretch=1.2,
]
{sql} 
WITH cantidades AS (SELECT n.codigo, n.numero, COUNT(*) AS cant
	FROM notas n
	GROUP BY n.codigo, n.numero)
SELECT codigo, numero, cant
FROM cantidades c
WHERE c.cant = (SELECT MAX(cant) FROM cantidades)
\end{minted}


\begin{itemize}
\item No es valido ordenarlo y devolver el primero, puede darse el caso de que varios coincidan con el mejor valor. Queremos todas con la mayor cantidad.
\item No esta permitido realizar un MAX del COUNT en el HAVING.
\item No podemos anidar funciones de agregación.
\end{itemize}

\subsection*{24 - Obtener el padrón de los alumnos que tienen nota en todas las materias}

\begin{minted}
[
frame=leftline,
framesep=5mm,
baselinestretch=1.2,
]
{sql} 

SELECT padron
FROM alumnos a
WHERE NOT EXISTS (
    SELECT * 
    FROM materias m 
    WHERE NOT EXISTS (
        SELECT * 
        FROM notas n
        WHERE n.padron = a.padron
		      AND n.codigo = m.codigo
		      AND n.numero = m.numero
        )
)

\end{minted}

\begin{itemize}
\item Es una división.
\item Buscamos los alumnos con nota en todas las materias, tal que para los alumnos no exista materia en la que no tiene nota.
\item Alumnos para los que no existe una materia en la que no existe una nota de ese alumno en esa materia
\end{itemize}



\begin{minted}
[
frame=leftline,
framesep=5mm,
baselinestretch=1.2,
]
{sql} 

SELECT padron
FROM alumnos a
WHERE NOT EXISTS (
    (SELECT codigo, numero FROM materias)
    EXCEPT
    (SELECT codigo, numero FROM notas n WHERE n.padron=a.padron)
)

\end{minted}


\begin{itemize}
\item Primer conjunto: todas las materias
\item Segundo conjunto: todas las materias en las que un alumno tiene nota
\end{itemize}


\begin{minted}
[
frame=leftline,
framesep=5mm,
baselinestretch=1.2,
]
{sql} 


SELECT padron
FROM notas n
GROUP BY n.padron
HAVING COUNT( DISTINCT(to_char(codigo,'fm00') ||'.'|| to_char(numero,'fm00') ))
       =
       (SELECT COUNT(*) FROM materias)

\end{minted}


\begin{itemize}
\item Contar cuantas materias hay y cuento en cuantas materias tiene nota. Tiene que dar lo mismo
\item El DISTINCT solo recibe un atributo, no es estandar que tome dos.
\end{itemize}

\subsection*{25 - Obtener el promedio general de notas por alumno (cuantas notas tiene en promedio un alumno), considerando únicamente alumnos con al menos una nota}

\begin{minted}
[
frame=leftline,
framesep=5mm,
baselinestretch=1.2,
]
{sql} 


SELECT CAST(COUNT(*) AS DECIMAL) / COUNT(DISTINCT padron)
FROM notas n;

\end{minted}

Cuidado con realizar la división entera, esta trunca el resultado.
\newpage