\section{Practica: Resolución de consultas}
\begin{itemize}
\item Una consulta SQL se transforma en un árbol de consultas. Buscamos ver que árbol utilizar.
\item Para calcular los costos nos interesa conocer los meta datos
    \begin{itemize}
    \item n: cuantas filas tiene
    \item B: cuantos bloques tiene
    \item F: factor de bloque, cuantas filas entran en un bloque
    \item V: variabilidad de un atributo en una tabla (valores distintos)
    \item Height: altura del índice
    \end{itemize}
\item Buscamos realizar las selecciones lo antes posible, reemplazar productos cartesianos y selecciones por juntas, proyectar atributos lo antes posible priorizando la seleccion y realizar las juntas más restrictivas primero.
\item El objetivo es reducir el tamaño de relaciones intermedias
\end{itemize}


\subsection*{Selección}

\begin{itemize}
\item File Scan: Costo B(R), se recorre toda la tabla.
\item Index Scan: height(idx) + variante dependiendo del índice
\end{itemize}

Siempre acordarse de redondear hacia un valor entero.

\subsection*{Junta}
\begin{itemize}
\item Loops anidados: La cantidad de memoria ayuda mucho en los loops anidados
\item Único loop: Es cuando tenemos un índice que podemos aprovechar. Arranca por el que no tiene el índice como para usar.
\item Junta Hash: Surge de leer 3 veces ambas tablas 3 (B(R) + B(S))
\end{itemize}


Si es la misma tabla y se utiliza el mismo atributo de junta, solo carga la tabla una vez. No tendría sentido cargar devuelta la tabla.


\subsection*{Pipelining}
\begin{itemize}
\item Se concatena la salida de un operador con la entrada de otro. El resultado de un operador será la entrada de otro operador.
\item Puedo llegar a hacer un ahorro en el costo (puede aumentar el uso de memoria). Por ejemplo no tener que hacer una proyección extra.
\item En vez de procesar completamente un operador y luego el siguiente, puede irse procesando parcialmente el resultado a medida que se arma
\end{itemize}



\subsection*{Estimación de cardinalidad}
\begin{itemize}
\item Para poder usar las fórmulas de operadores que trabajan con resultados de otros operadores, precisamos conocer la información que nos dan los metadatos
\item Precisamos estimar la cantidad de filas n(R) 
\item Algunas operaciones modifican F(R), se recalcula B(R)
\end{itemize}



\subsection*{Histogramas}
\begin{itemize}
\item Guardan la frecuencia de ciertos valores.
\item Mantener actualizado el histograma es costoso.
\item En general tiene una buena performance
\end{itemize}






Se trata de no utilizar operaciones en las dos tablas a la vez, se trabaja en una, se hace el join y se sigue trabajando.


\newpage