\section{Practica: Neo4J}

\begin{itemize}
\item Orientada a grafos
\item Forma distinta de modelar.
\item Aplica a una gran cantidad de problemas, relaciones, distancias
\item Se tienen 2 elementos, nodos y arcos
\item Las ventajas son
    \begin{itemize}
    \item Que se pueden ver patrones de nodos conectados entre si
    \item Encontrar caminos entre nodos
    \item La ruta mas corta
    \item Medidas asociadas al grafo (centralidad,proximidad,periferia)
    \end{itemize}
\end{itemize}



\subsection*{Arcos}
\begin{itemize}
\item Tienen un único tipo
\item Pueden tener propiedades
\item Tienen una dirección si o si
\item Vinculan 2 nodos o con si mismo
\end{itemize}


\subsection*{Consultas}

\begin{verbatim}
MATCH patron
[WHERE filtros]
RETURN respuestas
[ORDER BY expresiones]
[SKIP cant LIMIT cant]  
\end{verbatim}


\begin{itemize}
\item patrón:estructura que quiero
\item filtros, como sql
\end{itemize}


\subsection*{Patrón de nodos}
\begin{itemize}
\item (alias:etiqueta:{filtros})
\item Es case-sensitive
\item Manejo de labels: No tengo que definir esquemas
\end{itemize}


\subsection*{Taller}
\subsubsection*{1 Muestre en orden alfabetico, las 10 primeras pelıculas del genero (genre) Science Fiction}
 
\begin{verbatim}
MATCH (n:Movie)
WHERE n.genre = "Science Fiction"
RETURN n
ORDER BY n.title
LIMIT 10
\end{verbatim}

\subsubsection*{2 Muestre 20 peliculas del genero “Drama” y cuyo estudio contenga la palabra “Pictures”, devolviendo el titulo de la pelicula y el nombre del estudio}

\begin{verbatim}
MATCH (n:Movie)
WHERE n.genre = "Drama" AND n.studio=~".*Pictures.*"
RETURN n.title, n.studio
LIMIT 20
\end{verbatim}

Otra opción es con n.studio CONTAINS "Picture"

\subsubsection*{3 Muestre el nombre y fecha de nacimiento de 20 personas que sean actores Y directores, y que tengan fecha de nacimiento registrada}

\begin{verbatim}
MATCH (n:Actor:Director)
WHERE n.birthdate IS NOT NULL
RETURN n.name, n.birthdate
LIMIT 20
\end{verbatim}

\subsubsection*{4 Muestre el grafo de las películas filmadas por Oliver Stone}

\begin{verbatim}
MATCH (os:Director{name: 'Oliver Stone'})-[d:DIRECTED]-(pelis:Movie)
RETURN os, pelis LIMIT 20
\end{verbatim}

\begin{verbatim}
MATCH (p:Director{name=”Oliver Stone”}) - [d:DIRECTED] -> (m:Movie)
RETURN p,d,m
\end{verbatim}

\subsubsection*{5 Muestre el grafo de todos los co-actores de Tom Hanks}

\begin{verbatim}
MATCH (th:Actor{name=”Tom Hanks”}) -[a: ACTS_IN]-> (m: Movie),
	  (m) <- [a2: ACTS_IN]- (ca:Actor)
RETURN  th, a, a2, ca
\end{verbatim}


\subsubsection*{6 Muestre 10 actores que están a distancia 4 de Kevin Bacon}

\begin{verbatim}
MATCH (n:Actor)-[*4]-(kb:Actor{name:'Kevin Bacon'})
RETURN n
LIMIT 10
\end{verbatim}

\subsection*{Alias de subgrafos}

\begin{itemize}
\item Para usar funciones sobre los subgrafos
    \begin{itemize}
    \item Length
    \item Relationships
    \item Nodes
    \end{itemize}
\item Se pueden usar como booleanos
\item Se pueden poner varios subgrafos en el MATCH
\item Se pueden tener OPTIONAL MATCH que es similar a un outer join
\item Caminos mas cortos, son funciones que reciben un subgrafo: shortestPath, allShortestPaths
\end{itemize}


\subsubsection*{7 Muestre las películas que Clint Eastwood dirigió y en las que no actuó}

\begin{verbatim}
MATCH (d:Actor:Director{name:'Clint Eastwood'})-[:DIRECTED]->(m:Movie)
WHERE NOT (d)-[:ACTS_IN]->(m) 
RETURN d,m
\end{verbatim}



\subsubsection*{8 Muestre un camino más corto de Meg Ryan a Kevin Bacon.}

\begin{verbatim}
MATCH r = shortestPath( (n{name:'Meg Ryan'})-[*]-(m{name:'Kevin Bacon'})  )
RETURN r
\end{verbatim}

\subsubsection*{9 Muestre los actores que trabajaron en la trilogía completa de The Matrix ("The Matrix", "The Matrix Reloaded" y "The Matrix Revolutions")}

\begin{verbatim}
MATCH (a:Actor)-[:ACTS_IN]-(m:Movie{title:'The Matrix'}),
      (a)-[:ACTS_IN]-(m2:Movie{title:'The Matrix Reloaded'}),
      (a)-[:ACTS_IN]-(m3:Movie{title:'The Matrix Revolutions'})
RETURN a
\end{verbatim}

Otra opción

\begin{verbatim}
MATCH (a:Actor)-[:ACTS_IN]-(m:Movie{title:"The Matrix"})
WHERE (:Movie{title:"The Matrix Revolutions"})<-[:ACTS_IN]-(a)-[:ACTS_IN]->(:Movie{title:"The Matrix Reloaded"})
RETURN a
\end{verbatim}

\subsection*{Funciones de agregación}
\begin{itemize}
\item Ya incluirlas en el RETURN nos devuelve lo que pedimos
\item Si devuelvo algo sin una función de agregación, se agrupa por esta propiedad.
\item WITH reemplaza al return (Seria como el HAVING en sql)
\item Permite manipular resultados antes de pasar a la siguiente parte de la consulta
\item Se puede usar para filtrar agregación
\item Con WITH tengo que definir alias a lo que voy a estar trabajando con AS.
\end{itemize}



\subsubsection*{10 Obtenga el nombre de los 10 directores que han trabajado con más actores (y la cantidad de actores)}

\begin{verbatim}
MATCH (d:Director)-[:DIRECTED]-(m:Movie), (a:Actor)-[:ACTS_IN]-(m)
RETURN d.name, COUNT(DISTINCT a) AS cantidad
ORDER BY cantidad DESC
LIMIT 10
\end{verbatim}


\subsubsection*{11 ¿Quien/es es/son el/los actor/es más joven/es de la base de datos?}
\begin{verbatim}
MATCH (a:Actor) 
WITH MAX(a.birthday) AS mas_joven
MATCH (j:Actor) 
WHERE j.birthday=mas_joven
RETURN j
\end{verbatim}

\subsubsection*{12 Devuelva el nombre de personas que hayan actuado en al menos 10 películas y hayan dirigido al menos 5}

\begin{verbatim}
MATCH (person:Person)-[acts:ACTS_IN]->(:Movie)
WITH COUNT(acts) as cant_actuaciones, person
MATCH (person:Person)-[directs:DIRECTED]->(:Movie)
WITH COUNT(directs) as cant_direcciones, cant_actuaciones, person
WHERE cant_direcciones >= 5 AND cant_actuaciones >= 10
RETURN person.name, cant_direcciones, cant_actuaciones
\end{verbatim}


\subsubsection*{13 ¿A qué distancia se encuentra el actor más viejo de Kevin Bacon?}
\begin{verbatim}
MATCH (a:Actor) WITH MIN(a.birthday) AS mas_viejo
MATCH (j:Actor) WHERE j.birthday=mas_viejo
MATH g=shortestPath((j)-[*]-(k:Actor{name:'Kevin Bacon'})
RETURN length( g )
\end{verbatim}


\subsection*{ABM}
\begin{itemize}
\item No es lo mismo crear nodos que arcos
\item Para nodos es CREATE y el patron del nodo
\item Para arcos es con MATCH para vincular los nodos y un CREATE para crear el arco, por cada arco
\end{itemize}


\newpage