
\section{Practica: Taller Valores Nulos}

\begin{itemize}
\item La política del estándar del NULL es ignorarlo en las consultas. Si son todos NULL, ahi da NULL.
\item Código para crear las tablas e insertar los valores.
\end{itemize}
\begin{minted}
[
frame=leftline,
framesep=5mm,
baselinestretch=1.2,
]
{sql} 
CREATE TABLE  clientes (
  cod_cliente integer             NOT NULL,
  nombre      character varying   NOT NULL,
  saldo       float8              NULL    ,
  localidad   character varying   NULL  
)
\end{minted}


\begin{minted}
[
frame=leftline,
framesep=5mm,
baselinestretch=1.2,
]
{sql} 
INSERT INTO clientes(
	cod_cliente, nombre, saldo, localidad)
	VALUES (10, 'Juan' , 20    , NULL ),
	       (20, 'Pablo', 40    , NULL ),
	       (30, 'Ana'  , NULL  , 'Mar del Plata')
	;
\end{minted}

\medskip
2) Las preguntas pasan a tener distinto significado con valores nulos, cambia la semántica.
\begin{enumerate}
\item El saldo total de los clientes es 60
\item Matemáticamente el saldo de Ana tiene que ser 0. Aun ignorando los valores nulos estos afectan en cierta forma.
\item Depende de la politica. Da 30, ignora los valores nulos. Para que sea correcto Ana deberia de tener saldo 30. (Llegamos a un absurdo, tiene 0 en uno, 30 en este)
\item Maximo 40, minimo 20. Estan condicionando el saldo de Ana dentro de estos limites.
\end{enumerate}

\medskip
3) En el orden hay 3 políticas, los deja al final, al comienzo, o lo deja donde lo encontró. Esto tiene implicaciones también.

\medskip
4) La consulta nos da en realidad los clientes que \textbf{sabemos que viven} en Mar del Plata, no nos da todos.
\begin{minted}
[
frame=leftline,
framesep=5mm,
baselinestretch=1.2,
]
{sql} 
SELECT *
FROM clientes
WHERE localidad ILIKE 'mar del plata';
\end{minted}

\medskip
5) El conjunto mas su complemento no me pueden dar el universo. Esto es por los valores nulos.
\begin{minted}
[
frame=leftline,
framesep=5mm,
baselinestretch=1.2,
]
{sql} 
SELECT *
FROM clientes
WHERE localidad ILIKE 'mar del plata'
UNION
SELECT *
FROM clientes
WHERE localidad NOT ILIKE 'mar del plata';
\end{minted}

\medskip
\begin{itemize}
\item NULL esta en el rango? Da NULL, no lo toma.
\item Hay que tener mucho cuidado al pasar de un diseño sin valores nulos a uno con nulos.
\end{itemize}

\medskip
8) Una tautología puede dar falso debido a los valores nulos
\begin{minted}
[
frame=leftline,
framesep=5mm,
baselinestretch=1.2,
]
{sql} 
--  E8 (Una Tautologías puede dar Falso!) 
--  idea  adaptad de : C. J. Date-SQL and Relational Theory-O'Reilly (2012)
-- Ejecute la sentencia:
DELETE FROM taller_05.fabricas WHERE cod_fabrica = 40;
-- Luego obtenga todos los pares (cod cliente, cod fabrica) tales que
-- no se encuentren en la misma ciudad, o bien la fabrica no se encuentre
-- en Rosario. Analice el resultado e indique si el mismo se contrapone a la realidad.
-- veamos esta consulta:
SELECT cod_cliente, cod_fabrica
FROM taller_05.clientes c , taller_05.fabricas f
WHERE c.localidad <> f.localidad
   OR f.localidad <> 'ROSARIO'
   
-- estudiemos estos dos predicados:  
--   (c.localidad <> f.localidad)  OR  (f.localidad <> 'ROSARIO')  
-- cual es la respuesta?
-- pero si en el mundo real suponemos :
--  H0: la fabrica 50 esta en Rosarios
--  H1: la fabrica 50 no esta en Rosario
-- observar que  en el mundo real no existe otra hipotesis posible!
-- entonces:
-- si se cumple la la H0: el primer predicado es Verdadero y el segundo predicado es Falso 
-- y me da ANA (<30 , 50>)
-- si se cumple la H1: el primer predicado es Falso y el segundo predicado es Verdadero 
-- y en este caso me deberia dar  a los tres clientes con la f 50
-- por lo tanto este predicado es una tautologia y siempre sera verdadero.
-- Sin embargo SQL nos devuelve un valor  que no se corresponde con el mundo real
\end{minted}
\newpage


