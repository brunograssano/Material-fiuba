\section{Practica: Costos}

\begin{itemize}
\item Medimos en unidades de I/O a disco
\item Cualquier otro proceso que se haga en comparación tiene un costo ínfimo.
\item Cuando tenemos varias tablas en un join, empezamos por la que nos va a dejar menos resultados.
\item Tabla mas pequeña es la que usamos en el ciclo externo (algoritmo para la junta)
\item Es para disminuir el costo
\item Analizar arboles que estén ramificados a izquierda o derecha (left-deep o right-deep) para acotar las posibilidades.
\item \textbf{R-trees}: Indexan objetos geométricos a través de su minimum boungind rectangle (MBR) Pongo en las hojas del R-tree los MBR de las partes que contienen cada hoja.
\item \textbf{Quad-trees}: El espacio se descompone en cuadrantes disjuntos. Cada cuadrante lo subdividimos si tiene un mínimo de puntos. Se usa para indexar puntos geométricos o datos rater en dos dimensiones.
\end{itemize}

\subsection*{Índices}

\begin{itemize}
\item Su objetivo es acceder de forma mas rápida a la búsqueda de datos. También es hacer cumplir las reglas de negocio (Ejemplo no hay padrones duplicados. La estructura sirve para hacerla cumplir)
\item Cluster: El orden del índice con el orden de los datos coinciden. El orden en que vienen los datos del índice es como vienen los datos en el archivo
\item No cluster: Es cuando no coinciden los ordenes, se cruza del índice a los datos con el camino que toma otra entrada del índice.
\item La misma consulta se comporta de forma distinta dependiendo del tipo de índice.
\item Simples: Están creados sobre un único atributo. I(A), I(B)
\item Compuestos: Están creados por 2 o mas atributos. I(A,B) != I(B,A). El orden de los atributos dentro del índice es importante para la optimización de la consulta. Conviene poner en el primer lugar el atributo mas discriminante.
\item Pueden tener duplicados o no, con duplicados seria que nos manda a un lugar donde hay varias entradas. (Ejemplo índice por comuna, una comuna nos manda a una entra con muchos alumnos.)
\item Podemos indexar parcialmente las tablas. (De forma completa o densa significa que cada registro de la tabla tiene su representación en el índice)
\item El índice sirve al momento de acceder a la estructura física, si proyectamos lo perdimos, ya no nos sirve.
\end{itemize}









\newpage