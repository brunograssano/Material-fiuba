\documentclass[titlepage,a4paper]{article}

\usepackage{a4wide}
\usepackage[colorlinks=true,linkcolor=black,urlcolor=blue,bookmarksopen=true]{hyperref}
\usepackage{bookmark}
\usepackage{fancyhdr}
\usepackage[spanish]{babel}
\usepackage[utf8]{inputenc}
\usepackage[T1]{fontenc}
\usepackage{graphicx}
\usepackage{float}

\usepackage{amssymb} % simbolos matematicos

\usepackage{minted} % codigo
\usepackage[table,xcdraw]{xcolor} % para los colores en las tablas


\pagestyle{fancy} % Encabezado y pie de página
\fancyhf{}
\fancyhead[L]{Apuntes de BDD - BG}
\fancyhead[R]{2C2021}
\renewcommand{\headrulewidth}{0.4pt}
\fancyfoot[C]{\thepage}
\renewcommand{\footrulewidth}{0.4pt}

\begin{document}
\begin{titlepage} % Carátula
	\hfill\includegraphics[width=6cm]{logofiuba.jpg}
    \centering
    \vfill
    \Huge \textbf{Apuntes de Bases de datos}
    \vskip2cm
    \Large [7515] Base de Datos\\
    2C 2021
    \vfill
    \begin{tabular}{ | l | } % Datos del alumno
      \hline
      Grassano, Bruno \\ \hline
      bgrassano@fi.uba.ar \\ \hline
  	\end{tabular}
    \vfill
    \vfill
\end{titlepage}

\tableofcontents % Índice general

\newpage

\section*{Introducción}\label{sec:intro}
El presente archivo contiene los apuntes que fueron tomados a lo largo de la cursada de la materia Base de Datos (75.15).

La materia esta muy buena, sin embargo hay que tener en cuenta que es pesada debido a la gran cantidad de temas que abarca. 

Muy recomendable ir siguiendo la materia al día e ir haciendo los parcialitos (fueron 6 este cuatri) para llegar bien al parcial. Después para el final también recomiendo seguirla al día, ya que se puede evaluar cualquier tema de los vistos (ya sea teórico o practico)

\newpage
\section*{Clase 1}
\section{Desarrollo de software}
Desarrollar software es una tarea complicada. No existe la bala de plata (\textit{El famoso paper, No silver bullet - Brooks}). La complejidad se compone por:

\begin{center}
$complejidad = complejidad\_solucion + complejidad\_problema$ 
\end{center}

Para atacarla, se utiliza abstracción para eliminar lo irrelevante y amplificar lo esencial, agrupar y ocultar (ignorar detalles y evitar verlos), restricciones para simplificar el enfoque, y la visibilidad (desacoplamiento).

La complejidad no esta en la solución, esta en el problema. Por lo que hay que desarrollar un buen criterio para resolver el problema. Se utilizan los mecanismos de descomposición (descubrir), abstracción (pensar en la esencia), y establecer jerarquías (inventar \textit{generalizaciones. Ej Documentos comerciales - son facturas, tickets y demás - El documento comercial por si solo no existe}). La descomposición y la abstracción se van a usar siempre en relación al negocio.

Es fundamental entender el problema para proponer un diseño que no se quede corto y que no sea algo complejo que sume al problema. 

\section{Modelo de dominio}
El modelo de dominio vive dentro del problema.  Este tiene patrones de análisis/colaboración y reglas del negocio, se arman \textit{tests}. Una vez que no se rompan estos \textit{tests}, se tiene el modelo de negocio.

\begin{figure}[!htb]
    \centering
    \includegraphics[width=0.8\textwidth]{img/ModeloDeDominio.PNG}
\end{figure}

En la aplicación están los requisitos funcionales \textit{Casos de uso, user stories}, prototipos de interfaces, diagramas de colaboración y secuencia. Estos van aportando al modelo de negocio.

\begin{figure}[!htb]
    \centering
    \includegraphics[width=0.8\textwidth]{img/ProcesoModeloDeDominio.PNG}
\end{figure}

Antes de tener el modelo de negocio, se inicia con un modelo de dominio, se busca descubrir. A esto le sigue un modelo de análisis que representa la dinámica del modelo de dominio. A partir de esto se llega al diagrama de clases.

\begin{itemize}
\item El modelo de negocio tiene como objetivo entender en detalle el negocio y sus reglas, para esto utiliza patrones de análisis o colaboración.
\item Se diferencia del Modelo de diseño, ya que este ultimo tiene como objetivo implementar una solución al problema planteado en el análisis mas las restricciones impuestas por los requisitos no funcionales. Utiliza los patrones de diseño.
\end{itemize}


\section{Patrones de análisis o colaboración}

\begin{figure}[!htb]
    \centering
    \includegraphics[width=0.8\textwidth]{img/PatronesAnalisisColaboracion.PNG}
    \caption{Las diferentes instancias que tenemos.}
\end{figure}

\begin{itemize}
\item Suelen aparecer de a dos relaciones. 
\item Tenemos 12 patrones en total, estos se usan para armar el modelo de dominio.
\item La idea es descubrir en base a ellos siguiendo los pasos que se van a mostrar después.
\item En esta etapa del modelo de dominio, no hay que usar interfaces y herencia, solo hay que usar los patrones de colaboración.
\item \textit{En las resoluciones de ejercicios, indicar que patrón se esta utilizando.} 
\end{itemize}

\newpage
\subsubsection*{Actor-Rol}
\begin{figure}[!htb]
    \centering
    \includegraphics[width=0.6\textwidth]{img/PatronActorRol.PNG}
\end{figure}

\subsubsection*{GranLugar-Lugar}
\begin{figure}[!htb]
    \centering
    \includegraphics[width=0.6\textwidth]{img/PatronGranLugarLugar.PNG}
\end{figure}
\newpage
\subsubsection*{Item-ItemEspecifico}

\begin{figure}[!htb]
    \centering
    \includegraphics[width=0.6\textwidth]{img/PatronItemItemmEspecifico.PNG}
\end{figure}
\textit{Un titulo de una película puede ser un LineItem.}

\subsubsection*{Ensamble-Parte}
\begin{figure}[!htb]
    \centering
    \includegraphics[width=0.6\textwidth]{img/PatronEnsambleParte.PNG}
\end{figure}

\subsubsection*{Contenedor-Contenido}
\begin{figure}[!htb]
    \centering
    \includegraphics[width=0.6\textwidth]{img/PatronContenerdorContenido.PNG}
\end{figure}
\newpage
\subsubsection*{Grupo-Miembro}
\begin{figure}[!htb]
    \centering
    \includegraphics[width=0.6\textwidth]{img/PatronGrupoMiembro.PNG}
\end{figure}

\subsubsection*{Rol-Transacción}

\begin{figure}[!htb]
    \centering
    \includegraphics[width=0.6\textwidth]{img/PatronRolTransaccion.PNG}
\end{figure}
Le da la dinámica al negocio. \textit{Un orden de compra como registro de la compra}
\subsubsection*{Transacción-Lugar}

\begin{figure}[!htb]
    \centering
    \includegraphics[width=0.6\textwidth]{img/PatronLugarTransaccion.PNG}
\end{figure}
Da dinámica del negocio también.
\newpage
\subsubsection*{Transacción-ItemEspecifico}

\begin{figure}[!htb]
    \centering
    \includegraphics[width=0.6\textwidth]{img/PatronItemEspecificoTransaccion.PNG}
\end{figure}

\subsubsection*{ItemEspecifico-LineItem}
\begin{figure}[!htb]
    \centering
    \includegraphics[width=0.6\textwidth]{img/PatronItemEspecificoLineItem.PNG}
\end{figure}

\subsubsection*{TransaccionCompuesta-LineItem}
\begin{figure}[!htb]
    \centering
    \includegraphics[width=0.6\textwidth]{img/PatronTransaccionCompuestaLineItem.PNG}
\end{figure}

\newpage
\subsubsection*{Transaccion-TansaccionCronologica}

\begin{figure}[!htb]
    \centering
    \includegraphics[width=0.6\textwidth]{img/PatronTransaccionTransaccionCronologica.PNG}
\end{figure}

\subsection*{Pasos para su búsqueda}
\begin{enumerate}
\item Preguntarse quiénes realizan tareas en este dominio/negocio, de esta forma se identifican actores y se les asigna un rol
\item Preguntarse qué tareas realizan, así se asignan transacciones
\item Preguntarse sobre qué objetos de negocio se realiza estas transacciones, así se asignan items específicos
\item Preguntarse qué transacciones deben realizarse antes que otras, así se asignan transacciones cronológicas
\item Preguntarse dónde se realizan las transacciones, así se asigna un lugar a estas transacciones
\item Preguntarse si los Items tienen identificación o no, para identificar si se usa Lines Items o no
\item No confundir transacciones de negocio con transacciones de sistema.
\end{enumerate}

La base que tiene que quedar en todo modelo de dominio es la siguiente.

\begin{figure}[!htb]
    \centering
    \includegraphics[width=0.8\textwidth]{img/BaseModeloDeDominio.PNG}
\end{figure}

\section{Reglas de negocio}

\begin{itemize}
\item Las reglas de negocio son las restricciones que gobiernan las acciones dentro de un dominio de negocio. En el modelo se traducen en reglas de colaboración. Estas se incorporan al modelo en restricciones a ser probadas en la colaboración entre los distintos objetos del modelo.
\item Si no se ubican en el modelo, el modelo esta incompleto.
\end{itemize}

Hay 5 tipos de reglas de negocio: 
\begin{itemize}
\item De tipo: \textit{Un medicamento puede ser cargado solo en un container refrigerado.}
\item De multiplicidad: \textit{Un pallet refrigerado puede contener hasta 10 cajas.}
\item De propiedad: Se refiere a validación o comparaciones. \textit{Un pago debe registrar un número válido de tarjeta de crédito, la temperatura de un container refrigerado debe ser menor a 0 grados centígrados.}
\item De estado: \textit{Una orden no debe ser entregada si antes fue cancelada.}
\item De conflicto: \textit{Un vuelo no puede ser programado en una puerta en un mismo horario que otro vuelo, un producto no puede ser sumado a una orden de compra de un menor de edad si la venta está prohibida a menores.}
\end{itemize}

\begin{figure}[!htb]
    \centering
    \includegraphics[width=0.8\textwidth]{img/ReglasDeNegocioSegunPatrones.PNG}
\end{figure}

\textit{Por ejemplo, la regla de tipo de Granlugar-lugar, la valida el lugar}. Si tiene ambos, involucra a ambos objetos.

\section*{Clase 2}
\section{Practica: Ejercicio AFA y Taller I}
% libro ] Fundamentals of Database Systems,Elmasri, S. Navathe

\subsubsection*{Conceptos}
\begin{itemize}
\item Cardinalidad: primer termino mínimo, segundo máximo en la cardinalidad (de ambos lados)
\item Conjunto de entidad
\item Entidad
\item Atributos (claves, otros)
\item Semántica, el significado de la interrelación
\item Interrelación
\item Conjunto de interrelación
\end{itemize}


\subsubsection*{Notas}

\begin{itemize}
\item La cardinalidad máxima tiene que estar si o si, se describe con esta siempre. Para enriquecer se agrega la cardinalidad mínima
\item Una entidad débil puede ser a su vez una entidad fuerte de otra.
\item No puedo tener un 0 como cardinalidad mínima en una entidad fuerte por la dependencia existencial.
\item Prestar atención a los sustantivos y a los verbos. Para ver cuales son las entidades mas visibles, se puede contar cuantas veces aparece un sustantivo.
\item En el diagrama los sustantivos los ponemos en singular, cuando lo pasamos a la tabla los ponemos en plural generalmente.
\item En las relaciones ternarias usamos solo las cardinalidades máximas.
\item Por mas que se tenga la misma cardinalidad como mínima y máxima, se ponen ambas (1,1)
\item Siempre tienen que estar los identificadores únicos.
\item Tratar de no usar abreviaturas.
\item Lo que no entra en el diagrama, se anota aparte (en el diccionario de datos)
\end{itemize}



\subsubsection*{AFA}
\begin{itemize}
\item Estamos modelando los torneos solamente. No irse de la narrativa de estos.
\item Equipo y encuentro es una relación particular, una binaria que tratamos como ternaria. 
\item En este caso usamos 2 binarias para evitar esa relación particular
\end{itemize}


\begin{figure}[!htb]
    \centering
    \includegraphics[width=0.8\textwidth]{img/EjercicioAFA.PNG}
\end{figure}

\subsubsection*{Taller}

\begin{itemize}
\item Usamos schemas para ordenar las tablas lógicamente. (tiene otros motivos de seguridad y demás también)
\item Las primeras lineas usarlas para documentación. Nombre del archivo, resumen del script, fecha creado y modificado, y autor. (medio innecesario los últimos tres si se usa git)
\end{itemize}


\begin{minted}
[
frame=leftline,
framesep=5mm,
baselinestretch=1.2,
]
{sql}
DROP TABLE IF EXISTS paradas -- Buena practica

CREATE TABLE paradas (
	cod_parada integer NOT NULL,
    longitud numeric NOT NULL,
    latitud numeric NOT NULL,
    tipo_parada varchar,
    calle varchar,
    altura integer,
    entre1 varchar,
    entre 2 varchar
);

ALTER TABLE paradas -- Se cambia el dueño de la tabla
	OWNER TO postgres;
\end{minted}

\begin{itemize}
\item Al hacer consultas siempre tratar de poner un limit, en caso de tener tablas grandes. \textit{Ejemplo:} \mint{sql}|SELECT ... limit 10;|
\item Al trabajar con $csv$ tener cuidado con los '.' y ',' cuando tenemos numeros. Revisar tambien los simbolos que no sean estandar \textit{por ejemplo la 'ñ'}.
\item Para exportar los archivos, irían a un lugar publico. \textit{En Windows: C:$\backslash$Users$\backslash$Public}
\end{itemize}


\textit{Articulo recomendado: \href{https://martinfowler.com/articles/evodb.html}{Evolutionary Database Design - Martin Fowler}}


\newpage

\section*{Clase 3}
\section{Practicas del código}

\begin{itemize}
\item Hay que estar orgullosos del código que hacemos, tratar de hacerlo de la mejor forma posible. El costo de poseer código no mantenible es muy alto en el tiempo, la productividad se reduce rápidamente.
\item \textit{Lo podemos medir con los WTF/minute (clean code)}
\item Buscamos que nuestro código sea mantenible, claro, legible, flexible, y simple.
\item Siguen solo algunas practicas mencionadas de los primeros capítulos del libro \textit{Clean Code de Robert C. Martin}, si no lo leíste todavía, te lo recomiendo.
\end{itemize}


\subsection*{Nombres}
\begin{itemize}
\item Usar nombres claros antes que comentarios.
\item Usar nombres pronunciables
\item Tienen que revelar la intención que quieren transmitir.
\item Usar constantes antes que valores fijos
\item Cuidado con las notaciones y prefijos
\item No usar nombres con na letra distinta porque ya se uso \textit{class - klass}
\item Usar nombres del dominio de la solución, pero también de cuestiones mas de programación! \textit{AccountVisitor}, el Visitor indica el patrón de diseño.
\end{itemize}

\subsection*{Funciones}
\begin{itemize}
\item Devuelta los nombres!
\item Una solo motivo
\item Lo mas chicas posibles. \textit{Clean code dice máximo 20 lineas}
\item Evitar switch. Encapsularlo si es necesario, que estén solo una vez.
\item Cuidado con los argumentos, menos es lo mejor.
\item Evitar booleanos en argumentos o flags. Indica que la función hace mas de una función.
\item Evitar argumentos de entrada como salida.
\end{itemize}

\subsection*{Comentarios}
\begin{itemize}
\item Dependen del caso
\item El código debería de auto documentarse.
\item Tratar de evitarlos
\item Hay algunos que son buenos, informativos, o que explican la intención de algo.
\item Pasan a ser parte del código, hay que mantenerlo si cambia o  se vuelven desinformativos.
\item Los TODO son temporales y a corto plazo.
\item Revisar el código comentario. ¿Lo borro o no?
\end{itemize}

\subsection*{Tests unitarios}
Tienen que cumplir F.I.R.S.T. Solo los menciono, para mas detalles buscarlo o revisar los apuntes de Algoritmos III.
\begin{itemize}
\item \textbf{F}ast
\item \textbf{I}ndependent
\item \textbf{R}epeatable
\item \textbf{S}elf-Validating
\item \textbf{T}imely
\end{itemize}

Se recomiendan para expandir las lecturas recomendadas de Algoritmos III.
\begin{itemize}
\item \href{https://esj.com/articles/2012/09/24/better-unit-testing.aspx}{Better unit testing}
\item \href{http://fluxens.com/unittesting.html}{Unit testing}
\end{itemize}


\section{Principios SOLID}
\begin{itemize}
\item Buscan llegar a un buen diseño. Cuando tenemos uno malo, el código se vuelve rígido (dificultad de cambiar), frágil (toco y se rompe), inmóvil (imposibilidad de reutilizar software) y viscoso (el ambiente de desarrollo se vuelve lento e ineficiente)
\item Las características de un buen diseño es una alta cohesión y un bajo acoplamiento.
\item Lectura recomendada \href{https://github.com/7510-tecnicas-de-disenio/material-clases/blob/master/Solid/Principles_and_Patterns.pdf}{Principles and patterns}
\end{itemize}

\subsection*{Single Responsibility}
Tienen que tener una sola responsabilidad. Una sola razón de cambio. Un solo nivel de abstracción.

\subsection*{Open Closed}
Abierto para la extensión, cerrado para la modificación. No debe de ser necesario tocar el código que ya esta para extender/agregar funcionalidad. Va de la mano con el polimorfismo. Queremos que nuestros diseños sean simples, para ser mantenible y entendible.  No podemos tener un diseño Open-Closed para todo el código, tenemos que tenerlo para lo que sabemos que puede venir un cambio.

\textit{Tratar de atrapar la primera bala.} Quizás al comienzo no sabemos que es lo va a cambiar, entonces puede convenir un código simple antes que preparar todo con Open-Closed. Cuando llegue el cambio, hacer refactors y ahí aplicar el principio.

Un \textit{if} ya elimina el principio, porque separa en A y en B, y puede aparecer una tercera.

\subsection*{Liskov Substitution}
Las clases padres deberían de poder utilizarse a través de las clases hijas, y no debería notar la diferencia. Se puede romper cuando una clase no es realmente el \textit{es un} del padre. \textit{El ejemplo de los patos y el pato de goma o de la elipse y el circulo.}

\subsection*{Interface Segregation}
Tratar de mantener interfaces simples, lo mas atómicas posibles. Dar a conocer solo lo que necesitan y nada mas. 

Los clientes no deberían de estar forzados a depender de interfaces que no usan.

\subsection*{Dependency Inversion}

No depender directamente de una clase o sistema, poner una interfaz o capa de abstracción en el medio. Evitar que el código de alto nivel dependa de cosas de bajo nivel, hay que invertir esta lógica.

Generalmente si se cumple este, se esta cumpliendo el Open-Closed.

\section*{Clase 4}
\section{Practica: Modelo relacional}
\textit{Vamos a ver como mapear de un modelo de conceptual de Chen al relacional.
Obtener criterios para decidir cual es conveniente entre dos o mas modelos posibles.}

\begin{itemize}
\item Las entidades fuertes del conceptual pasan a ser entidades del relacional. 
\item La primary key pasa al relacional subrayada también,
\item Las relaciones binarias pasan también, tiene ambas ids como clave. Se subrayan cortado como foreign key (además del normal)
\item Si una interrelación tiene un atributo, va adentro de la relación.
\item La relación es un concepto matemático, la interrelación no (es mas conceptual).
\end{itemize}


\textbf{Entidades} A(\underline{idA},...) B(\underline{idB},...)

\medskip

\textbf{Relación} R(\underline{\underline{i}d\underline{A}},\underline{\underline{i}d\underline{B}})

\medskip

\begin{itemize}
\item En las relaciones unarios, para el id de la otra entidad (mismo tipo) se le pone otro nombre cambiado arbitrariamente.
\item En el caso binario 1:N (donde el valor mínimo de una es 0), como primary key de la relación usamos el id de la relación que tiene (1,1). Si ponemos a los dos, no seria mínima (seria una super clave). Hay que analizar también si conviene mover el atributo de la relación en la que tiene (1,1).
\end{itemize}

\medskip

\textbf{Entidades}
A(\underline{idA},...)
B(\underline{idB},...)

\medskip

\textbf{Relación}
R(\underline{i}d\underline{A},\underline{\underline{i}d\underline{B}})

\medskip

\textbf{Pasa a estar como}
B(\underline{idB},...,\underline{i}d\underline{A})


\begin{itemize}
\item Esto es para minimizar la cantidad de tablas.
\item Si tenemos el caso binario 1:N con opción de ninguno, la relación es necesaria, no se puede mover a una entidad. No esta bien ingresar algún carácter que indique que significa ninguno.
\end{itemize}

\medskip

\textbf{En el caso 1:1 hay varias opciones.}

\begin{itemize}
\item No olvidarse de definir las claves candidatas dependiendo el caso.
\item En las entidades débiles se tiene como PK el atributo discriminante y la clave de la entidad fuerte también (se marca como clave foránea además)
\end{itemize}

\medskip

\textbf{Relaciones ternarias}

\begin{itemize}
\item En el caso de las relaciones ternarias N:N:N se ponen como clave las tres ids, y las tres son foreign key.
\item En el caso 1:N:N se pone la relación con clave primaria de N y N, y la de 1 la tiene como foránea solo:
\end{itemize}
R(\underline{i}d\underline{A},\underline{\underline{i}d\underline{B}},\underline{\underline{i}d\underline{C}})

\medskip

\begin{itemize}
\item En 1:1:N las relaciones tenemos:
\end{itemize}
R(\underline{i}d\underline{A},\underline{\underline{i}d\underline{B}},\underline{\underline{i}d\underline{C}}) o 
R(\underline{\underline{i}d\underline{A}},\underline{i}d\underline{B},\underline{\underline{i}d\underline{C}})

\medskip

\begin{itemize}
\item No podemos ahorrarnos la tabla ya que estaríamos forzando a que todos los elementos tengan una asociación
\end{itemize}

\begin{itemize}
\item En 1:1:1 las claves candidatas son:
\end{itemize}
R(\underline{i}d\underline{A},\underline{\underline{i}d\underline{B}},\underline{\underline{i}d\underline{C}}) o 
R(\underline{\underline{i}d\underline{A}},\underline{i}d\underline{B},\underline{\underline{i}d\underline{C}}) o
R(\underline{\underline{i}d\underline{A}},\underline{\underline{i}d\underline{B}},\underline{i}d\underline{C})

\medskip

\begin{itemize}
\item Todas son claves candidatas (las 3), la PK es solo una. Las claves candidatas son igual de importantes, quiero que todas se verifiquen. Elijo una como PK, pero las otras siguen estando (UNIQUE).
\item En las especializaciones/generalizaciones (\textbf{caso disjunto y total (sin solapamiento)} - es uno o el otro), B es un A, y C es un A.
\item Tenemos las tres entidades, B y C con sus atributos propios y no comunes.
\item Los atributos comunes van en la tabla A. 
\item En A se puede agregar un atributo \emph{tipo} para decir en que tabla voy a poder buscar el resto de la información. (como es total, esto se puede asegurar que el tipo no sea nulo)
\end{itemize}


\subsection*{Taller}

\begin{itemize}
\item Si es una restricción de PK se le agrega el prefijo 'pk\_' Se hace también para las foreign keys (fn\_). 
\item Conviene definirlas como constraint, ya que es mas robusto que definirla en la creación inicial porque ante un error nos indica mas claramente el problema.
\end{itemize}

\medskip

Ejemplo de agregado de una clave como CONSTRAINT (parte 2 y 4):
\begin{minted}
[
frame=leftline,
framesep=5mm,
baselinestretch=1.2,
]
{sql}
ALTER TABLE paradas ADD CONSTRAINT pk_paradas PRIMARY KEY(cod_parada);

ALTER TABLE colectivos_por_parada ADD CONSTRAINT fk_colectivos_por_parada
FOREIGN KEY(cod_parada) REFERENCES paradas(cod_parada);
\end{minted}

\medskip

Intente provocar una violación a la restricción de integridad de entidad de
la tabla colectivos por parada a través de un INSERT.
\begin{minted}
[
frame=leftline,
framesep=5mm,
baselinestretch=1.2,
]
{sql}
INSERT INTO colectivos_por_parada VALUES(NULL,44)
\end{minted}

\medskip

Intente provocar una
violación a la restricción de integridad referencial definida, a través de un INSERT en la
tabla que considere apropiada.
\begin{minted}
[
frame=leftline,
framesep=5mm,
baselinestretch=1.2,
]
{sql}
INSERT INTO colectivos_por_parada VALUES(99999999,44);
\end{minted}

\medskip

Intente provocar una violación a la restricción de integridad referencial definida, a través de un DELETE en la tabla
que considere apropiada
\begin{minted}
[
frame=leftline,
framesep=5mm,
baselinestretch=1.2,
]
{sql}
DELETE FROM paradas WHERE cod_parada=1000086;
\end{minted}

\medskip

Intente modificar el código de la parada 1004561 por el código 1007800
utilizando un script de UPDATE. ¿Es posible hacerlo? No, es referida por la tabla colectivos\_por\_parada todavia.
\begin{minted}
[
frame=leftline,
framesep=5mm,
baselinestretch=1.2,
]
{sql}
UPDATE paradas SET cod_parada=1007800 WHERE cod_parada=1004561;
\end{minted}

\medskip

Formas de delete (aplican tambien al ON UPDATE):
\begin{itemize}
\item ON DELETE NO ACTION: Si hay referencias a otra tabla, no la borra.
\item ON DELETE CASCADE: Elimina las referencias también (las tuplas) en forma de cascada sucesivamente
\item Esta también SET NULL (cuidado si tenemos NOT NULL, sigue valiendo) y SET DEFAULT
\end{itemize}

\smallskip

Las autoridades de la Dirección de Transito quieren que sea
posible cambiar el código de una parada, modificando automáticamente todas las filas que
hacen referencia a ella en otras tablas. Modifique para ello el script de CREATE TABLE,
definiendo una CONSTRAINT de ON UPDATE en la tabla correspondiente.
\begin{minted}
[
frame=leftline,
framesep=5mm,
baselinestretch=1.2,
]
{sql}
ALTER TABLE colectivos_por_parada ADD CONSTRAINT fk_colectivos_por_parada 
FOREIGN KEY(cod_parada) REFERENCES paradas(cod_parada)
ON UPDATE CASCADE;
\end{minted}

Es posible actualizar la tabla ahora.


\newpage


\section*{Clase 5}
En realidad se dio durante la tercera clase debido a que caía un feriado en el día de la clase. (A partir de este punto no es del todo correcto el orden de clases mencionado en este apunte)
\section{UML}
Mas información para cada uno de estos diagramas en el apunte de Análisis de la Información. Acá solo se dejan breves descripciones.
\subsection*{Diagramas de actividad}
Para representar dinámicas del negocio, donde el tiempo va transcurriendo de arriba hacia abajo. Nos dan pie a los casos de uso.

\subsection*{Diagrama de Casos de uso}
Cada una de las actividades esta representada acá. Puede ocurrir que después algún caso de uso se parta en varios mas, que se representarían en otro diagrama. 

\subsection*{Diagrama de estados}
Se realizan para algunos de los objetos creados para indicar el paso de estados. No se indica que hace que pase de estados, sino los cambios posibles.

\subsection*{Diagrama de paquetes}
\begin{itemize}
\item Los negocios por lo general tienen múltiples contextos, que analizados con las herramientas empiezan a tomar forma en paquetes (parte de nuestros negocios). 
\item Permiten concentrar la dinámica de nuestro negocio desde el punto de vista del negocio.
\item Paquetes de un alto nivel, paquetes de nivel intermedio, y después de mas bajo nivel.
\item Los paquetes se asocian con componentes, que a su vez estan asociados a nodos.
\item Son muy intuitivos, pero poco rigurosos. Hay que tener en cuenta la inversión en la cadena de dependencia (principio). Esto no es tenido en cuenta en los siguientes gráficos de paquetes.
\end{itemize}


\begin{figure}[!htb]
    \centering
    \includegraphics[width=0.6\textwidth]{img/paq.PNG}
\end{figure}

Para que nuestro diseño goce de esto, el nivel superior debe estar compuesto por abstracciones/generalidades, y los niveles inferiores de cuestiones concretas que dependen de las generalidades. 

\begin{figure}[!htb]
    \centering
    \includegraphics[width=0.8\textwidth]{img/paqbien.PNG}
\end{figure}

% imagen
% ca y ia dentro del paquete PA
Se cumple la inversión en la cadena de dependencia
Hay que hacer el gráfico de la derecha para cuestiones vinculadas al diseño. % cuestion para ser mas cuidadoso en uml

\subsection*{Otras cuestiones de UML mencionadas}
\begin{itemize}
\item Composición, no puede existir la clase A sin la clase B, seria incompatible. Corresponde a un Ensamble-Parte.
\item Agregación, puedo construir tipo A sin B. Se agrega un agregador de B. Cuando lo tenga, A no es dueño de B.\textit{ (Llevándolo a C++, A no seria el encargado de destruir a B, ya que no es dueño de B)}
\item De uso, asociación, solamente utiliza al objeto.
\end{itemize}



\section*{Clase 6}

\section{Practica: Álgebra relacional}
\begin{itemize}
\item El álgebra relacional es al SQL lo que el latín es a las lenguas romances. Es el fundamento.
\item Proporciona un fundamento formal para las operaciones del modelo relacional el álgebra relacional.
\item Dos tipos de operaciones, unarias y binarias.
\item Unarias:selección, proyección
\item Binarias: join y variantes, en conjuntos esta la unión, intersección, diferencia y producto cartesiano.
\item Hay operadores fundamentales y no fundamentales.
\end{itemize}


\begin{figure}[!htb]
    \centering
    \includegraphics[width=0.8\textwidth]{img/operadores.PNG}
\end{figure}

\subsection*{Operadores en forma gráfica}

\begin{figure}[!htb]
    \centering
    \includegraphics[width=0.8\textwidth]{img/seleccionGrafico.PNG}
\end{figure}


\begin{figure}[!htb]
    \centering
    \includegraphics[width=0.8\textwidth]{img/productoCartesianoGrafico.PNG}
\end{figure}

\begin{figure}[!htb]
    \centering
    \includegraphics[width=0.8\textwidth]{img/divisionGrafica.PNG}
\end{figure}

\subsection*{División}

\begin{itemize}
\item Si surge la palabra \emph{todos/todas} quiero relacionar dos conjuntos, surge la división.
\item Es importante identificar de donde obtengo la información y no proyectar de menos/mas.
\end{itemize}

\medskip
\textbf{Forma general de la división}

\begin{center}
    $S \ \% \ T =  \pi_{A1 \ldots A_{n}}(S) - \pi_{A1 \ldots A_{n}}(\pi_{A1 \ldots A_{n}}(S)\ \times\ T\ -\ S \ )$
\end{center}



\begin{itemize}
\item No puedo dejar otra cosa que no quiero realmente.
\item La proyección elimina los repetidos, no los muestra.
\end{itemize}

\newpage
\subsection*{Taller}
\href{https://dbis-uibk.github.io/relax/calc/gist/552932a29392f8272951e01ada813ae1}{Link para probarlo}


\begin{itemize}
\item Listar las películas del año 2000.
\end{itemize}

$\sigma$ year=2000 (movies)


\begin{itemize}
\item Mostrar el nombre y apellido de los directores de la base que tienen películas fechadas en el año 2000
\end{itemize}


peliculas2000 = $\sigma$ year=2000 (movies)

idsPeliculas2000 = $\pi$ id (peliculas2000)

relacionPeliConDirector = (idsPeliculas2000) $\bowtie$ id=movie\_id (movies\_directors)

idsDirectores2000 = $\pi$ director\_id (relacionPeliConDirector)

directores2000 = (idsDirectores2000) $\bowtie$ director\_id=id (directors)

nombreApellidoDirectores2000 = $\pi$ first\_name,last\_name (directores2000)

nombreApellidoDirectores2000



\begin{itemize}
\item Mostrar los nombres de las películas filmadas por Woody Allen
\end{itemize}

woodyAllen = $\sigma$ first\_name='Woody' and last\_name='Allen' (directors)

idWoodyAllen = $\pi$ id woodyAllen

relacionDirectorPeliculas = idWoodyAllen $\bowtie$ id = director\_id (movies\_directors)

idsPeliculasWoodyAllen = $\pi$ movie\_id (relacionDirectorPeliculas)

peliculasWoodyAllen = (idsPeliculasWoodyAllen) $\bowtie$ movie\_id=id (movies)

nombrePeliculasWoodyAllen = $\pi$ name (peliculasWoodyAllen)

nombrePeliculasWoodyAllen



\begin{itemize}
\item Mostrar los nombres de las películas en que Hitler figura como actor
\end{itemize}

actorHitler = $\sigma$ last\_name='Hitler' (actors)

idActorHitler = $\pi$ id (actorHitler)

relacionActorRolesPeliculas = (idActorHitler) $\bowtie$ id=actor\_id (roles)

idPeliculaActorHitler = $\pi$ movie\_id (relacionActorRolesPeliculas)

peliculasActorHitler = (idPeliculaActorHitler) $\bowtie$ movie\_id=id (movies)

nombrePeliculasConHitler = $\pi$ name (peliculasActorHitler)

nombrePeliculasConHitler


\begin{itemize}
\item Otra opción renombrando (hace un natural join)
\end{itemize}

ID\_HITLER = $\pi$ id ($\sigma$ last\_name = 'Hitler' actors)

ID\_MOVIES = $\rho$ id $\leftarrow$ movie\_id ($\pi$ movie\_id (roles $\bowtie$ ($\rho$ actor\_id$\leftarrow$id ID\_HITLER)))

$\pi$ name (movies $\bowtie$ ID\_MOVIES)


\begin{itemize}
\item ¿Algún director abarca todo los géneros?
\end{itemize}


directoresConGeneros = $\pi$ director\_id,genre (directors\_genres)

generosPosibles = $\pi$ genre (movies\_genres)

directoresenTodosLosGeneros = directoresConGeneros $\div$ generosPosibles

directoresenTodosLosGeneros


\begin{itemize}
\item Mostrar el nombre y apellido de los directores que abarcaron (al menos) los mismos géneros que Polanski. ¿Y que Scorsese? ¿Y que Tarantino?
\end{itemize}


directorPolansky = $\sigma$ last\_name='Polanski' (directors)

idDirectorPolansky = $\pi$ id (directorPolansky)

relacionGenerosPolansky = (idDirectorPolansky) $\bowtie$ id=director\_id (directors\_genres)

generosPolansky = $\pi$ genre (relacionGenerosPolansky)

directoresConGeneros = $\pi$ director\_id,genre (directors\_genres)

idDirectoresConMismosGeneros = (directoresConGeneros) $\div$ (generosPolansky)

directoresConMismosGeneros = (idDirectoresConMismosGeneros) $\bowtie$ director\_id=id (directors)

nombresApellidosdirectores = $\pi$ first\_name,last\_name (directoresConMismosGeneros)

$\sigma$ last\_name$\ne$'Polanski' (nombresApellidosdirectores)


\begin{itemize}
\item Mostrar el año de la ultima película.
\end{itemize}

aniosPeliculas = $\pi$ year (movies)

aniosPeliculas2 = $\rho$ year2$\leftarrow$movies.year aniosPeliculas

aniosDoble = aniosPeliculas $\times$ aniosPeliculas2

aniosMenores = $\sigma$ year < year2 (aniosDoble)

ultimoAnio = $\pi$ year aniosDoble - $\pi$ year aniosMenores

ultimoAnio



\begin{itemize}
\item Listar las películas del ultimo año.
\end{itemize}

aniosPeliculas = $\pi$ year (movies)

aniosPeliculas2 = $\rho$ year2$\leftarrow$movies.year aniosPeliculas

aniosDoble = aniosPeliculas $\times$ aniosPeliculas2

aniosMenores = $\sigma$ year < year2 (aniosDoble)

ultimoAnio = $\pi$ year aniosDoble - $\pi$ year aniosMenores

peliculasUltimoAnio = ultimoAnio $\bowtie$ movies

peliculasUltimoAnio

\begin{itemize}
\item Listar las películas del director Hitchcock en las que actuó Carroll.
\end{itemize}


directorHitchcock = $\sigma$ last\_name='Hitchcock' (directors)

idDirectorHitchcock = $\pi$ id (directorHitchcock)

relacionPeliculasDirector = (idDirectorHitchcock) $\bowtie$ id=director\_id (movies\_directors)

idPeliculasHitchcock = $\pi$ movie\_id (relacionPeliculasDirector)

actoresEnPeliculasDeHitchcock = roles $\bowtie$ idPeliculasHitchcock

actorCarroll = $\sigma$ first\_name = 'Leo G.' and last\_name='Carroll' (actors)

idActorCarroll = $\pi$ id (actorCarroll)

rolesDeCarrollEnPeliculasDeHitcock = actoresEnPeliculasDeHitchcock $\bowtie$ roles.actor\_id=actors.id idActorCarroll

idPeliculasBuscadas = $\pi$ movie\_id rolesDeCarrollEnPeliculasDeHitcock

peliculasBuscadas = movies $\bowtie$ id=movie\_id idPeliculasBuscadas

peliculasBuscadas



\begin{itemize}
\item Listar las películas del director Hitchcock en las que NO actuó Carroll.
\end{itemize}

directorHitchcock = $\sigma$ last\_name='Hitchcock' (directors)

idDirectorHitchcock = $\pi$ id (directorHitchcock)

relacionPeliculasDirector = (idDirectorHitchcock) $\bowtie$ id=director\_id (movies\_directors)

idPeliculasHitchcock = $\pi$ movie\_id (relacionPeliculasDirector)

actoresEnPeliculasDeHitchcock = roles $\bowtie$ idPeliculasHitchcock

idsActoresEnPeliculasDeHitchcock = $\pi$ actor\_id,movie\_id  actoresEnPeliculasDeHitchcock

actorCarroll = $\sigma$ first\_name = 'Leo G.' and last\_name='Carroll' (actors)

idActorCarroll = $\pi$ id (actorCarroll)

rolesDeCarrollEnPeliculasDeHitcock = actoresEnPeliculasDeHitchcock $\bowtie$ roles.actor\_id=actors.id idActorCarroll

idsRolesDeCarrollEnPeliculasDeHitcock = $\pi$ actor\_id,movie\_id rolesDeCarrollEnPeliculasDeHitcock

peliculasSinCarrol = idsActoresEnPeliculasDeHitchcock - idsRolesDeCarrollEnPeliculasDeHitcock

idPeliculasBuscadas = $\pi$ movie\_id peliculasSinCarrol

peliculasBuscadas = movies $\bowtie$ id=movie\_id idPeliculasBuscadas

peliculasBuscadas



\begin{itemize}
\item Listar los actores que participan de al menos 3 películas.
\end{itemize}
La única forma de contar es realizando productos cartesianos.


PRIMER\_PROD = $\pi$ r.actor\_id, r.movie\_id, ro.movie\_id ($\rho$ r (roles) $\bowtie$ r.actor\_id = ro.actor\_id and r.movie\_id $\ne$ ro.movie\_id $\rho$ ro (roles))

SEGUNDO\_PROD = PRIMER\_PROD $\bowtie$ r.actor\_id =  roles.actor\_id and r.movie\_id $\ne$ roles.movie\_id and ro.movie\_id $\ne$ roles.movie\_id roles

IDS\_ACTORES = $\rho$ id$\leftarrow$r.actor\_id ($\pi$ r.actor\_id (SEGUNDO\_PROD))

actors $\bowtie$ IDS\_ACTORES 

\newpage

\section*{Clase 7}
\section{SQL}

\begin{itemize}
\item Es un lenguaje no procedural, basado en el calculo relacional de tuplas.
\item Tiene 9 partes el ISO SQL actualmente, las mas importantes son 2, la de Foundation (2) y Information and Definition Schemas (11). Se las conoce como Core SQL.
\item Es una gramática libre de contexto. Su sintaxis puede describirse a través de reglas de reglas de producción.
\item Para estudiar esto lo mejor es ir a la documentación, tiene las opciones y ejemplos.
\end{itemize}


\subsection*{Tipos de dato (en el estándar)}
\begin{itemize}
\item INTEGER
\item SMALLINT
\item FLAOT
\item DOUBLE PRESICION
\item NUMERIC(i,j): tipo numérico exacto. Permite especificar la precisión i y la escala j en dígitos.
\item CHARACTER(n) longitud fija
\item CHARACTER VARYING(n) longitud variable VARCHAR(n)
\item DATE: yyyy-mm-dd
\item TIME(i)
\item TIMESTAMP(i)
\item BOOLEAN: true, false, unknown - Logica de tres valores.
\item CLOB: Character Large Object - Generalmente guardados aparte 
\item BLOB: Binary Large Object
\end{itemize}


El usuario puede definir tipos de dato también:
\begin{itemize}
\item CREATE DOMAIN CODIGO\_PAIS AS CHAR(2)
\end{itemize}


\subsubsection*{Configuraciones}
\begin{itemize}
\item Las columnas pueden configurarse con valores por defecto con DEFAULT o auto incrementales AUTO\_INCREMENT.
\item PRIMARY KEY
\item UNIQUE: una columna o conjunto de columnas no puede estar con valores repetidos.
\item En el modelo relacional una relación es un conjunto cuyos elementos son las tuplas. Por lo tanto, una tupla no puede estar repetida en una relación. 
\item En SQL esto no pasa, pueden repetirse. Se conoce como multiset o bag of tuples.
\end{itemize}

\subsection*{Consultas}

\begin{itemize}
\item La consulta básica es SELECT ... FROM ... [ WHERE ... ]
\item SELECT es equivalente a la proyección del álgebra relacional, y el WHERE a la selección del álgebra relacional. La diferencia es que el WHERE no elimina duplicados, no es estrictamente una proyección.
\end{itemize}


\medskip

\subsubsection*{Condiciones en el WHERE}
\begin{itemize}
\item Podemos comparar atributos con otros, o con valores.
\item Realizar un pattern matching
\item Que este o no en un conjunto
\item Que este en rangos
\item Que no sea nulo
\item Que exista
\item Que este o no en una tabla
\item Permite poner otra consulta adentro también, podemos generar anidamientos de consultas.
\end{itemize}


\medskip
\textit{Nota: el distinto se escribe con <> , y no se pone =NULL se pone es IS NULL}

\subsubsection*{Otros}
\medskip
\begin{itemize}
\item Podemos usar alias para el FROM (tambien para el SELECT para redenominar atributos) con AS para las tablas.
\item Se permiten operaciones adentro del SELECT (+,-,*,/, etc)
\item Tenemos funciones de agregación: SUM, COUNT, AVG, MAX, MIN
\item DISTINCT después del SELECT elimina duplicados
\item Para comparar patrones LIKE
\end{itemize}


\subsubsection*{Join}
\begin{itemize}
\item Clausula JOIN para evitar escribir todas las condiciones de junta. INNER JOIN ... ON, NATURAL JOIN, LEFT/RIGHT/FULL OUTER.
\item Si no escribo OUTER, sql interpreta que queremos INNER.
\item Se puede plantear con el WHERE tambien.
\end{itemize}



\subsubsection*{Operaciones de conjuntos}
\begin{itemize}
\item UNION, INTERSECT, EXCEPT, deben de tener compatibilidad de tipos (misma cantidad de columnas). Si no se agrega ALL se eliminan duplicados
\end{itemize}


\subsubsection*{Ordenamiento y paginación}
\begin{itemize}
\item ORDER BY por defecto es ascendente
\item La forma estándar de la paginación es OFFSET n ROWS FETCH FIRST .. ROWS ONLY. Algunos SGBD implementan LIMIT o TOP
\end{itemize}

\subsubsection*{Agrupamiento}

\begin{itemize}
\item Queremos hacer un resumen de ciertos datos.
\item La agregación colapsa las tuplas que coinciden con una serie de atributos.
\item Formato: \mint{sql}|SELECT ... FROM ... GROUP BY ... [HAVING ...]|
\item Puedo usar sin agregar los atributos por los que agrupo. El resto si los quiero los tengo que agregar. (Promedio, suma, cantidad, etc) (Resumir muchos valores en uno)
\item HAVING es para filtrar con algún valor en particular, es una clausula opcional.
\item Si en una agregación no se especifican atributos de agregación, el resultado tendrá una única tupla.
\end{itemize}


\subsection*{Subconsultas}

\begin{figure}[!htb]
    \centering
    \includegraphics[width=0.8\textwidth]{img/subconsultas.PNG}
\end{figure}

\begin{itemize}
\item Dependencia con lo de afuera
\item Si no depende se dice que no están correlacionadas, tiene menor costo.
\item Me doy cuenta si esta correlacionada si usamos algo de la consulta mayor en la subconsulta. Puede pasar que en la ejecución el gestor la logre separar.
\end{itemize}


\subsubsection*{Tablas intermedias}
Para poder usar el resultado de consultas intermedias, definimos esas tablas como sigue:
\mint{sql}|WITH nombre AS consulta|

\medskip
Esta tambien el WITH RECURSIVE que amplia el poder expresivo de SQL.

\begin{itemize}
\item No se guardan en ningún lado, son solo sintaxis. Seria lo mismo que poner la consulta adentro de la otra.
\item Son mucho mas claras para documentar la consulta.
\end{itemize}

\subsection*{Inserciones}

Es posible agregar valores a las tablar mediante:
\mint{sql}|INSERT INTO tabla VALUES (valores)|

\medskip
Se puede insertar también el resultado de un SELECT, los tipos tienen que ser compatibles

\medskip
En cualquiera de los siguientes casos no se inserta el valor.
\begin{itemize}
\item Se asigna a una columna un valor fuera de su dominio
\item Se omitió una columna que no podía ser NULL
\item Se puso en NULL una columna que no podía ser NULL
\item La clave primaria asignada ya existe en la tabla
\item Una clave foránea hace referencia a una clave no existente
\end{itemize}


\subsection*{Eliminaciones}

\mint{sql}|DELETE FROM ... FROM ...WHERE ...|

Si se tenia ON DELETE RESTRICT no se elimina la fila.

\subsection*{Modificaciones}
\mint{sql}|UPDATE ... SET ... WHERE ...|

Para cada fila t que cumpla alguna condición de las siguientes no se actualiza.
\begin{itemize}
\item Se modifica una columna asignándole un valor fuera de su dominio
\item Se pasó a NULL una columna que no podía tomar ese valor
\item Se asignó a la clave primaria un valor que ya existe en la tabla
\item Se modificó una clave foránea para hacer referencia a una clave no existente
\item Se configuro ON UPDATE RESTRICT
\end{itemize}


\subsection*{Eliminando tablas}

Para tablas: DROP TABLE tabla [RESTRICT|CASCADE]

\medskip
Para esquemas: DROP SCHEMA esquema [RESTRICT|CASCADE]


\subsection*{Otras funciones}
\begin{itemize}
    \item SUBSTRING(string FROM start FOR length)
    \item UPPER/LOWER(string)
    \item CHAR\_LENGTH(string)
    \item CAST( attr AS tipo)
    \item EXTRACT(campo FROM attr) - (para las fechas)
    \item || para concatenar
\end{itemize}

\begin{minted}
[
frame=leftline,
framesep=5mm,
baselinestretch=1.2,
]
{sql} 
SELECT nro_factura ,
    CAST(
        CAST(año_venc AS CHAR) || '-' ||
        SUBSTRING(('0' || CAST(mes_venc AS VARCHAR)
                FROM (2 CAST(mes_venc <10 AS INTEGER)) FOR 2)
        || '-' ||
        SUBSTRING(('0' || CAST(dia_venc AS VARCHAR))
                FROM (2 CAST(dia_venc <10 AS INTEGER)) FOR 2) AS DATE
        ) AS fecha_venc
FROM Facturas f;
\end{minted}

\subsection*{Estructura CASE}
\mint{sql}|CASE WHEN .. THEN .. ELSE ..END|
Permite agregar lógica de programación a la salida de una sentencia de sql.


\subsection*{Vistas en SQL}
Podemos mostrar a los usuarios determinadas partes de la base de datos a través de las vistas. Se puede decir que es una tabla virtual, el resultado de una operación ejecutada en ese momento. 

Se crean de la siguiente forma:
\begin{minted}
[
frame=leftline,
framesep=5mm,
baselinestretch=1.2,
]
{sql}
CREATE VIEW nombreVista
[(nuevoNombreColumna [,...])] AS consulta
[WITH [CASCADED | LOCAL] CHECK OPTION]
\end{minted}


Ejemplo

\begin{minted}
[
frame=leftline,
framesep=5mm,
baselinestretch=1.2,
]
{sql}
CREATE VIEW EMP_ACC AS
SELECT Nombre, Documento, Sector, Sucursal FROM EMPLEADOS;
\end{minted}


El usuario vería solo la información provista por:

\begin{minted}
[
frame=leftline,
framesep=5mm,
baselinestretch=1.2,
]
{sql}
SELECT * FROM EMP_ACC;
\end{minted}


En las vistas también se pueden buscar casos específicos, agrupar, combinar, son condiciones que se agregan en el SELECT.

Se pueden eliminar vistas con 

\begin{minted}
[
frame=leftline,
framesep=5mm,
baselinestretch=1.2,
]
{sql}
DROP VIEW nombreVista [RESTRICT | CASCADE]
\end{minted}


RESTRICT si tengo algún objeto relacionado, no ejecuta el borrado
CASCADE borra todos los objetos dependientes relacionados, aquellos que hagan referencia a la vista

\subsubsection*{Cambios en las vistas}

Como los cambios en las tablas bases se ven automáticamente en las vistas, también se pueden modificar datos a través de las vistas y que modifique las tablas base. Para que pueda pasar esto, no debe de tener operadores conjuntistas, el operador DISTINCT, funciones agregadas, o GROUP BY

Para ser actualizable, FROM debe referenciar a solo una tabla base.

\subsubsection*{Materialización}

Las vistas pueden tomar mucho tiempo si son complejas, se puede evitar con la materialización. Esto realiza una tabla temporal que almacena la vista. Se mantiene la vista por cada cambio de los datos.

\subsubsection*{¿Para que usarlas?}

Las vistas se usarían para:
\begin{itemize}
\item Ocultar información
\item Administración simple de permisos
\item Personalizar datos
\item Menor complejidad
\item Proporcionar compatibilidad con las consultas ya armadas
\item Integridad de los datos
\item Combinar datos de servidores
\end{itemize}


\subsubsection*{Privilegios sobre ellas}

Se otorgan privilegios a los usuarios de la siguiente forma
\begin{minted}
[
frame=leftline,
framesep=5mm,
baselinestretch=1.2,
]
{sql}
GRANT {Lista Privilegios | ALL PRIVILEGES}
ON NombreObjeto
TO {ListaIdentifiadoresAutorizacion | PUBLIC}
[WITH GRANT OPTION]
\end{minted}

Se pueden revocar con: \mint{sql}|REVOKE|


\newpage

\section*{Clase 8}
\section{Practica: Taller SQL}
\begin{itemize}
    \item Es un lenguaje declarativo, le digo que quiero y de alguna forma lo hace.
    \item ¿Cual forma de hacer algo es mas performante? No es fácil saberlo, generalmente el motor se puede dar cuenta cual hacer.
    \item Las cadenas de caracteres con comillas simples. Las comillas dobles las interpreta como que queremos acceder a una columna.
    \item Operadores lógicos AND, OR, NOT
    \item Evitar usar SELECT * ... , ya que no queda definida la consulta en ese momento.
    \item Definir forma de almacenar datos, esto es buena practica, pero trae problemas ante alguna equivocación.
    \item Con los operadores de conjuntos por defecto no hay duplicados.
    \item No usar NATURAL JOIN en aplicativos para evitar problemas a futuro, ya que la consulta no queda totalmente definida. Si el esquema de la tabla cambia, nos enteramos cuando se ejecuta.
    \item El resultado de una consulta se puede usar como tabla en la entrada de otra consulta (subconsulta)
    \item Cuando agrupo puedo devolver alguna de las columnas agrupadas, o resultados que surgen de alguna función de agregación.
\end{itemize}

\subsection*{1 - Devuelva todos los datos de las notas que no sean de la materia 75.1}

\begin{itemize}
\item Con que difiera alguno de ellos ya nos alcanza. Fijarse con De Morgan. Plantearlo con \mint{sql}|AND| es un error común, ya que obligamos a que sea distinto si o si, en este caso hay que negar el resultado si queremos el \mint{sql}|AND|.
\end{itemize}


\begin{minted}
[
frame=leftline,
framesep=5mm,
baselinestretch=1.2,
]
{sql} 
SELECT * FROM notas WHERE codigo <> 75 OR numero <> 1
\end{minted}

\subsection*{2 Devuelva para cada materia dos columnas: una llamada “código” que contenga una concatenación del código de departamento, un punto y el numero de materia, con el formato “XX.YY” (ambos valores con dos dígitos, agregando ceros a la izquierda en caso de ser necesario) y otra con el nombre de la materia.}

\begin{minted}
[
frame=leftline,
framesep=5mm,
baselinestretch=1.2,
]
{sql} 
SELECT to_char(codigo,'fm00') || '.' || to_char(numero,'fm00') AS "codigo",nombre
FROM materias
\end{minted}

\begin{itemize}
\item Con el doble pipe lo estamos concatenando con un punto en el medio.
\item Con el AS renombramos la columna inicial.
\item Documentación del \href{https://www.postgresql.org/docs/current/functions-formatting.html}{to\_char}.
\end{itemize}



\subsection*{3 - Para cada nota registrada, devuelva el padrón, código de departamento, número de materia, fecha y nota expresada como un valor entre 1 y 100.}

\begin{minted}
[
frame=leftline,
framesep=5mm,
baselinestretch=1.2,
]
{sql} 
SELECT padron,codigo,numero,fecha,nota*10 AS "nota"
FROM notas
\end{minted}

\subsection*{4 -  Ídem al anterior pero mostrando los resultados paginados en páginas de 5 resultados cada una, devolviendo la segunda página}

\begin{minted}
[
frame=leftline,
framesep=5mm,
baselinestretch=1.2,
]
{sql} 
SELECT padron,codigo,numero,fecha,nota*10 AS "nota"
FROM notas
OFFSET 5 ROWS FETCH FIRST 5 ROWS ONLY
\end{minted}


\begin{itemize}
\item Esta \mint{sql}|LIMIT 5| después del offset también.
\item Si no le digo algún orden, no se que me va a devolver. Si queremos garantizarnos los resultados, hay que tomarse la costumbre de ordenar de la forma que no deje ambigüedad en las filas. En este caso agregando antes del \mint{sql}|FETCH| un \mint{sql}|ORDER BY padron| - Así evitamos tener un orden indeterminado.
\end{itemize}

\begin{minted}
[
frame=leftline,
framesep=5mm,
baselinestretch=1.2,
]
{sql} 
SELECT padron,codigo,numero,fecha,nota*10 AS "nota"
FROM notas
ORDER BY padron,codigo,numero,fecha
OFFSET 5 ROWS FETCH FIRST 5 ROWS ONLY;
\end{minted}

\subsection*{5 - Ejecute una consulta SQL que devuelva el padrón y nombre de los alumnos cuyo apellido es “Molina”}
\begin{minted}
[
frame=leftline,
framesep=5mm,
baselinestretch=1.2,
]
{sql} 
SELECT padron,nombre
FROM alumnos
WHERE apellido = 'Molina';
\end{minted}

\begin{itemize}
\item No es lo mismo que: (el siguiente estaría mal)
\end{itemize}

\begin{minted}
[
frame=leftline,
framesep=5mm,
baselinestretch=1.2,
]
{sql} 
SELECT padron,nombre
FROM alumnos
WHERE apellido = 'molina';
\end{minted}


\begin{itemize}
\item Postgres es una base de datos CASE SENSITIVE. Por lo que no devolvería nada en este caso. \mint{sql}|LIKE| es case sensitive también, se puede usar \mint{sql}|ILIKE| (no es estándar y no aprovecha índices)
\end{itemize}

\begin{minted}
[
frame=leftline,
framesep=5mm,
baselinestretch=1.2,
]
{sql} 
SELECT padron,nombre
FROM alumnos
WHERE apellido ILIKE 'Molina';
\end{minted}


\begin{itemize}
\item Otra alternativa es llevándolo a una forma común con Upper o Lower.
\end{itemize}

\begin{minted}
[
frame=leftline,
framesep=5mm,
baselinestretch=1.2,
]
{sql} 
SELECT padron,nombre
FROM alumnos
WHERE Upper(apellido) = 'MOLINA';
\end{minted}


\subsection*{6 - Obtener el padrón de los alumnos que ingresaron a la facultad en el año 2010}

\begin{minted}
[
frame=leftline,
framesep=5mm,
baselinestretch=1.2,
]
{sql} 
SELECT padron
FROM alumnos
WHERE fecha_ingreso>='2010-01-01' AND fecha_ingreso<='2010-12-31';
\end{minted}

Con otra sintaxis:
\begin{minted}
[
frame=leftline,
framesep=5mm,
baselinestretch=1.2,
]
{sql} 
SELECT padron
FROM alumnos
WHERE fecha_ingreso BETWEEN '2010-01-01' AND '2010-12-31';
\end{minted}

\subsection*{7 - Obtener la mejor nota registrada en la materia 75.15}

\begin{minted}
[
frame=leftline,
framesep=5mm,
baselinestretch=1.2,
]
{sql} 
SELECT MAX(nota)
FROM notas
WHERE codigo=75 AND numero=15;
\end{minted}

\subsection*{8 - Obtener el promedio de notas de las materias del departamento de código 75}

\begin{minted}
[
frame=leftline,
framesep=5mm,
baselinestretch=1.2,
]
{sql} 
SELECT AVG(nota)
FROM notas
WHERE codigo=75;
\end{minted}


\subsection*{9 - Obtener el promedio de nota de aprobación de las materias del departamento de código 75.}

\begin{minted}
[
frame=leftline,
framesep=5mm,
baselinestretch=1.2,
]
{sql} 
SELECT AVG(nota)
FROM notas
WHERE codigo=75 and nota>=4;
\end{minted}

\subsection*{10 - Obtener la cantidad de alumnos que tienen al menos una nota}

\begin{minted}
[
frame=leftline,
framesep=5mm,
baselinestretch=1.2,
]
{sql} 
SELECT COUNT(DISTINCT padron)
FROM notas
\end{minted}


\begin{itemize}
\item \mint{sql}|COUNT| solo cuenta los distintos de nulos, si queremos que sean distintas lo hacemos con \mint{sql}|DISTINCT|.
\end{itemize}


\subsection*{11 - Devolver los padrones de los alumnos que no registran nota en materias}


\begin{itemize}
\item Estamos realizando la operación de conjuntos: A - B
\end{itemize}

\begin{minted}
[
frame=leftline,
framesep=5mm,
baselinestretch=1.2,
]
{sql} 
(SELECT padron
FROM alumnos)
EXCEPT
(SELECT padron
FROM notas)
\end{minted}

\begin{minted}
[
frame=leftline,
framesep=5mm,
baselinestretch=1.2,
]
{sql} 
SELECT padron
FROM alumnos
WHERE padron
NOT IN
(SELECT DISTINC(padron) FROM notas)
\end{minted}

\subsection*{12 - Con el objetivo de traducir a otro idioma los nombres de materias y departamentos, devolver en una única consulta los nombres de todas las materias y de todos los departamentos}

\begin{minted}
[
frame=leftline,
framesep=5mm,
baselinestretch=1.2,
]
{sql} 
(SELECT nombre
FROM materias)
UNION
(SELECT nombre
FROM departamentos)
\end{minted}

\subsection*{13 - Devolver para cada materia su nombre y el nombre del departamento}


\begin{itemize}
\item Con \mint{sql}|JOIN|
\end{itemize}
\begin{minted}
[
frame=leftline,
framesep=5mm,
baselinestretch=1.2,
]
{sql} 
SELECT m.nombre, d.nombre
FROM materias AS m, departamentos AS d
WHERE m.codigo=d.codigo 
\end{minted}


\begin{itemize}
\item Otras forma para hacerlo:
\end{itemize}
\begin{minted}
[
frame=leftline,
framesep=5mm,
baselinestretch=1.2,
]
{sql} 
SELECT m.nombre, d.nombre
FROM materias AS m INNER JOIN  departamentos AS d
ON m.codigo=d.codigo 
\end{minted}

\begin{minted}
[
frame=leftline,
framesep=5mm,
baselinestretch=1.2,
]
{sql} 
SELECT m.nombre, d.nombre
FROM materias AS m INNER JOIN departamentos AS d
USING (codigo) 
\end{minted}


\subsection*{14 - Para cada 10 registrado, devuelva el padrón y nombre del alumno y el nombre de la materia correspondientes a dicha nota}

\begin{minted}
[
frame=leftline,
framesep=5mm,
baselinestretch=1.2,
]
{sql} 
SELECT a.padron,a.nombre,m.nombre
FROM alumnos AS a, notas AS n, materias AS m
WHERE n.nota=10 AND a.padron=n.padron
AND n.codigo=m.codigo AND n.numero=m.numero;
\end{minted}

\begin{minted}
[
frame=leftline,
framesep=5mm,
baselinestretch=1.2,
]
{sql} 
SELECT a.padron,a.nombre,m.nombre
FROM notas AS n
INNER JOIN materias AS m ON m.numero=n.numero AND m.codigo=m.codigo
INNER JOIN alumnos AS a ON a.padron=n.padron
WHERE n.nota=10;
\end{minted}

\begin{minted}
[
frame=leftline,
framesep=5mm,
baselinestretch=1.2,
]
{sql} 
SELECT n.padron,a.nombre AS nombre_alumno,m.nombre AS nombre_materia
FROM (notas n INNER JOIN alumnos a USING (padron)) INNER JOIN materias m 
USING (codigo,numero)
WHERE n.nota = 10
\end{minted}

\subsection*{15 - Listar para cada carrera su nombre y el padrón de los alumnos que estén anotados en ella. Incluir también las carreras sin alumnos inscriptos}

\begin{minted}
[
frame=leftline,
framesep=5mm,
baselinestretch=1.2,
]
{sql} 
SELECT c.nombre,i.padron
FROM carreras AS c LEFT OUTER JOIN inscripto_en AS i
USING (codigo)
\end{minted}

\begin{minted}
[
frame=leftline,
framesep=5mm,
baselinestretch=1.2,
]
{sql} 
SELECT carreras.nombre,alumnos.nombre,alumnos.apellido
FROM carreras
LEFT OUTER JOIN inscripto_en ON inscripto_en.codigo = carreras.codigo
LEFT OUTER JOIN alumnos ON inscripto_en.padron = alumnos.padron
\end{minted}

\subsection*{16 - Ídem punto anterior pero teniendo en cuenta únicamente alumnos con padrón mayor a 75000}

% REVISAR
\begin{minted}
[
frame=leftline,
framesep=5mm,
baselinestretch=1.2,
]
{sql} 
SELECT carreras.nombre,alumnos.nombre,alumnos.apellido
FROM carreras
LEFT OUTER JOIN inscripto_en ON inscripto_en.codigo = carreras.codigo
LEFT OUTER JOIN alumnos ON inscripto_en.padron = alumnos.padron
WHERE alumnos.padron > 75000
\end{minted}

\begin{minted}
[
frame=leftline,
framesep=5mm,
baselinestretch=1.2,
]
{sql} 
SELECT c.nombre,i.padron
FROM carreras AS c LEFT OUTER JOIN inscripto_en AS i
USING (codigo)
WHERE padron > 75000
\end{minted}

\begin{minted}
[
frame=leftline,
framesep=5mm,
baselinestretch=1.2,
]
{sql} 
SELECT c.nombre, ie.padron
FROM taller_4.CARRERAS c 
LEFT OUTER JOIN taller_4.inscripto_en ie ON (ie.codigo = c.codigo AND ie.padron > 75000);
\end{minted}

\subsection*{17 - Listar el padrón de aquellos alumnos que tengan m´as de una nota en la materia 75.15}


\begin{itemize}
\item De esta forma no se tiene la mejor performance, ya que se esta revisando fila por fila con el valor de la consulta mayor.
\end{itemize}

\begin{minted}
[
frame=leftline,
framesep=5mm,
baselinestretch=1.2,
]
{sql} 
SELECT a.padron
FROM alumnos as a
WHERE 
(SELECT count(n.nota)>1 FROM notas AS n
WHERE n.padron=a.padron AND n.codigo=75 AND n.numero=15)
\end{minted}

\begin{itemize}
\item Otra opción poniendo el >1 afuera de la subconsulta.
\end{itemize}
\begin{minted}
[
frame=leftline,
framesep=5mm,
baselinestretch=1.2,
]
{sql} 
SELECT a.padron
FROM alumnos as a
WHERE 
(SELECT count(n.nota) FROM notas AS n
WHERE n.padron=a.padron AND n.codigo=75 AND n.numero=15) >1 
\end{minted}

\begin{minted}
[
frame=leftline,
framesep=5mm,
baselinestretch=1.2,
]
{sql} 
SELECT DISTINCT n1.padron
FROM notas n1, notas n2
WHERE n1.codigo = 75 AND n2.codigo = 75
AND n1.numero = 15 AND n2.numero = 15
AND n1.padron = n2.padron
AND n1.fecha <> n2.fecha
\end{minted}

\subsection*{18 - Obtenga el padrón y nombre de los alumnos que aprobaron la materia 71.14 y no aprobaron la materia 75.15}

\begin{minted}
[
frame=leftline,
framesep=5mm,
baselinestretch=1.2,
]
{sql} 
SELECT A.PADRON, A.NOMBRE 
FROM ALUMNOS A
WHERE EXISTS(SELECT 1 FROM NOTAS N WHERE N.NOTA >= 4
            AND N.CODIGO = 71 AND N.NUMERO = 14 
            AND N.PADRON = A.PADRON) 
AND NOT EXISTS(SELECT 1 FROM NOTAS N WHERE N.NOTA >= 4 
            AND N.CODIGO = 71 AND N.NUMERO = 15 
            AND N.PADRON = A.PADRON);
\end{minted}
\begin{itemize}
\item Se fija que exista un aprobado en la materia, y que no exista un aprobado en la otra.
\end{itemize}


Otra forma:
\begin{minted}
[
frame=leftline,
framesep=5mm,
baselinestretch=1.2,
]
{sql} 
SELECT A.PADRON, A.NOMBRE 
FROM ALUMNOS A
WHERE A.PADRON IN (SELECT N.PADRON FROM NOTAS N WHERE N.NOTA >= 4 
			 AND N.CODIGO = 71 AND N.NUMERO = 14 ) 
AND A.PADRON NOT IN (SELECT N.PADRON FROM NOTAS N 
				WHERE N.NOTA >= 4 
				AND N.CODIGO = 71 AND N.NUMERO = 15 );
\end{minted}

\subsection*{19 - Obtener, sin repeticiones, todos los pares de padrones de alumnos tales que ambos alumnos rindieron la misma materia el mismo día. Devuelva también la fecha y el código y numero de la materia}

\begin{minted}
[
frame=leftline,
framesep=5mm,
baselinestretch=1.2,
]
{sql} 
SELECT DISTINCT N.PADRON PADRON, NOTA.PADRON PADRON2,
N.FECHA, N.CODIGO, N.NUMERO FROM TALLER_4.NOTAS N
INNER JOIN
TALLER_4.NOTAS AS NOTA
ON NOTA.PADRON < N.PADRON
AND NOTA.FECHA = N.FECHA 
AND N.CODIGO = NOTA.CODIGO 
AND N.NUMERO = NOTA.NUMERO;
\end{minted}


\subsection*{20 - Para cada departamento, devuelva su código, nombre, la cantidad de materias que tiene y la cantidad total de notas registradas en materias del departamento. Ordene por la cantidad de materias descendente}

\begin{minted}
[
frame=leftline,
framesep=5mm,
baselinestretch=1.2,
]
{sql} 
SELECT d.codigo, d.nombre, COUNT(DISTINCT n.numero) AS "cant_mat",
COUNT(*) AS "cant_not"
FROM notas n INNER JOIN materias m USING (codigo, numero) 
INNER JOIN departamentos d USING (codigo)
GROUP BY d.codigo, d.nombre
ORDER BY COUNT(DISTINCT n.numero) DESC
\end{minted}

\begin{minted}
[
frame=leftline,
framesep=5mm,
baselinestretch=1.2,
]
{sql} 
WITH cant AS (
	SELECT codigo, COUNT(numero) AS cant FROM materias GROUP BY codigo
), notasD AS (
	SELECT codigo, COUNT(notas) AS cant FROM notas GROUP BY codigo
)
SELECT d.codigo,d.nombre, c.cant, n.cant
FROM departamentos d, cant c, notasD n
WHERE d.codigo = c.codigo AND d.codigo=n.codigo
ORDER BY c.cant DESC

\end{minted}

\subsection*{21 - Para cada carrera devuelva su nombre y la cantidad de alumnos inscriptos. Incluya las carreras sin alumnos}

\begin{minted}
[
frame=leftline,
framesep=5mm,
baselinestretch=1.2,
]
{sql} 
SELECT c.nombre, COUNT(i.padron)
FROM carreras c LEFT OUTER JOIN  inscripto_en i USING (codigo)
GROUP BY c.codigo, c.nombre
\end{minted}

Es importante contar un atributo que no tenga nulos, así lo cuentan.

\subsection*{22 - Para cada alumno con al menos tres notas, devuelva su padrón, nombre, promedio de notas y mejor nota registrada}

\begin{minted}
[
frame=leftline,
framesep=5mm,
baselinestretch=1.2,
]
{sql} 
SELECT a.padron, a.nombre, AVG(n.nota) AS promedio, MAX(n.nota) AS mejor_nota
FROM alumnos AS a, notas AS n
WHERE a.padron = n.padron
GROUP BY a.padron, a.nombre
HAVING COUNT(*) >= 3
\end{minted}

\begin{itemize}
\item Condiciones que se aplican grupo a grupo no van en el WHERE, ya que viene en una etapa después.
\item Puedo evaluar cosas que después no devuelvo en el HAVING.
\end{itemize}


\subsection*{23 - Obtener el código y numero de la o las materias con mayor cantidad de notas registradas.}

\begin{minted}
[
frame=leftline,
framesep=5mm,
baselinestretch=1.2,
]
{sql} 
SELECT n.codigo, n.numero, COUNT(*)
FROM notas n
GROUP BY n.codigo, n.numero
HAVING COUNT(*) >= ALL (
    SELECT COUNT(*)
    FROM notas n
    GROUP BY n.codigo, n.numero
)
\end{minted}

\begin{minted}
[
frame=leftline,
framesep=5mm,
baselinestretch=1.2,
]
{sql} 
WITH cantidades AS (SELECT n.codigo, n.numero, COUNT(*) AS cant
	FROM notas n
	GROUP BY n.codigo, n.numero)
SELECT codigo, numero, cant
FROM cantidades c
WHERE c.cant = (SELECT MAX(cant) FROM cantidades)
\end{minted}


\begin{itemize}
\item No es valido ordenarlo y devolver el primero, puede darse el caso de que varios coincidan con el mejor valor. Queremos todas con la mayor cantidad.
\item No esta permitido realizar un MAX del COUNT en el HAVING.
\item No podemos anidar funciones de agregación.
\end{itemize}

\subsection*{24 - Obtener el padrón de los alumnos que tienen nota en todas las materias}

\begin{minted}
[
frame=leftline,
framesep=5mm,
baselinestretch=1.2,
]
{sql} 

SELECT padron
FROM alumnos a
WHERE NOT EXISTS (
    SELECT * 
    FROM materias m 
    WHERE NOT EXISTS (
        SELECT * 
        FROM notas n
        WHERE n.padron = a.padron
		      AND n.codigo = m.codigo
		      AND n.numero = m.numero
        )
)

\end{minted}

\begin{itemize}
\item Es una división.
\item Buscamos los alumnos con nota en todas las materias, tal que para los alumnos no exista materia en la que no tiene nota.
\item Alumnos para los que no existe una materia en la que no existe una nota de ese alumno en esa materia
\end{itemize}



\begin{minted}
[
frame=leftline,
framesep=5mm,
baselinestretch=1.2,
]
{sql} 

SELECT padron
FROM alumnos a
WHERE NOT EXISTS (
    (SELECT codigo, numero FROM materias)
    EXCEPT
    (SELECT codigo, numero FROM notas n WHERE n.padron=a.padron)
)

\end{minted}


\begin{itemize}
\item Primer conjunto: todas las materias
\item Segundo conjunto: todas las materias en las que un alumno tiene nota
\end{itemize}


\begin{minted}
[
frame=leftline,
framesep=5mm,
baselinestretch=1.2,
]
{sql} 


SELECT padron
FROM notas n
GROUP BY n.padron
HAVING COUNT( DISTINCT(to_char(codigo,'fm00') ||'.'|| to_char(numero,'fm00') ))
       =
       (SELECT COUNT(*) FROM materias)

\end{minted}


\begin{itemize}
\item Contar cuantas materias hay y cuento en cuantas materias tiene nota. Tiene que dar lo mismo
\item El DISTINCT solo recibe un atributo, no es estandar que tome dos.
\end{itemize}

\subsection*{25 - Obtener el promedio general de notas por alumno (cuantas notas tiene en promedio un alumno), considerando únicamente alumnos con al menos una nota}

\begin{minted}
[
frame=leftline,
framesep=5mm,
baselinestretch=1.2,
]
{sql} 


SELECT CAST(COUNT(*) AS DECIMAL) / COUNT(DISTINCT padron)
FROM notas n;

\end{minted}

Cuidado con realizar la división entera, esta trunca el resultado.
\newpage

\section*{Clase 9}
\section{Principios de diseño}
Los principios de diseño (SOLID), la ley de Demeter, separación de incumbencias están siempre presentes en cada decisión que tomemos sobre la arquitectura.

Los criterios generales son:
\begin{itemize}
\item Economía
\item Visibilidad
\item 'Espaciamiento'
\item Simetría
\item Emergencias
\end{itemize}


\subsection*{Economía}
\begin{itemize}
\item Tiene que ver con el listado de métodos que componen una misma interfaz.
\item Busca evitar complejidades innecesarias, flexibilidad innecesaria, y enfocarnos en la usabilidad mas que en la reusabilidad.
\item La interfaz tiene que estar limpia y económica. Tener la mínima cantidad de métodos necesarios que sean útiles.
\end{itemize}


\subsection*{Visibilidad}

\begin{itemize}
\item La especificación en términos de interfaces y herencia o implementación suele ser obscura, poco visible y difícil de entender y navegar.
\item Es preferible la composición y delegación que la herencia.
\end{itemize}


\begin{figure}[!htb]
    \centering
    \includegraphics[width=0.8\textwidth]{img/VisibilidadPatrones.PNG}
\end{figure}

\subsection*{'Espaciamiento'}
\begin{itemize}
\item Tiene que ver con el desacoplamiento.
\item Contribuye a la visibilidad y el reuso.
\item Busca definir cuales son los roles de cada componente del diseño, lo cual nos ayuda a separarlos. Si nos cuesta definirlos es que esta en demasiadas cosas.
\item Distancia entre responsabilidades funcionales claramente distintas y autónomas conduce a componentes.
\item  Espaciado entre las perspectivas de uso de un componente conduce a interfaces con rol específico.
\item La separación entre grupos de componentes conduce a capas y subsistemas.
\item El espacio entre el contrato y la realización conduce a interfaces explícitas e implementaciones separadas.
\end{itemize}



\subsection*{Simetría}

\begin{itemize}
\item Facilita la comprensión, comunicación, extensión y mantenimiento.
\end{itemize}

\begin{figure}[!htb]
    \centering
    \includegraphics[width=0.8\textwidth]{img/simetriaPatrones.PNG}
    \caption{Diagrama de arriba se instancian instancias a través de un Factory que no destruye los objetos. Esto es asimétrico, solo crea no destruye.}
\end{figure}



\subsection*{Emergencias}
\begin{itemize}
\item Tiene que ver con algo que emerge por su forma, por su repetición en contexto similar.
\item Facilita la organización y escalabilidad.
\item Emergente es el comportamiento de auto-organización y junto con el control son a menudo la clave para la escalabilidad, eficiencia y economía en las arquitecturas
\end{itemize}


\section{Patrones}

\begin{itemize}
\item Situaciones que se empiezan a repetir y se puede solucionar de la misma forma.
\item Los definimos por un contexto en la cual se plantea un problema, el problema (\textit{que según Bushman puede estar compuesto por fuerzas que se contraponen y los patrones buscan balancear}), y por la solución (relación/interacción entre varias clases)
\item Nadie inventa patrones, estos se descubren.
\item Esta implícito en el nombre del patrón todo lo que constituye.
\item Son buena documentación de diseño.
\end{itemize}

\textit{Más adelante se sigue con patrones de diseño.}

\textit{Parte de donde surge la idea (arquitectura - de los edificios): A Pattern Language - Christopher Alexander }


\section{Patrones de arquitectura}
Para saber cuando usar cada uno, necesitamos saber el contexto y el problema que se presenta. Orientados a organizar la plataforma en la cual se va a construir el sistema.

\subsection*{Layer}

\begin{itemize}
\item Idea: Tenemos un sistema que tiene un volumen grande, en el cual aparecen conceptos de distinto nivel de abstracción.
\item Hay diferentes capas con distintos niveles de abstracción. \textit{Ej. Teléfonos presentado en clase, comunicación, tarifas de las llamadas, las impresoras, administración del locutorio} Cuando tenemos esto conviene el uso de este patrón.
\item Cada capa (layer) utiliza un servicio de la capa superior. Permite reutilizar partes.
\item Las ventajas son reuso, cambiabilidad, desarrollo simultaneo, ya que las capas se pueden manejar por separado.
\item La desventaja es un diseño mas elaborado
\item El patrón no ofrece una respuesta a cuantas capas tengo que tener, esto depende de los requerimientos. \textit{Ej. Modelo OSI de redes}
\end{itemize}

\begin{figure}[!htb]
    \centering
    \includegraphics[width=0.8\textwidth]{img/LayerPatron.PNG}
\end{figure}

\subsection*{EA}
\begin{itemize}
\item Enterprise Architecture
\item Es un Mega patrón, compuesto por muchos patrones. (se va a ver mejor mas adelante)
\item Es conveniente usarlo cuando hay un sistema cliente-servidor (remoto). También cuando hay usuarios concurrentes, múltiples interfaces de usuario, reglas de negocio y datos persistentes.
\item Se usa desacoplando los diferentes 'tiers' para lograr separar incumbencias.
\item Las ventajas son el reuso y la cambiabilidad,
\item La desventaja es un mayor trabajo.
\end{itemize}


\begin{figure}[!htb]
    \centering
    \includegraphics[width=0.8\textwidth]{img/EAPatron.PNG}
    \caption{Ejemplo}
\end{figure}


\subsection*{Microkernel}
El patrón arquitectónico de Microkernel se aplica a un sistema de software que debe ser capaz de adaptarse a los cambios en los requisitos del sistema. Separa un núcleo funcional mínimo de la funcionalidad extendida y de las piezas específicas del cliente. El microkernel también sirve como enchufe para conectar estas extensiones y coordinar su colaboración
\begin{itemize}
\item Conviene utilizarlo en sistemas de larga vida que deben evolucionar ante cambios de distinto tipo con diferente frecuencia.
\item Con un sistema monolítico se complica.
\item Asigna componentes esenciales al microkernel y distribuye las componentes entre los servers.
\item Esta compuesto por External Servers, el Microkernel, y el Internal Server
\item Las ventajas son que soporta cambios con menor impacto.
\item La desventaja es un diseño mas elaborado
\end{itemize}

\begin{figure}[!htb]
    \centering
    \includegraphics[width=0.8\textwidth]{img/MicrokernelPatron.PNG}
\end{figure}


\subsection*{Pipe Filter}

\begin{itemize}
\item Conviene usarlo cuando debemos procesar un stream de datos, y este procesamiento se puede descomponer en N procesamientos atómicos.
\item Permite sistemas mas reutilizables, es reuso.
\item Para usarlo, distribuir los procesamientos en filtros, definir el formato de datos en los pipes y definir la implementación.
\item La ventaja es el reuso y el rápido prototipado.
\item La desventaja es que no se puede usar en sistemas críticos ya que no podemos controlar el estado global. (en caso de que falle un filtro)
\end{itemize}

\begin{figure}[!htb]
    \centering
    \includegraphics[width=0.8\textwidth]{img/PipeFilterPatron.PNG}
\end{figure}


\subsection*{MVC}

\begin{itemize}
\item Modelo-Vista-Controlador
\item Es conveniente usarlo cuando es necesario contar con interfaces de usuario múltiples y diferentes para los mismos datos, flexibles y cambiables.
\item El usuario solo interactúa con el modelo a través de los controladores.
\item El modelo esta desacoplado, se vuelve transportable.
\item Los ve en las vistas.
\item Viene de la mano con el patrón Observer, se encarga de propagar los cambios en el modelo.
\item La ventaja es la reusabilidad y facilidad de cambios y extensión
\item Las desventajas son que se tiene un mayor diseño y que hay que tener cuidado con las bibliotecas que ya implementan algo.
\end{itemize}

\begin{figure}[!htb]
    \centering
    \includegraphics[width=0.7\textwidth]{img/MVCPatron.PNG}
\end{figure}

\subsection*{Broker}
El Broker es un patrón de arquitectura que se utiliza para estructurar sistemas de software distribuidos con componentes desacoplados que interactúan por invocaciones de servicios remotos. Esto quiere decir que, si un componente necesita un servicio de otro que está en otra ubicación que no conoce, el broker se encarga de proporcionar la conexión. Si los componentes manejaran la comunicación por sí mismos, el sistema se enfrentaría a diversas dependencias y limitaciones.
\begin{itemize}
\item Es necesario usarlo cuando se tiene que contar con procesamiento distribuido a partir de objetos distribuidos. También cuando es necesario abstraerse de la tecnología que implementa la comunicación entre procesos.
\item Conviene siempre que se pueda no distribuir los objetos.
\item Las ventajas son la cambiabilidad, la portabilidad, y el reuso.
\item Las desventajas son la baja eficiencia, dificultad de desarrollo y que no tolera fallas.
\item Esta compuesto por el Broker, dos proxies (Stub y Skeletor) que se ocupan de la serializacion e ida y vuelta.
\end{itemize}

\begin{figure}[!htb]
    \centering
    \includegraphics[width=0.5\textwidth]{img/BrokerPatron.PNG}
\end{figure}








\section*{Clase 10 y 12}
\section{Teoría del Diseño Relacional}
\section{Forma Normal}

Son reglas para determinar si un esquema que hicimos es lo suficientemente prolijo.

\bigskip
Para que un modelo sea correcto:
\begin{itemize}
\item Tiene que preservar la información, todo lo hecho en el diseño conceptual este en el relacional.
\item Tiene que tener redundancia mínima, que no hagamos la misma cosa mas de una vez. (Por ejemplo, código postal y localidad, con el código postal ya podemos tener el otro). La redundancia no solo ocupa mas espacio, lleva a inconsistencias.
\end{itemize}

\bigskip
Cuando se hace un correcto pasaje, se tiene un esquema sin redundancia y preserva la información que queríamos. Para saber si esto lo hicimos bien necesitamos las formas normales.

\subsection*{Dependencias funcionales}
Dada una relación, podemos identificar una dependencia funcional como una restricción de que un atributo determina otro $X\rightarrow Y$. \textit{Dos tuplas con igual código postal deben de tener la misma localidad. (localidad es función de código postal) (RIGE esta dependencia) (No es cierto del otro lado)}. 

Para trabajar con esto, se hace conceptualmente, tenemos que pensar que es lo que representan los datos de mi relación. NO tenemos que guiarnos por los datos que tenemos.

\begin{itemize}
\item Cuando Y esta incluido en X, decimos que Y es trivial.
\item Las dependencias funcionales siempre se definen a partir de la semántica de los datos, no es posible inferirla con los datos.
\item Una dependencia que esta bien es legajo implica apellido, CP, localidad (surge a partir de la clave primaria/candidatas).
\item Hay varias formas normales, si una cumple con una, cumplen con las anteriores. (son subconjuntos) (\textit{Si esta en 3FN esta en 2FN, si esta en 4FN esta en 3FN, ... incluyen a las anteriores.})
\item El proceso de normalización es una forma de pasar de una forma normal a una superior que preserva toda la información. Partimos de un conjunto de dependencias funcionales que supondremos definido por el diseñador de la base de datos.
\end{itemize}



\subsection*{Primera forma normal}
Los dominios de todos sus atributos solo permiten valores atómicos y monovaluados. En el modelo relacional todos deben de ser así, por lo que no es algo que se vea en las bases de datos.
Se resolvería creando una tabla separando.

\begin{figure}[!htb]
    \centering
    \includegraphics[width=0.8\textwidth]{img/1FN.PNG}
\end{figure}


\subsection*{Segunda forma normal}
Cuando una dependencia solo depende de parte de una clave primaria (Siempre ver las claves candidatas), decimos que es una dependencia funcional parcial. Buscamos que no haya de estas.

\medskip
$X\rightarrow Y$ es parcial cuando existe un subconjunto propio A incluido en X, con A distinto de X para el cual A implica Y. Una dependencia funcional $X \rightarrow Y$ es completa si y sólo si no es
parcial.

\noindent\rule{\textwidth}{0.5pt}
\textbf{Atributo primo de una relación}: es aquel que es parte de una clave candidata de la relación\\
\noindent\rule{\textwidth}{0.5pt}

\medskip
Decimos que una relación esta en 2FN cuando todos sus atributos no primos dependen por completo de las claves candidatas. Cuando no es así, solucionamos esto realizando otra tabla, separamos lo que participa en una dependencia llevándola a otra.
\begin{enumerate}
\item Identificamos las claves candidatas.
\item Clasificamos a los atributos en primos y no primos.
\item Revisamos dependencias y resolvemos.
\end{enumerate}


\noindent\rule{\textwidth}{0.5pt}
\textbf{Descomposición}: Es tener una relación y partirla en varias mas. Es una descomposición de la relación cuando se preserva la información.\\

Si una descomposición cumple que para toda instancia posible de
R, la junta de las proyecciones sobre los $R_i$ permite recuperar la
misma instancia de relación, entonces decimos que la
descomposición preserva la información.

\medskip
Diremos que la descomposición preserva las dependencias
funcionales cuando toda dependencia funcional $X \rightarrow Y$ en R
puede inferirse a partir de dependencias funcionales definidas en
los $R_i$.

\bigskip
Una descomposición de R que cumple con ambas propiedades se denomina descomposición equivalente de R.\\
\noindent\rule{\textwidth}{0.5pt}

\begin{figure}[!htb]
    \centering
    \includegraphics[width=0.8\textwidth]{img/2FN1.PNG}
\end{figure}

\begin{figure}[!htb]
    \centering
    \includegraphics[width=0.8\textwidth]{img/2FN2.PNG}
\end{figure}

\begin{figure}[!htb]
    \centering
    \includegraphics[width=0.8\textwidth]{img/2FN3.PNG}
\end{figure}

\subsection*{Tercera Forma Normal}
\begin{itemize}
\item No se esta en 3FN cuando hay un atributo no primo que implica otro atributo no primo y no es trivial. Hay una dependencia transitiva de la clave candidata.
\item Separamos a Z y a Y en una nueva relación dejando en la relación original al Z. (Nuevas tablas)
\item Toda dependencia funcional parcial no trivial es transitiva.
\item Se esta en 3FN cuando no existen dependencias transitivas de atributos no primos.
\item Otra definición es: para toda dependencia funcional no trivial $X \rightarrow Y$, o bien X es superclave, o bien Y - X contiene sólo atributos primos.
\end{itemize}

\begin{figure}[!htb]
    \centering
    \includegraphics[width=0.8\textwidth]{img/3FN.PNG}
\end{figure}

\subsection*{Forma Normal Boyce-Codd}
\begin{itemize}
\item La FNBC dice que no quiere ver ninguna dependencia transitiva (no importa si es primo)
\item La parte izquierda de toda dependencia funcional no trivial debe de ser superclave. Se arregla separando en una tabla.
\item Esta forma resuelve casos en los que hay varias clases candidatas, pero pierde dependencias funcionales. Es un compromiso entre la mínima redundancia y perder dependencias funcionales (no siempre se pierden).
\end{itemize}

\begin{figure}[!htb]
    \centering
    \includegraphics[width=0.8\textwidth]{img/FNBC.PNG}
\end{figure}

\begin{figure}[!htb]
    \centering
    \includegraphics[width=0.8\textwidth]{img/FNBC2.PNG}
\end{figure}

\newpage

% administrador es probable que no vaya a cambiar las tareas
% operador e inspector pueden llegar a tener cambios
% ea, sistema cliente servidor que se tiene que conectar a la red con el sistema viejo
% microkernel con capas, layers, ordenadas segun el grado de abstraccion
% ea arriba quiere decir que lo reusable es el microkernel
% hay que tener en cuenta las cuestiones independientemente de la arquitectura

\textit{Se dio una clase antes en la que se explico el parcial y la parte de paradigmas. Uní lo de paradigmas con la siguiente clase por ser corto.}

\section{Paradigmas}
\begin{itemize}
\item En una manera es un modo de pensar. En nuestro caso es un enfoque, una forma de pensar un problema.
\item C, C++, Java, son lenguajes usados para modelar conceptos y relacionarlos con otros. No están pensados para procesar datos. (Las ultimas versiones están incluyendo ya cuestiones de programación funcional)
\item Programación lógica que esta orientado a la demostración de teoremas. Navega estructuras simbólicas complejas según un conjunto de reglas. (\textit{Ej. Prolog})
\item La programación funcional, algoritmos complejos resueltos a partir de la composición de funciones. Es ideal para implementar algoritmos y procesar datos.
\item La idea es resolver los distintos aspectos que nos presenta un problema con el enfoque adecuado.
\end{itemize}




\section{Programación Funcional}

\begin{itemize}
\item Es un paradigma de programación basado en funciones matemáticas y declarativo. La ejecución de programas es ir evaluando las funciones.
\item Algunos ejemplos de lenguajes son Lisp, Haskell, Erlang y esta en otros multiparadigma.
\item Es simulado al resolver problemas cuya solución natural es funcional (Git, Docket, REST, React)
\item \textit{Se ve el tema de forma más completa en Lenguajes Formales.}
\end{itemize}

\subsection*{Características}

\begin{itemize}
\item Las \textbf{funciones se tratan como cualquier otro valor}. 
\begin{itemize}
    \item Tienen expresiones literales
    \item Pueden ser argumentos a funciones
    \item Pueden ser retornadas por otras funciones
\end{itemize}
\item Las funciones tienen que ser \textbf{predecibles}, producir la misma salida para una determinada entrada. Esto se rompe con aleatoriedad, entrada/salida y estados variables (tiempo/ubicación)
\item Se tiene que tener \textbf{inmutabilidad} también. Evaluar una función no modifica el estado del programa. En lugar de actualizar valores, se crean nuevos valores. El estado mutable genera race conditiones, acciones a distancia, y cada mutación puede romper variantes. Las funciones no pueden cambiar su entrada o contexto.
\item En funcional se suele usar la \textbf{recursividad}. Se define una función en termino de si misma. Se determina un caso base y uno recursivo que nos acerca al final. Una propiedad que hay que aprovechar es la de llamados de cola. Remplaza el stack frame en lugar de crear uno nuevo. 
\item \textbf{Funciones de orden superior}, son funciones que operan sobre otras funciones como argumento. Abstraen estructuras de control, describen la intención con la que se manipulan valores. Algunos ejemplos son map (aplica una función a cada uno de los elementos), filter (devuelve los que cumplen la condición) y reduce (combina valores). Se prefiere el uso de funciones de orden superior a la recursividad.
\item El modelo de ejecución es la especificación del orden y ejecución del programa. El aplicativo evalúa cada argumento, remplaza los argumentos por su valores y evalúa el cuerpo hasta llegar a una expresión primitiva (los no funcionales usan solo esta forma). El otro orden es el normal. Busca el llamado a función mas externo y lo remplaza por su cuero sustituyendo los argumentos por su expresión. Puede evaluar fu unciones que son ciclos infinitos en orden aplicativo.
\item Hay que minimizar los efectos implícitos no locales.
\end{itemize}






\section*{Clase 11}

% justificar en 4 renglones o menos porque usamos x patron de arquitectura
% identificar cuales son los requerimientos no funcionales
% hacer la matriz donde aparecen las vistas y los requerimientos por el otro lado
% marcar en cada celda el impacto que tiene cada uno en las vistas (suma puntos justificar cada marcada)
% ej es importante la performance - x en performance y las vistas impactados - afecta en concurrencia por tal cosa
% si tenemos dudas de algo, suponemos sobre eso.
% foto del documento y/o libreta - incluir en el archivo
% a las 9 se inhabilita el espacio de entrega
% a las 7 se habilita
% son 2 horas
% minutos antes subir respuesta

\section{Presentación de una arquitectura}
La arquitectura de software es aquellas decisiones que son importantes y difíciles de cambiar.
Para armar un informe donde queremos transmitir alguna información debemos tener presente que es lo que queremos contar y a quien.
Un documento de arquitectura esta dirigido a muchos interesados, desarrolladores, lideres de proyecto, algún usuario, encargados de infraestructura, despliegue, bases de datos.

\subsubsection*{¿Como hacemos para escribir un único documento para todos ellos?}
Usamos el modelo de Vistas 4+1 que permite a cada interesado encontrar lo que necesitan saber acerca de la arquitectura.

\begin{figure}[!htb]
    \centering
    \includegraphics[width=0.6\textwidth]{img/4+1.png}
\end{figure}

\textit{Se puede leer el paper }\href{https://www.cs.ubc.ca/~gregor/teaching/papers/4+1view-architecture.pdf}{aquí}

\subsection*{Vista lógica}
\begin{itemize}
\item Tratar de poner los requisitos funcionales (los relevantes). Se aplican los principios de encapsulamiento y herencia.
\item Diagramas de estados, clases, secuencia y actividad son herramientas que vamos a necesitar en esta vista.
\item Le interesa al usuario final.
\end{itemize}


\subsection*{Vista de desarrollo (o componentes)}
\begin{itemize}
\item Se centra en la organización de los módulos de software en el ambiente de desarrollo del software.
\item Tiene en cuenta los requisitos internos relativos a la facilidad de desarrollo, administración del software, reutilización y elementos comunes, y restricciones impuestas por las herramientas.
\item Usamos diagramas de componentes.
\item Le interesa al desarrollador.
\end{itemize}



\subsection*{Vista procesos}
\begin{itemize}
\item Se centra en requisitos no funcionales tales como rendimiento, disponibilidad, tolerancia ante fallas e integridad.
\item Relacionada con la de componentes.
\item Se usan diagramas de clases
\item Le interesa al diseñador del sistema y al que lo integre.
\end{itemize}


\subsection*{Vista física (o despliegue)}
\begin{itemize}
\item Toma en cuenta los requisitos no funcionales de escalabilidad, performance y disponibilidad.
\item Le interesa al diseñador del sistema.
\end{itemize}


\subsection*{Vista de Escenarios}
\begin{itemize}
\item Para que escenarios estamos planteando nuestra solución. Son una abstracción de los requisitos mas importantes.
\item Le interesa al usuario final.
\item Busca ver el entendimiento del sistema.
\end{itemize}



\section*{Clase 13}
\section{Practica: Normalización II}
\subsection*{Algoritmo de descomposición en 3FN}

Este algoritmo es constructivo, garantiza la preservación de dependencias y la información (por incluir una clave del esquema original)
Entra un esquema R y salen varios Ri en 3FN que preservan información y dependencias funcionales.
\begin{enumerate}
\item Buscamos un conjunto minimal para F
\item Encontramos las claves de R
\item Para cada dependencia funcional $X\rightarrow A_i$ en $F_{min}$ creamos un esquema que tenga a X y a $A_i$.
\item Si ningún esquema de los generados contiene una clave de R, se crea un esquema adicional que contenga atributos que formen una superclave de R. Necesitamos encontrar por lo menos una en las relaciones resultantes.
\item Opcional, pero bueno, es unir los esquemas que tengan la misma clave primaria.
\end{enumerate}

\medskip
Un problema del algoritmo es que puedo fragmentar demasiado, tener muchos esquemas chicos.


% R(ABCDE)
% AB->C , A->D , BD->C
% donde tengo D, puedo poner A
% cc = ABE
% Fmin = A->D, BD->C
% estamos en 1ra forma normal
% lo descomponemos
% R1(AD) A->D
% R2(BCD) BD->C
% R3(ABE) cc=ABE - como la cc no estaba en ninguno creamos una relacion mas, sin esto no sirve la descomposicion


% R4(ABCDEGH)
% Fmin = c->e,g->a,b->d,h->a,h->e,bc->g,acd->g,abe->h,gh->c,gh->b
% cc = bc, beg, cdh, gh
% R1(CE) c->e
% R2(GA) g->a ESTA INCLUIDO EN R7
% R3(BD) b->d
% R4(HA) h->a POSIBLE UNION CON R5
% R5(HE) h->e
% R6(BCG) bc->g <- CLAVE CANDIDATA
% R7(ACDG) acd->g
% R8(ABEH) abe->h
% R9(GHC) gh->c TIENE MISMA CLAVE PRINCIPAL CON R10D
% R10(GHB) gh->b

\subsection*{Algoritmo de descomposición de FNBC}
Tiene la propiedad NJB (comprobación de concatenación no aditiva para descomposiciones binarias) si y solo si la df entre las relaciones R1 y R2 implican la resta de R1 con R2, o R2 con R1 esta en la clausura transitiva de F.

Cuando descomponemos puede pasar que perdamos df. Se garantiza que se mantiene la información por propiedad NJB.

Entra un esquema R y salen varios Ri en FNBC que preservan información.

Vemos en que FN esta, y ahi decidimos si corresponde aplicar el algoritmo.
\begin{enumerate}
\item Establecemos D=$\{R\}$
\item Mientras que exista un esquema Q en D que no este en FNBC
\begin{enumerate}
    \item Escogemos Q en D que no este en FNBC
    \item Encontrar una dependencia funcional $X\rightarrow Y que viole FNBC$
    \item Reemplazamos Q en D por dos esquemas. Un esquema es $R_1 =\{X\}^+ S_1 =proyección S en R_1$ y el otro $R2=\{Q - \{X\}^+ \cup X\} S_2 =proyección S en R_2$
\end{enumerate}
\item Procesar recursivamente R1 y R2
\end{enumerate}

\medskip
El resultado es que se va formando un árbol. Este conviene dibujarlo para no perderse.

Cuando descomponemos puede que perdamos
df’s, pero también puede que en los nuevos
esquemas se puedan representar otras
dependencias que no estén explicitadas en
nuestro conjunto de df´s, pero que
implícitamente existan. Entonces debemos
proyectar nuestro conjunto de df´s en el nuevo
conjunto.

% para el parcial marcar el resultado
% tener bien por lo menos un ej de cada tema

\subsection*{Algoritmo Tableau}
Detecta perdida de información. Lo hace mapeando todas las descomposiciones de R y luego se va operando con las dependencias funcionales para tratar de reconstruir la relación universal. Si podemos reconstruirla no hubo perdida de información. Este método es una simplificación del método Chase

\begin{itemize}
\item Entra una relación, sale un si o un no. 
\item Construimos una matriz con NFIL y NCOL, la columna k corresponde al atributo Aj y la fila i corresponde al esquema Ri de PROY. Los valores de las columnas, para cada una de las filas Ri de la matriz, se rellenan de la siguiente forma:
\begin{itemize}
\item Si Aj está en Ri, MATij=Aj
\item Si Aj no está en Ri, MATij=biAj
\end{itemize}
\end{itemize}

\medskip
Luego vamos aplicando las dependencias funcionales. 
Tenemos que lograr que una fila con todos los atributos.

\medskip
Donde participa cada atributo en las relaciones, pongo las letras correspondientes a esos atributos en la matriz, el resto lo relleno con bij.

\medskip
Aplicamos dependencias, donde tengo mismos valores, tiene que implicar lo mismo, remplazando los valores (priorizamos las mayúsculas). (Queremos lograr una fila con letras mayúsculas) (Si la implicación tiene mas de un atributo de un lado, buscamos esa cantidad, esos pares tienen que ser iguales)

\medskip
No importa el orden en que apliquemos las dependencias, es recursivo, ante cualquier cambio hay que volver a aplicar todas las dependencias. 


\newpage

\section*{Clase 14}
\section{Concurrencia y Transacciones}

El interés de la concurrencia surge de aprovechar la capacidad de procesamiento lo mejor posible para atender a los usuarios. Que no estén en una cola de espera.

Tenemos sistemas monoprocesador que permiten hacer multitasking (varios hilos o procesos) y sistemas multiprocesador o distribuidos (varias unidades de procesamiento o nodos que replican la base de datos en distintas unidades de procesamiento). Aun con multiprocesador la concurrencia nos puede hacer ganar. Se mejora el response time al solapar las tareas.

\noindent\rule{\textwidth}{0.5pt}

\textbf{Transacción} es una unidad lógica de trabajo compuesta por una secuencia de instrucciones atómicas (no se pueden dividir). Pueden ser operaciones de consulta o ABM. Queremos que se ejecuten por completo o que no se ejecuten. No queremos que queden por la mitad.

\medskip

La \textbf{concurrencia} es la posibilidad de ejecutar múltiples transacciones en forma simultanea (tareas). Vamos a querer aprovechar toda la capacidad de computo.
El problema que genera la ejecución concurrente es la gestión de los recursos compartidos. Al nivel de los SGBDs los recursos compartidos son los datos, a los cuales distintas transacciones quieren acceder en forma simultanea.

\noindent\rule{\textwidth}{0.5pt}

El modelo de procesamiento que usaremos es el de concurrencia solapada que considera que:
\begin{itemize}
\item Hay un solo procesador.
\item Cada transacción esta formada por instrucciones atómicas.
\item El scheduler puede suspender en cualquier momento las instrucciones.
\end{itemize}


\noindent\rule{\textwidth}{0.5pt}

\textbf{Item}: Puede representar el valor de un atributo en una fila determinada de una tabla, una fila de una tabla, un bloque del disco, una tabla. El tamaño de este se conoce como \textbf{granularidad}, afecta sustancialmente al control de concurrencia.

\noindent\rule{\textwidth}{0.5pt}

Las instrucciones atómicas serian:
\begin{itemize}
\item leer\_item(X) lee el valor de X cargándolo en memoria
\item escribir\_item(X) ordena escribir el valor que esta en memoria del item X en la base de datos
\end{itemize}

Hay que tener en cuenta que:
\begin{itemize}
\item En el medio de las instrucciones se realizan otras operaciones en el CPU que no nos afectan al análisis de concurrencia. (Por ejemplo una junta en memoria)
\item Ordenar escribir no necesariamente lleva a disco. (puede quedar en un buffer)
\end{itemize}


\subsection*{Propiedades ACID}

El gestor tiene que garantizar estas propiedades en todo momento.

\begin{itemize}
\item \textbf{Atomicidad}: Las transacciones deben ejecutarse de forma atómica, se ejecutan o no. (El usuario las ve de esta forma) (se garantiza con el log)
\item \textbf{Consistencia}: Cada ejecución debe preservar la consistencia de los datos. Se define con reglas de integridad (se deben de verificar en todo momento, ej. no puede haber mas de un gerente por departamento).
\item \textbf{Aislamiento}: El resultado de la ejecución concurrente de las transacciones debe ser el mismo que si las transacciones se ejecutaran en forma aislada una tras otra, es decir en \textbf{forma serial}. La ejecución concurrente debe entonces ser equivalente a alguna ejecución serial. Lo que percibe el usuario de datos. Si el gestor cumple con el aislamiento y miro un valor en la base de datos con un solapamiento, este debe de tener el mismo valor si fuera serial la ejecución (una transacción después de otra) (en alguno de todos los ordenes posibles que hay) (se garantiza con el log)
\item \textbf{Durabilidad}: Una vez que el SGBD informa que la transacción se completo, debe garantizarse la persistencia de la misma independientemente de toda falla que pueda ocurrir. (se garantiza con el log, deshace o rehace con lo que le dice)
\end{itemize}



Las propiedades se garantizan con mecanismos de recuperación. Buscan garantizar la visión de todo o nada de las transacciones. Se agregan algunas instrucciones especiales
\begin{itemize}
\item \textbf{begin}: se comenzó la transacción.
\item \textbf{commit}: se termino con éxito.
\item \textbf{abort}: ocurrió un error y todos los efectos de la transacción deben ser deshechos. (rollback)
\end{itemize}


Estos mecanismos de recuperación se van registrando en un archivo. Esto permite cumplir ACID. Si ocurre una falla, el log le dice como volver atrás. Siempre se tiene que escribir a disco, se suele usar uno especial de acceso rápido que se escribe en forma secuencial. No es lo mismo que estar escribiendo a la base de datos en disco. Cada cierto tiempo este log se va limpiando.

\subsection*{Anomalías}
Son situaciones que pueden violar ACID. Les decimos anomalía cuando ya no se puede arreglar, si se puede arreglar lo que paso se llama fenómeno.

\subsubsection*{Dirty read}
\begin{itemize}
\item Sucede cuando una transacción lee un valor modificada por otra transacción \textbf{que aun no se commiteo}. Se conoce también como Read uncommitted data. Es un conflicto de escritura lectura (WR). (Si aborta la transaccion que lo modifico al dato se produce la anomalia)
\item No tiene solución una vez que pasa, arruina la consistencia de la base de datos.
\item La forma de evitarlo es no permitir que la transacción haga commit hasta que la otra haga abort o commit. La otra opción es que no lea, pero es mas drástica esta solución.
\end{itemize}



\subsubsection*{Actualización perdida - lost update}
\begin{itemize}
\item Alguien escribió lo que otro ya había leído, deriva en una anomalía si después el primero vuelve a escribir (lost update) o vuelve a leer (lectura no repetible).
\item Se pisa una modificación con algo ya leído. Si la primera transacción luego modifica y escribe lo que se leyó y se pierde por otra transacción.
\item Puede pasar que esa misma transacción vuelva a leer el mismo item y tiene algo distinto, causando una rotura del aislamiento (no hay ningún orden serial en el cual pueda pasar esto). (unrepeatable read)
\item Ambas situaciones se conocen como RW (read write) seguido por otro de tipo WW o WR.
\item Ej. Una transaccion deposita 100\$ y otra retira \$100, en este caso el cambio debería de ser 0, pero puede pasar que aparezcan -100 o +100.
\item Rompe el aislamiento, no tiene un orden serial, no es serializable.
\end{itemize}


\subsubsection*{Dirty write}
Ocurre cuando una transacción T2 escribe un item que ya había sido escrito por otra transacción T1 que luego se deshace. Se conoce como WW (write-write) o Overwrite uncommited. El gestor cuando aborte va a intentar poner el valor que habia antes, pisando lo que hizo otro.


\subsubsection*{Phantom}
\begin{itemize}
\item \textit{Aparecen y desaparecen cosas en la base de datos}
\item Transacción T1 que observa un conjunto de items que cumplen una condición y luego el conjunto cambia porque algunos de sus items fueron modificados/creados/eliminados. Si esto sucede mientras T1 se esta ejecutando podría encontrarse con un conjunto que cambio.
\item Si está modificación se hace mientras T1 aún se está ejecutando, T1 podría encontrarse con que el conjunto de ítems que cumplen la condición cambió.
\item Atenta contra la serializabilidad. Para evitarlo es necesario usar locks a nivel de tabla o predicado.
\end{itemize}


\subsection*{Notación}

\begin{figure}[!htb]
    \centering
    \includegraphics[width=0.8\textwidth]{img/NotacionConcurrencia.jpg}
\end{figure}

\begin{itemize}
\item No nos importan las operaciones que realiza en memoria.
\item Un \textbf{solapamiento} entre dos transacciones T1 y T2 es una lista de m(T1) + m(T2) instrucciones, en donde cada instrucción de T1 y T2 aparece una única vez, y las instrucciones de cada transacción conservan el orden entre ellas dentro del solapamiento.
\item Cantidad de solapamientos posibles: $ \frac{(m(T_1) + m(T_2))!}{m(T_1)!\ m(T_2)!}$
\item Nos interesa ver si el solapamiento es serializable
\end{itemize}


\noindent\rule{\textwidth}{0.5pt}

\textbf{Ejecución serial} es aquella en que las transacciones se ejecutan por completo una detrás de otra en base a algún orden. Existen $n!$ ejecuciones seriales distintas.

Decimos que un solapamiento de un conjunto de transacciones
T1, T2, ..., Tn es \textbf{serializable} cuando la ejecución de sus
instrucciones en dicho orden deja a la base de datos en un
estado \textbf{equivalente} a aquél en que la hubiera dejado alguna
ejecución serial de T1, T2, ..., Tn. Nos interesa esto porque garantizan la propiedad de aislamiento de las transacciones.

\noindent\rule{\textwidth}{0.5pt}

\subsection*{Equivalencia}
\begin{itemize}
\item Por \textbf{resultados}, mi ejecución de solapamiento da lo mismo \textbf{para un estado inicial particular}.  Cuando, dado un estado inicial particular, ambos órdenes de ejecución dejan a la base de datos en el mismo estado. (Es una equivalencia mas floja)
\item Por \textbf{conflictos}, implica la de por resultados (no al revés), no depende del estado inicial. Cuando ambos órdenes de ejecución poseen los mismos conflictos entre instrucciones. (Da lo mismo, es serializable independientemente de los valores) (El solapado como el serial tienen los mismos conflictos) \textbf{No dependen del estado inicial}
\item De \textbf{vistas}, es una intermedia, más débil que la de conflictos, más fuerte que la de resultados. Cuando en cada orden de ejecución, cada lectura RTi(X) lee el valor escrito por la misma transacción j, WT(X). Además se pide que en ambos órdenes la última modificación de cada ítem X haya sido hecha por la misma transacción.
\end{itemize}


\subsection*{Conflicto}
Dado un orden de ejecución, un conflicto es un par de instrucciones (I1, I2) ejecutadas por dos transacciones distintas Ti y Tj, tales que I2 se encuentra más tarde que I1 en el orden, y que responde a alguno de los siguientes esquemas:
\begin{itemize}
\item una transacción escribe un item que otra leyó. (Ri, Wj)
\item una transacción lee un item que otra escribió. (Wi, Rj)
\item dos transacciones escriben un mismo item. (Wi, Wj)
\end{itemize}

Todo par de instrucciones consecutivas (I1, I2) de un solapamiento con transacciones distintas que no constituye un conflicto puede ser invertido en su ejecución. (Tienen que estar pegadas las instrucciones) Haciendo este swap podemos llegar a el orden serial.

\subsection*{Grafo de precedencias}
\begin{itemize}
\item Nos dice si hay conflictos, no los evita.
\item Queremos ir swappeando y llegar a que quede toda la transacción 1 a la izquierda.
\item Buscamos evaluar si un solapamiento es serializable o no.
\item Los nodos son transacciones y se agrega un arco entre los nodos i,j si existe algún conflicto de los mencionados. (WR,RW,WW)
\item Opcionalmente se etiqueta el arco con el item que causa el conflicto.
\item Si el grafo tiene ciclos, no es serializable.
\item Un orden de ejecución es serializable por conflictos si y solo si su grafo de precedencias no tiene ciclos.
\item Algoritmo de ordenamiento topológico si tengo un grafo acíclico y quiero el orden.
\end{itemize}

\subsection*{Control de concurrencia}

Tenemos dos formas:
\begin{itemize}
\item Enfoque pesimista, busca garantizar que no se produzcan ciclos de conflictos.
\item Enfoque optimista consiste en dejar hacer a las transacciones y deshacer (rollback) una de ellas si en fase de validación se descubre un conflicto.
\end{itemize}

\subsubsection*{Basado en locks}

\begin{itemize}
\item El objetivo es usar locks para bloquear a los recursos (items) y no permitir que más de una transacción los use en forma simultanea.
\item Los inserta el SGBD como instrucciones especiales.
\item No es trivial el agregado de estos.
\item (Lock y Unlock) tienen carácter bloqueante, cuando lo adquiero nadie mas puede adquirir un lock sobre el mismo item hasta que no se libere. (debe de ser atómico el lock - Sistemas Operativos)
\item Hay locks de varios tipos, los principales son de escritura (acceso exclusivo) y de lectura (acceso compartido)
\item No alcanza con usar locks por si solo.
\item \textbf{No puedo adquirir un lock luego de desbloquear un lock.} (Protocolo de lock de 2 fases - 2PL) Los unlock/lock los puedo poner donde quiera mientras no haga lock despues de un unlock
\item El protocolo nos dice que va a ser serializable, pero puede ocurrir un deadlock (un conjunto de transacciones quedan bloqueadas entre ellas bloqueadas a la espera de recursos que otra transacción posee.) Se pueden prevenir tomando todos los locks que se necesitan de forma preventiva, la desventaja es que no sabemos todo lo que vamos a necesitar. Se puede definir un ordenamiento de los recursos y timestamps también.
\item Se utiliza el grafo de alocación de recursos para detectar deadlocks. Si encontramos un ciclo en el grafo tenemos un deadlock.
\item Puede hacer inanición.
\item No resuelve generalmente la lectura sucia, las otras si.
\item Estructura de tipo árbol para buscar los bloques. (Se usan Arboles-B) (Están hechos en forma eficiente para almacenar en disco) Los bloques de datos (un puntero a estos) esta en las hojas de los arboles. Se aplican index locks para bloquear los recursos (bloquean nodos de un índice.) Bloqueo al padre, bloqueo al hijo, puedo desbloquear al padre. Cada nodo subsiguiente se puede bloquear si tengo el del padre. Un nodo deslockeado no puede volver a lockearse. Protocolo del cangrejo. Bloqueamos la raiz y vamos adquiriendo locks con los hijos hasta llegar a lo que queremos, liberando el padre a menos que sepamos que se puede partir el nodo hijo.
\end{itemize}



\subsubsection*{Timestamps}
\begin{itemize}
\item A cada transacción se le asigna una marca de tiempo que se lo asigna el gestor cuando inicia la transacción.
\item Los timestamps deben de ser únicos y determinaran el orden serial respecto al cual el solapamiento deberá ser equivalente.
\item Se permite la ocurrencia de conflictos pero siempre que las transacciones aparezcan en el orden serial equivalente.
\item No tiene deadlocks (no usa locks)
\item Para cada item X se debe de mantener:
    \begin{itemize}
    \item read\_TS(X): Es el TS(T) correspondiente a la transacción mas joven que leyo el item X. (mayor TS(T))
    \item write\_TS(X): Es el TS(T) correspondiente a la transacción mas joven que escribio el item X. (mayor TS(T))
    \end{itemize}
\item La desventaja es que impone un orden 'caprichosos' porque una transacción recibió un timestamp menor a otra nos obliga el orden serial equivalente.
\end{itemize}

Reglas:
\begin{itemize}
\item Cuando una transacción Ti quiere ejecutar R(X), si una transacción posterior Tj modifico el item, Ti deberá ser abortada. De lo contrario se actualiza el valor de read\_TS(X) y lee.
\item Cuando una transacción Ti quiere ejecutar W(X), si una transacción posterior Tj leyó o escribió el item, Ti deberá ser abortada (write too late). De lo contrario se actualiza el valor de write\_TS(X) y escribe.
\end{itemize}



\subsubsection*{Regla de escritura de Thomas}
\begin{itemize}
\item Si cuando Ti intenta escribir un item que una transacción posterior Tj ya lo escribió entonces Ti puede descartar su actualización sin riesgos siempre y cuando el item no haya sido leído por ninguna transacción posterior Ti.
\item Si yo soy anterior a una transacción que ya escribió y yo quiero escribirlo entonces si entre los dos timestamps nadie la leyó entonces puedo evitar esa actualización y no tengo que abortar esa transacción.
\end{itemize}


\subsubsection*{Control de concurrencia multiversion}
Snapshot Isolation es una de las implementaciones posibles. El objetivo es que cada transacción vea una imagen (una 'foto') de la base de datos correspondiente al instante de inicio, Que sea una foto quiere decir todo lo que estaba commiteado solo. Me deshago de la snapshot cuando termina la ultima transacción que la estaba usando.

\medskip
Cuando dos transacciones intentan modificar el mismo item gana la que commitea primero, la otra se aborta. (first-committer-wins). Por sí solo no alcanza para garantizar la serializabilidad, para eso se debe validar permanentemente con el grafo de precedencias buscando ciclos de conflictos RW y usar Locks de predicados en el proceso de detección de conflictos, para evitar la anomalía del fantasma.

\medskip
Write Skew
No se evita en snapshot isolation.

\medskip
Para evitarlo podemos usar locks de tablas o locks de predicados (mas eficiente) (Bloquea las tuplas que podrían cumplir la condición, para eso se aprovecha la estructura de árbol y el índice que se usa). Se llaman estructuras índice, (CREATE INDEX) ayudan a evitar tener que recorrer toda la tabla. (Los creamos si nos sirve para determinado atributo)

\medskip
Si tengo estas estructuras las puedo aprovechar para evitar el lock de fantasma. Bloqueamos el nodo del árbol y los bloques, por lo que nadie puede insertar o modificar lo que nos interesa. (Range lock)


\subsection*{Recuperabilidad}

\begin{itemize}
\item La serializabilidad de las transacciones ya nos asegura la propiedad de aislamiento.
\item Queremos que una vez que una transacción commiteo, no se deba deshacer. Ayuda a implementar durabilidad.
\item \textbf{Definición}: un solapamiento es recuperable si y solo si ninguna transacción T realiza el commit hasta tanto todas las transacciones que escribieron datos antes de que T los leyera hayan commiteado. (Nunca pasa que alguien commitea cuando quienes le dieron datos todavía no commitearon)
\item Hay que mirar las lecturas, si alguien lee algo que otro modifico.
\item Un SGBD no debería jamás permitir la ejecución de un solapamiento que no sea recuperable.
\item Si un solapamiento de transacciones es recuperable, entonces nunca será necesario deshacer transacciones que ya hayan commiteado. Aún así puede ser necesario deshacer transacciones que no aún no han commiteado. (puede que produzca rollbacks)
\item Para evitar los rollbacks en cascada es necesario que una transacción no lea valores que aún no fueron commiteados. Esto es más fuerte que la condición de recuperabilidad y evita la lectura sucia. (No leo algo a menos que este commiteado)
\item Los locks pueden ayudar. Protocolo de lock de dos fases estricto (S2PL) y Protocolo de lock de dos fases riguroso (R2PL). Garantizan que todo solapamiento sea serializable, recuperable y no produzca rollbacks en cascada.
\end{itemize}

\newpage

\section*{Clase 15}
\section{Bases de datos espaciales}
\begin{itemize}
\item Tenemos que ver las distorsiones que ocurren al pasar de un mapa geoide al plano. Introducción al tema de geodesia \href{https://www.ign.gob.ar/NuestrasActividades/Geodesia/Introduccion}{aquí}
\item Las bases de datos espaciales (llamadas también geográficas) permiten representar en forma eficiente objetos definidos en un espacio geométrico. 
\item Tienen 3 objetos simples (puntos, lineas, polígonos) y 2 complejos (objetos 3D, teselados).
\item Son parte de un contexto más general, los sistemas de información geográfica GIS que permite almacenar y manipular datos geográficos y capturar, analizar y presentar visualmente datos geográficos.
\end{itemize}


\subsection*{Representaciones}
Tenemos 2 representaciones posibles:

\begin{itemize}
\item En las representaciones vectorizadas los objetos se identifican con vectores que identifican los bordes. Tiene un alto nivel de detalle y es fácil de mantener y actualizar, pero tiene una pobre representación de datos continuos.
\item En las rasterizadas los objetos se proyectan sobre una matriz de celdas. Se pueden representar datos continuos (Ej. elevación), se puede hacer un análisis cuantitativo de forma fácil, es fácil de renderizar. La desventaja es que el detalle depende de la resolución y que suelen ocupar mas espacio.
\end{itemize}

En general se trabaja con los dos tipos de imágenes juntas. (basemap) (Ej, poner un camino de GPS arriba de una imagen satelital.)

\begin{figure}[htb]
    \centering
    \includegraphics[width=0.7\textwidth]{img/RepresentacionesImagenes.PNG}
\end{figure}


\subsection*{Simple Features}

\begin{itemize}
\item Define una representación textual estándar para los objetos genéricos. Se conoce como WKT (Well Known Text). Hay otra que es Well Known Binary Representation (WKB)
\item Define cómo agregar funcionalidad espacial a las bases de datos a través de una representación vectorizada, especificando con notación UML.
\end{itemize}

\begin{figure}[htb]
    \centering
    \includegraphics[width=0.7\textwidth]{img/SistemaGraficoBasesEspaciales.PNG}
\end{figure}

\begin{itemize}
\item Sistema de coordenadas (spatial reference system) SRID, usamos el 4326 en el taller (usado por el sistema GPS)
\item spatial\_ref\_sys tiene los sistemas referenciales que podemos usar.
\item Para buscar y combinar objetos geograficos se utilizan estructuras eficientes. Por ejemplo kd-trees o R-trees.
\end{itemize}


\subsection*{Taller}

\begin{itemize}
\item Creamos base de datos, vamos a extensiones. Creamos una extensión postgis y otra postgis\_raster. Luego creamos las tablas, las importamos desde el postgis.
\item Nos conectamos con la base de datos e importamos los archivos poniendo el SRID en la columna. Una vez que lo hacemos seleccionamos importar.
\item Hacemos el refresh de las tablas en la base de datos y ya podemos ver los datos.
\item Stack Builder para instalar paquetes al postgres.
\item QGis para las imágenes satelitales, instalamos el complemento y vamos a administrar base de datos donde importamos el archivo de la imagen satelital.
\end{itemize}

\begin{minted}
[
frame=leftline,
framesep=5mm,
baselinestretch=1.2,
]
{sql} 
SELECT gid, nombre_est, comuna, barrio, geom
FROM primarias
WHERE gid IN (2808, 1191, 190, 559, 1512)
\end{minted}

\subsubsection*{Obtener el área de todas las comunas expresada en hectáreas}
\begin{minted}
[
frame=leftline,
framesep=5mm,
baselinestretch=1.2,
]
{sql}
SELECT comunas, area, ST_Area(geom::GEOGRAPHY)/10000 AS area_en_ha
FROM comunas
ORDER BY 1;
\end{minted}

No esperar encontrar geometrías exactas, puede no dar lo mismo que el valor que se tiene al calcular el área.


\subsubsection*{¿Hay barrios que sean iguales a comunas? ¿Cuales?}
Vemos si hay polígonos iguales.
\begin{minted}
[
frame=leftline,
framesep=5mm,
baselinestretch=1.2,
]
{sql} 
SELECT c.comunas, c.geom
FROM barrios AS b, comunas AS c
WHERE st_equals(c.geom, b.geom)
ORDER BY 1 ASC
\end{minted}

Hay 3 en realidad pero da 1 debido a los errores en el calculo. Si queremos los otros 2 hay que 'perdonar' la superposición.

\subsubsection*{Visualice la comuna 1 dentro del entramado de barrios}
Ejemplo de unión de los barrios con la comuna.
\begin{minted}
[
frame=leftline,
framesep=5mm,
baselinestretch=1.2,
]
{sql} 
SELECT st_union(c.geom, b.geom)
FROM comunas c, barrios b
WHERE c.comunas=1
\end{minted}

\subsubsection*{Calcule la distancia entre todas las escuelas}
\begin{minted}
[
frame=leftline,
framesep=5mm,
baselinestretch=1.2,
]
{sql} 
SELECT p1.nombre_est, p2.nombre_est, ST_DistanceSphere(p1.geom,p2.geom)
FROM primarias AS p1, primarias AS p2
WHERE p1.gid < p2.gid
ORDER BY 3 DESC
\end{minted}


\subsubsection*{Obtenga un ranking de las escuelas más aisladas}
\begin{minted}
[
frame=leftline,
framesep=5mm,
baselinestretch=1.2,
]
{sql} 
SELECT grid_1, estab_1, min(dist) AS distMin
FROM distancias AS p1
GROUP BY 1, 2
ORDER BY 3 DESC
\end{minted}


\subsubsection*{Muestre las escuelas primarias y los radios censales}
\begin{minted}
[
frame=leftline,
framesep=5mm,
baselinestretch=1.2,
]
{sql} 
SELECT p.geom
FROM primarias AS p
UNION
SELECT c.geom
FROM censo_2010_cada AS c
\end{minted}

Escuelas en comuna 1 mostrando radios censales
\begin{minted}
[
frame=leftline,
framesep=5mm,
baselinestretch=1.2,
]
{sql} 
SELECT p.geom
FROM primarias AS p
WHERE p.comuna = 1
UNION
SELECT ST_union(p.geom, c.geom)
FROM primarias AS p, censo_2010_cada AS c
WHERE p.ST_Within(p.geom, c.geom) AND p.comuna = 1
\end{minted}


\section*{Clase 16 y 18}
En realidad se extendió mas clases el tema.
\section{NoSQL}

\begin{itemize}
\item No responden al modelo relacional.
\item Surgen alrededor de los 2000 con la masificación de la Web y cambios tecnológicos.
\item Necesidades de almacenar muchos datos. (Google, Amazon)
\item Se tenían requerimientos
\item Mayor escalabilidad para trabajar con grandes volúmenes de datos.
\item Mayor performance en aplicaciones Web. Surge XML y JSON, formatos fáciles de serializar y procesar.
\item Mayor flexibilidad sobre las estructuras de datos. Los SGBD relacionales son muy rígidos, agregar una columna puede ser muy costoso.
\item Mayor capacidad de distribución. Viene de la mano con escalabilidad. Se busca mayor disponibilidad y tolerancia a fallas de parte del SGBD.
\end{itemize}

\medskip
Los SGBD relacionales tiene limitaciones en cuanto a los joins de tablas, son costosos y el manejo de transacciones en forma distribuida no escala (Se vuelve difícil garantizar ACID cuando tenemos muchos nodos).

\medskip
Lectura sugerida: \href{https://www.enterpriseintegrationpatterns.com/ramblings/18_starbucks.html}{Starbucks Does Not Use Two-Phase Commit}

\medskip

Se tenían redes cada vez mas rápidas, almacenamiento mas barato, pero velocidad de procesamiento estancada. Se pueden revisar los números del cambio \href{https://colin-scott.github.io/personal_website/research/interactive_latency.html}{aquí}. Debido a esto surge la necesidad de bases de datos distribuidas.

\medskip
Las bases de datos NoSQL buscan aumentar la velocidad de procesamiento y la capacidad de almacenar información. Para ello implementan un sistema de gestión de bases de datos distribuido.

\bigskip
Para hablar de las bases de datos no sql se necesitan algunos conceptos:

\begin{itemize}
\item \textbf{Fragmentación}: repartimos el contenido en muchos nodos. En uno solo no nos entra. Es la tarea de dividir un conjunto de agregados entre un conjunto de nodos. Puede ser fragmentación horizontal (los agregados se reparten entre los nodos de manera que cada nodo almacena un subconjunto de agregados. Se asigna el nodo a partir del valor de alguno de los atributos del agregado) o vertical ( Distintos nodos guardan un subconjunto de atributos de cada agregado. Todos suelen compartir los atributos que conforman la clave - no es lo mas común) (En un solo nodo no nos entra, toma mucho tiempo)
\item \textbf{Replicación}: si se cae uno que lo pueda manejar otro. Tiene varias ventajas, provee backup, repartir la carga de procesamiento, y garantizar la disponibilidad del sistema si se caen algunos nodos. El problema que genera es la consistencia de los datos. Cuando se usan solo como backup se dice replica secundaria. Cuando pueden hacer procesamiento también se conoce como replica primaria. La replica introduce el problema de consistencia, que un mismo item de datos tenga el mismo valor en todas las replicas. (A prueba de fallas)
\item \textbf{Búsqueda (lookup)}: saber donde esta lo que necesito, que nodo tiene determinado dato. No esta este problema en una base de datos relacional. Se usan tablas de hash distribuidas.
\item \textbf{Métodos de consistencia}: Que pasa si un mismo dato tiene valor distinto en distintos nodos. Puede suceder cuando los usuarios están modificando los datos. Tiene que haber una forma de controlar esto, que los usuarios se pongan de acuerdo o no pase. Surge por la replicación,
\item \textbf{Métodos de acceso (Access method)}: Como llego dentro del nodo a la información que quiero de forma rápida. (LSM Trees y estructuras) diferenciales)
\end{itemize}



\subsection*{Clasificación}
\subsubsection*{Clave-Valor}

\begin{itemize}
\item Almacenan vectores asociativos o diccionarios, es decir pares.
\item Las claves son únicas (no puede haber 2 pares con la misma clave).
\item Ejemplos son Dynamo y Redis.
\item Operaciones: PUT un nuevo par, UPDATE par de la clave, DELETE par de la clave, GET par de la clave
\item Las ventajas que tienen son: son simples, veloces (eficiencia de acceso sobre la integridad), y escalables (se provee replicación y se pueden repartir las consultas entre nodos).
\item El objetivo es consultar y guardar grandes cantidades de datos.
\end{itemize}

\bigskip
\textbf{Dynamo}\\
\begin{itemize}
\item Es el key-value de Amazon
\item Esta orientada a una arquitectura orientada a servicio (SoA)
\item La base de datos esta distribuida en un server cluster que posee servidores web, routers de agregación y nodos de procesamiento
\item Para consistencia usa un modelo llamado consistencia eventual. Tolera pequeñas inconsistencias (valores distintos).
\item De lookup usa un método de hashing consistente que reduce la cantidad de movimientos de pares necesarios cuando cambia la cantidad de nodos S. Hace que agregar nodos sea sencillo con un impacto mínimo.(no tiene nada que ver con la consistencia, solo minimiza la cantidad de movimientos)
\item La función de hash consistente a partir de una clave $k$ devuelve un valor $h(k)$ entre 0 y $2^M -1$ en donde M representa la cantidad de bits del resultado (hashing). Este valor para un par representa en que nodo se almacena (tabla de hash distribuida)
\item Al identificador de cada nodo de procesamiento (generalmente, su dirección IP) se le aplica la misma función de hash. Los nodos se van organizando virtualmente en una estructura de anillo por hash creciente. (para que si se cae alguno no reorganiza todo, si se cae alguno, el nodo siguiente se ocupa de los datos del nodo que se cayo)
\item Los nodos tienen un listado de preferencia, indica cuales nodos va a replicar. Si se cae un nodo, otro nodo que no esta en este listado va a recibir los datos para mantener (el único movimiento que hay)
\end{itemize}


\subsubsection*{Orientadas a Documentos}

\begin{itemize}
\item Un documento es un agregado, la unidad estructural que contiene la información. 
\item No tenemos un esquema rígido (columnas).
\item Un agregado es un conjunto de objetos relacionados que se agrupan en colecciones para ser tratados como unidad y ser almacenados en un mismo lugar. \textit{Un post de FB con sus comentarios}
\item Generalmente se representan con JSON, XML, YAML...
\item Ejemplos son MongoDB, RavenDB...
\end{itemize}


\bigskip
\textbf{MongoDB}\\
\begin{itemize}
\item Se basa en hashes para identificar los objetos.
\item No usa esquemas pero se le pueden definir.
\item Documentos son JSON.
\item Organiza los datos de una base de datos en colecciones que contienen documentos.
\item No esta pensado para operaciones de junta, en general se pone todo junto en los documentos.
\item La agregación se implementa con un pipeline, uno va definiendo las operaciones a realizar en el. La salida de uno es la entrada de otro.
\item Sharding es el modelo distribuido de procesamiento. Se basa en el particionamiento horizontal de las colecciones en chunks que se distribuyen en nodos llamados Shards. La idea de estos es que estén descentralizados.
\end{itemize}

El esquema de MongoDB se puede ver como:
\begin{figure}[!htb]
    \centering
    \includegraphics[width=0.7\textwidth]{img/MongoDB.PNG}
\end{figure}


\begin{itemize}
\item Son los shards (fragmentos) que se distribuyen los chunks, los routers que reciben las consultas, y los servidores de configuración sobre los shards y routers.
\item Mongo usa para particionar las colecciones una shard key. Es un atributo o conjunto de atributos.
\item Es posible tener colecciones sharded y otras unsharded. Las unsharded se almacenan en un shard particular del cluster. (El shard primario)
\item No es posible desfragmentar una colección ya fragmentada.
\item Usar el sharding permite disminuir el tiempo de respuesta en sistemas con alta carga de consultas al distribuir el procesamiento entre varios nodos y ejecutar consultas sobre conjuntos de datos muy grandes. El objetivo es que la base de datos sea escalable.
\item MongoDB nos permite que cada shard este replicado. Nos sirve para backup y responder consultas.
\item El esquema de replicas es de master-slave with automated failover. Las replicas eligen entre si un master con un algoritmo distribuido. Si el master falla los slaves elijen a un nuevo master.
\item Todas las operaciones de escritura se hacen sobre el master. Los slaves están como respaldo.
\item Los clientes pueden especificar para mandar las consultas de lectura a los nodos secundarios.
\item Durante operaciones de agregación, se puede dar el caso de que los nodos realice operaciones de forma paralela, y cuando no se pueda seguir de forma paralela (se necesita la información de los otros nodos) se le pasan al router para que lo termine la operación.
\end{itemize}


\subsubsection*{Wide Column}

\begin{itemize}
\item Evolución de las bases de datos de clave/valor.
\item Un valor particular de la clave primaria junto con todas sus columnas asociadas forma un agregado análogo a la fila de una tabla. Pero además, estas bases permiten agregar conjuntos de columnas en forma dinámica a una fila, convirtiéndola en un agregado llamado fila ancha (wide row, el nombre quedo como wide column)
\item Ejemplos son Google BigTable, Apache Cassandra...
\end{itemize}

\bigskip
\textbf{Cassandra}\\
\begin{itemize}
\item No es estrictamente orientada a columnas.
\item Usa esquemas.
\item Tiene una arquitectura de share-nothing, no hay un estado compartido centralizado, todos los nodos son pares. Permite que sea muy escalable.
\item Optimizado para ofrecer una alta tasa de escrituras.
\item Cada columna es un par clave-valor asociado a una fila.
\item Cada esquema puede estar distribuido en varios nodos.
\item Es obligatorio definir una clave primaria.
\item Cuando en una fila las columnas se repiten identificadas por el valor que toman las columnas clave, se dice que la fila se convirtió en una wide row (fila ancha).
\end{itemize}

\begin{figure}[!htb]
    \centering
    \includegraphics[width=0.7\textwidth]{img/EsquemaCassandra.PNG}
    \caption{Esquema simple de Cassandra}
\end{figure}

\begin{figure}[!htb]
    \centering
    \includegraphics[width=0.7\textwidth]{img/EsquemaWideRow.PNG}
    \caption{Esquema con Wide Row}
\end{figure}

\begin{itemize}
    \item La clave primaria se divide entonces en clave de partición y de clustering. (La primaria permite identificar a la fila todavía)
    \item Toda la wide-row se almacenara contigua en disco y la clave de clustering nos determina el ordenamiento interno de las columnas.
    \item Restricciones sobre la clave primaria: los valores deben ser comparados por igual contra valores constantes en los predicados, si una columna que forma parte de una clustering key es usada como predicado, deben usarse las restantes columnas también.
    \item No existe el concepto de junta.
    \item No existe el concepto de integridad referencial.
    \item Hay que pensar de antemano que consultas se van a realizar para diseñar las tablas. Se busca que cada consulta se resuelva accediendo a una unica column family, que los resultados esten en una unica particion, respetar las reglas del uso de la clave primaria.
    \item Usa una estructura LSM-Tree que mantiene parte de los datos en memoria para diferir cambios sobre el indice en disco.
    \item Se busca acceder en forma secuencial al disco.
\end{itemize}

\subsubsection*{Basadas en grafos}
\begin{itemize}
\item Los elementos principales son nodos y arcos (ejes)
\item Estas bases resultan útiles para modelar interrelaciones complejas entre las entidades.
\item Mantienen una referencia a sus nodos adyacentes.
\item Se almacena como una lista de adyacencias
\item Sirve para encontrar un patrón de nodos conectados entre si, un camino entre nodos, la ruta mas corta, calcular medidas de centralidad asociadas a los nodos.
\end{itemize}


\bigskip
\textbf{Neo4j}\\
\begin{itemize}
\item Tiene soporte para ACID
\item Formada por nodos que pueden tener distintos labels. 
\item Dentro de cada label el nodo tiene un conjunto de propiedades. Esta estructura no es rígida.
\item Los ejes son siempre direccionales, si se quiere trabajar con no dirigidos no se crean las interrelaciones en ambos sentidos.
\item Un patrón (pattern) puede especificarse a través de un nodo y sus propiedades, una interrelación y sus propiedades, o un camino y sus propiedades. A cada patrón podemos darle un nombre. 
\item Las funciones de agregación se realizan en el RETURN.
\end{itemize}

\subsection*{Consistencia}
\subsubsection*{Consistencia secuencial}

\begin{itemize}
\item Es el provisto por las bases de datos centralizadas.
\item Se dice que una base de datos distribuida tiene consistencia secuencial cuando “el resultado de cualquier ejecución concurrente de los procesos es equivalente al de alguna ejecución secuencial en que las instrucciones de los procesos se ejecutan una después de otra”
\item Garantizar consistencia secuencial es costoso, ya que requiere de mecanismos de sincronización fuertes que aumentan los tiempos de respuesta.
\end{itemize}

\subsubsection*{Consistencia causal}

\begin{itemize}
\item Busca captar eventos que puedan estar causalmente relacionados
\item Si un evento b fue influenciado por un evento a, la causalidad requiere que todos vean al evento a antes que al evento b.
\item Dos eventos que no estén causalmente correlacionados se dicen concurrentes. No se tienen restricciones en este caso.
\end{itemize}

\subsubsection*{Consistencia eventual}

\begin{itemize}
\item En algún momento los nodos se van a poner de acuerdo. Se basa en que son pocos los procesos que realizan modificaciones o escrituras mientras que la mayor parte solo lee.
\item Deja que pase y después se vea como resolverlo, eventualmente todas las replicas son consistentes.
\item Dynamo provee este sistema. 
\begin{itemize}
\item Se definen dos parámetros adicionales debido a que puede darse el caso de que se lea un valor desactualizado.
\item $W \leq N$: Quorum de escritura, se devuelve que la lectura fue exitosa cuando otros $W-1$ nodos confirman el valor. W=2 es el mínimo.
\item $R \leq N$: Quorum de lectura, se devuelve éxito cuando se tienen R nodos distintos. Generalmente R = 1 es suficiente. Valores mayores de R brindan tolerancia a fallas como corrupción de datos ó ataques externos, pero hacen más lenta la lectura
\item En Dynamo hay que tener en cuenta que no hay funciones de agregación, se implementan a mano o se usan herramientas que envuelven a Dynamo (Ej Spark).
\end{itemize}
\end{itemize}

\subsection*{Arboles}
\subsubsection*{Log Structured Merge Trees LSM-trees}
\begin{itemize}
\item Ofrece escrituras secuenciales (El B-Tree es de acceso aleatorio), es mucho más rápido que tener que posicionarse en el disco. (El B-Tree es peor para escritura, el costo de escritura es alto porque me estoy moviendo todo el tiempo en disco.)
\item La desventaja es que a veces se pierde tiempo cuando los datos no están cacheados. La lectura es más torpe. (El B-Tree es mejor para lectura)
\item Es usado por Cassandra, Mongo...
\end{itemize}

\subsection*{Map Reduce}
\begin{itemize}
    \item Es una técnica que brinda un marco flexible para el procesamiento paralelo de grandes volúmenes de datos.
    \item Divide la entrada en partes mas pequeñas que puedan ser ejecutadas por unidades de procesamiento para luego integrar el resultado a la salida.
    \item Map devuelve una secuencia de clave/valor
    \item Reduce recibe una clave con sus valores y devuelve valores.
\end{itemize}


\subsection*{Teorema CAP}
Es un teorema que postula la imposibilidad de que un sistema distribuido garantice simultaneamente el maximo nivel de:
\begin{itemize}
    \item Consistencia (Se muestre un único valor, requiere mucha sincronización)
    \item Disponibilidad (Cada consulta que llegue a un nodo no caido devuelva un resultado sin errores)
    \item Tolerancia a fallas (Que se pueda responder a consultas aun cuando algunos nodos estan caidos.)
\end{itemize}

A lo sumo podemos ofrecer 2 de las 3 garantias


\bigskip
En la realidad no es factible garantizar que una red no se
particione, con lo cual nuestro sistema deberá necesariamente
ser tolerante a particiones. Nos obliga entonces a encontrar una
solución de compromiso entre consistencia y disponibilidad

\subsection*{BASE}
\begin{itemize}
 \item (BA) Disponibilidad básica (basic availability): El SGBD distribuido
está siempre en funcionamiento, aunque eventualmente puede
devolvernos un error, o un valor desactualizado.

 \item (S) Estado débil (soft state): No es necesario que todas los nodos
réplica guarden el mismo valor de un ítem en un determinado
instante. No existe entonces un “estado actual de la base de datos”

 \item (E) Consistencia eventual (eventual consistency): Si dejaran de
producirse actualizaciones, eventualmente todos los nodos réplica
alcanzarían el mismo estado.
\end{itemize}


\newpage

\section*{Clase 17}
\section{Practica: MongoDB}

\begin{itemize}
\item Orientada a documentos
\item Sirve para información que subimos y bajamos por la Web.
\item Usamos el formato JSON
\item Los números son enteros o decimales.
\item Es muy libre la estructura, puedo tener una colección donde cada documento sea distinto. No estamos obligados a tener esquemas.
\item La desventaja es que voy a tener que realizar muchos más chequeos.
\item Lo natural es tener todo junto.
\item La ausencia de los datos no es algo que esta mal. A veces da mas información, por ejemplo la gente que no dice cuanto gana se sabe que generalmente es gente de ingresos altos.
\item En SQL se ponen nulos para rellenar, en Mongo es con JSON igual, pero al trabajar con los datos hay que tener en cuenta las faltas. Si queremos evitar los nulos en SQL usamos nuevas tablas que se relacionen por la clave.
\end{itemize}

'Renombrando' queda:

\begin{itemize}
\item Una tabla es una colección ahora.
\item Una fila es un documento.
\item Una columna es un campo. (field)
\item Un join es un documento embebido.
\end{itemize}

\subsection*{Sentencias CRUD}

\begin{itemize}
\item db.<collection>.find(<query>, <projection>)
\item db.user.find(\{firstName:"Juan"\})
\item El query es como el WHERE
\item La proyección para decir que campos queremos ver. Con 0 no los vemos, con 1 si. Es como un SELECT.
\item El id nos los trae siempre, si no lo queremos le tenemos que poner 0 si o si.
\item Podemos hacer búsquedas más complejas con \$regex
\item Tenemos and, or, eq, gt,ge,lt, lte.
\item Se le puede poner un limite de la cantidad de elementos que queremos que nos traiga.
\item Varias cosas más que se pueden ver de la documentación...
\end{itemize}

\subsection*{Taller}

\subsubsection*{Mostrar el país cuya capital es Vienna}

\begin{verbatim}
db.countries.find({'capital':'Vienna'},{_id:0})
\end{verbatim}

\subsubsection*{Mostrar los países con un área igual o menor a 20}

\begin{verbatim}
db.countries.find({area: {$lte : 20}}, {_id : 0, "name.common" : 1})
\end{verbatim}

\subsubsection*{Mostrar todos los países de América}

\begin{verbatim}
db.countries.find({region: "Americas"}, {_id : 0, "name.common" : 1})
\end{verbatim}

\subsubsection*{Mostrar todos los países de América con un área menor a 100}

\begin{verbatim}
db.countries.find({region: "Americas", area: {$lt : 100}}, {_id : 0, "name.common" : 1})
\end{verbatim}

\subsubsection*{Mostrar los países que estén en Europa o que use el euro (código: EUR) como moneda oficial}

\begin{verbatim}
db.countries.find({$or:[{region:'Europe'},{currency:'EUR'}]},{_id:0,"name.common":1})
\end{verbatim}

\subsubsection*{Mostrar las capitales de los países que no usen el dolar estadounidense (código: USD) ni el dolar canadiense (código: CAD) como moneda oficial}

\begin{verbatim}
db.countries.find({$nor:[{'currency':'USD'}, {'currency':'CAD'}]},{_id:0})
\end{verbatim}


\subsubsection*{Mostrar los países que limitan con Francia o Polonia}

\begin{verbatim}
db.countries.find({'borders': {$in: ['FRA','POL']}},{'name.common':1})
\end{verbatim}

\subsubsection*{Mostrar los países que limitan con Francia y Polonia}

\begin{verbatim}
db.countries.find({\$and :[{'borders': {$in: ['FRA']}}, {'borders': {$in: ['POL']} }]},{'name.common':1})
\end{verbatim}

\subsubsection*{Mostrar los países cuyo nombre oficial contenga la palabra “Republic” y que tengan exactamente 3 países limítrofes}

\begin{verbatim}
db.countries.find({$and: [ {"name.official": {$regex: /^Republic/} },
{ borders: {$size: 3 } } ] }, {_id:0, "name.common":1})
\end{verbatim}


\subsubsection*{Mostrar la cantidad de libros de m´as de 400 paginas y que tengan un solo autor}

\begin{verbatim}

db.books.find({$and:[{pageCount:{$gt:400}},{authors:{$size:1}}]})

\end{verbatim}

\subsubsection*{Mostrar en forma alfabética por t´ıtulo, los libros que en su t´ıtulo o en su descripción corta contengan la palabra “web”}

\begin{verbatim}
db.books.find({$or:[{title:{$regex:"web.*"}},{shortDescription:{$regex:"web.*"}}]})
        .sort({title:1})    

\end{verbatim}

\subsubsection*{Mostrar el nombre y la cantidad de paginas de 12 libros publicados, ordenados descendentemente por su cantidad de paginas}

\begin{verbatim}
db.books.find({_id:0,title:1,pageCount:1})
        .sort({pageCount:-1})
        .limit(12)
\end{verbatim}

\subsubsection*{Mostrar los libros publicados entre 2008 y 2010 (inclusive), ordenados ascendentemente por su id.}

\begin{verbatim}
db.books.find({$and:[{publishedDate:{$gte:new Date("01-01-2008")}},{publishedDate:{$lt:new Date("01-01-2011")}}]})

\end{verbatim}







\noindent updateOne remplaza el JSON directamente, si queremos evitarlo hay que usar un parámetro adicional.

\begin{verbatim}
db.countries.deleteOne({"name.common":{$regex:"Falkland"}})
\end{verbatim}

\newpage

\section*{Clase 19}
\section{Practica: Agregación en MongoDB}

\begin{itemize}
\item Los operadores van dentro del aggregate.
\item \$match nos permite realizar una query igual al find, es el filtrado
\item \$group agrupa por uno o mas atributos aplicando funciones de agregacion (\$sum,\$avg,\$first)
\item \$sort para ordenar los resultados
\item \$limit para limitar la cantidad de resultados dados
\item \$sample para que me de una muestra de resultados
\item \$project para proyectar campos, decir si queremos que estén o no.
\item \$unwind abre los documentos, si un documento tiene un vector, crearía un documento nuevo identico al del vector donde en vez del vector se tiene un item (por cada uno de los que tenga). Sirve si quiero reorganizar la información.
\item Accedo a los valores del documento con \$, agrupo poniendo el atributo en el \_id
\end{itemize}



\subsubsection*{1 Hallar el nombre de usuario y la cantidad de retweets para los tweets con más de 100000 retweets}

\begin{verbatim}
db.tweets.aggregate(
[{
$match: {
    retweet_count: {
        $gt: 100000
    }
}},
{
    $project: {
        _id: 0,
        "user.name": 1,
        retweet_count: 1
    }
}])
\end{verbatim}

\subsubsection*{2 Hallar los usuarios que hayan contestado (is reply status id != null) algún tweet, sea en español y sea entre las 12 y 13 hs. Ayuda: ISODate("2019-06-26T13:00:00.000Z")}

\begin{verbatim}
 db.tweets.aggregate(
 [{
 $match: {
    lang: 'es',
    created_at: {
        $gt: ISODate('1029-06-26T12:00:00.000Z'),
        $lt: ISODate('1029-06-26T13:00:00.000Z')
    },
    in_reply_to_user_id: {
        $exists:true -----otra opcion $ne:null 
    }
}},
{
$project: {
    _id:0,
    "user.name":1
    }
}])
\end{verbatim}

\subsubsection*{3 Hallar un tweet de fecha más antigua}
\begin{verbatim}
[{
$sort: {
    created_at:-1
}},
{
$limit: 1
}]
\end{verbatim}

\subsubsection*{4 Mostrar de los 10 tweets con más retweets, su usuario y la cantidad de retweets. Ordenar la salida de forma ascendente}
\begin{verbatim}
[{
$sort: {
    retweet_count:1
}},
{
$limit: 10
},
{
$project: {
    "user.name":1,
    _id:0,
    retweet_count:1
    }
}]
\end{verbatim}

\subsubsection*{5 Encontrar los 10 hashtags más usados}
\begin{verbatim}
[{
$unwind: {
    path: "$entities.hashtags",
    preserveNullAndEmptyArrays: false
}},
{
$group: {
    _id: "$entities.hashtags.text",
    cant: {
        $sum: 1
    }
}
}, 
{
$sort: {
    cant: -1
    }
},
{
$limit: 10
}]
\end{verbatim}


\subsubsection*{6 Encontrar a los 5 usuarios más mencionados. (les hicieron @)}
\begin{verbatim}
[{
$unwind: {
    path: "$entities.user_mentions",
    }
}, 
{$group: {
    _id: "$entities.user_mentions.id",
    cant: {
        $sum: 1
    }
    }
}, 
{$sort: {
    cant: -1
    }
},
{
$limit: 10
}]
\end{verbatim}

\subsubsection*{7 Para los tweets que empiezan con un hashtag, mostrar su índice y el hashtag}

Se puede ver con el índice también para no usar regex
\begin{verbatim}
[{
$unwind: {
    path: "$entities.hashtags",
    }
},
{$match: {
    text: {$regex: RegExp('^#')}
    }
}, 
{$project: {
    _id:0,
    "entities.hashtags.text":1,
    "entities.hashtags.indices":1,
    }
}]
\end{verbatim}

\subsubsection*{8 Por cada usuario obtener una lista de ids de tweets y el largo de la misma}
\begin{verbatim}
[{$group: {
  _id: "$user_id",
  ids: {$push : "$_id"},
  cant : {$sum : 1}
}}, {}]
\end{verbatim}


\subsubsection*{9 Hallar la máxima cantidad de retweets totales que tuvo algún usuario}
\begin{verbatim}
[{$group: {
  _id: {id:"$user_id",name : "$user.name"},
  retweets: {$sum : "$retweet_count"}
}}, 
{$sort: {
  retweets: -1
}}, 
{$limit: 1}]

\end{verbatim}

\subsubsection*{10 Hallar el usuario no verificado que tuvo mayor cantidad de tweets retweeteados}

\begin{verbatim}
[
    {
        '$match': {
            'user.verified': False, 
            'retweet_count': {
                '$gt': 0
            }
        }
    }, {
        '$group': {
            '_id': {
                'id': '$user_id', 
                'screen_name': '$user.screen_name', 
                'name': '$user.name'
            }, 
            'retweets': {
                '$sum': 1
            }
        }
    }, {
        '$project': {
            '_id': 0, 
            'id': '$_id.id', 
            'screen_name': '$_id.screen_name', 
            'name': '$_id.name', 
            'retweets': 1
        }
    }, {
        '$sort': {
            'retweets': -1
        }
    }, {
        '$limit': 1
    }
]

\end{verbatim}

\subsubsection*{11 Hallar para cada intervalo de una hora cuantos tweets realizó cada usuario}

\begin{verbatim}
[
    {
        '$group': {
            '_id': {
                'created_at': {
                    '$hour': '$created_at'
                }, 
                'id': '$user_id', 
                'screen_name': '$user.screen_name', 
                'name': '$user.name'
            }, 
            'count': {
                '$sum': 1
            }
        }
    }, {
        '$project': {
            '_id': 0, 
            'id': '$_id.id', 
            'screen_name': '$_id.screen_name', 
            'name': '$_id.name', 
            'count': 1
        }
    }
]

\end{verbatim}



\newpage

\section*{Clase 20}
\section{Recuperabilidad}

\begin{itemize}
\item En la vida real se pueden producir diferentes tipos de fallas.
\item Fallas del sistema por errores de software o hardware que detienen la ejecución de un programa.Ej fallas de segmentación, división por cero, fallas de memoria.
\item Fallas de aplicación, provienen desde la aplicación que utiliza la base de datos. Cancelación o vuelta atras de una transacción
\item Fallas de dispositivos, provienen de un daño físico
\item Fallas naturales externas que provienen afuera del hardware. Caídas de tensión, terremotos, incendios
\end{itemize}

Supongamos que se produce una falla no catastrófica en el momento en que se están ejecutando muchas transacciones. La base de datos debe ser llevada al estado inmediato anterior al comienzo de la transacción. Para esto se utiliza el log que se fue armando.

\subsection*{El log}
El gestor de recuperación de la SGBD almacena generalmente los siguientes registros:

\begin{itemize}
\item BEGIN T
\item WRITE T, X, xold, xnew
\item READ T, X
\item COMMIT T
\item ABORT T
\end{itemize}

Siendo T una transaccion, X un item, xold el valor viejo, xnew el valor nuevo.

\medskip
El gestor se guia principalmente por dos reglas:

\begin{itemize}
\item WAL: Write Ahead Log, antes de escribir un item a disco, se escribe el log a disco.
\item FLC: Force Log Commit, antes de realizar un commit se escribe el log a disco
\end{itemize}

\subsection*{Técnicas}

\begin{itemize}
\item Actualización inmediata: Los datos se guardan en disco lo antes posible, necesariamente antes del commit de la transacción.
\item Actualización diferida: Los datos se guardan en disco después del commit de la transacción.
\end{itemize}

\newpage
\subsection*{Algoritmos de recuperación}
Asumen que los solapamientos son recuperables y evitan ROLLBACKS en cascada. Si no se cumple esto hay que realizar pasos adicionales.

\subsubsection*{UNDO (Actualización inmediata)}
Todo valor $V_{old}$ asignado por una transacción que ya commiteo debe ser guardado en el log en disco antes de que su modificación por parte de otra transacción sea guardada en disco (flushed).

\medskip
Los pasos del algoritmo son:

\begin{figure}[!htb]
    \centering
    \includegraphics[width=0.8\textwidth]{img/PasosUndo.PNG}
\end{figure}

\begin{itemize}
\item Cuando hay un reinicio, miro el log y veo si hay transacciones no commiteadas. Si encuentro alguna la tengo que deshacer. (Se considera que la transacción commiteo cuando el registro en el log queda en disco)
\item El gestor se tiene que asegurar que antes del commit el dato en memoria vaya a disco.
\end{itemize}

\medskip
En el reinicio los pasos son:

\begin{figure}[!htb]
    \centering
    \includegraphics[width=0.8\textwidth]{img/PasosUndoReinicio.PNG}
\end{figure}


Si vuelve a fallar se ejecuta devuelta, no cambia el resultado.


\newpage
\subsubsection*{REDO (Actualización diferida)}
Antes de realizar el commit todo nuevo valor $v$ asignado por la transacción debe ser salvaguardado en el log en disco.. (Me obliga solo a guardar el log primero, no el dato.) (Se postergan los accesos a disco, las fallas no deberian de ser algo comun)

\medskip
Pasos:
\begin{figure}[!htb]
    \centering
    \includegraphics[width=0.8\textwidth]{img/PasosRedo.PNG}
    \caption{Se usa el valor nuevo, no el viejo!}
\end{figure}


\medskip
Cuando el sistema se reinicia se siguen los siguientes pasos.
\begin{figure}[!htb]
    \centering
    \includegraphics[width=0.8\textwidth]{img/PasosRedoReinicio.PNG}
    \caption{Se recorre de lo mas viejo a lo mas nuevo. Nos tiene que quedar lo mas nuevo en disco}
\end{figure}

\begin{itemize}
\item Si la transacción falla antes del commit, no se deshace nada (solo se abortan las transacciones encontradas). Si falla después de haber escrito el COMMIT en disco, hay que rehacer toda la transacción (no sabemo si estan los valores nuevos en disco).
\item En el algoritmo REDO, una transacción puede commitear sin haber guardado en disco todos sus ítems modificados.
\item Ante una falla previa posterior al commit, entonces, será necesario reescribir (REDO) todos los valores que la transacción había asignado a los ítems (Implica recorrer todo el log de atrás para adelante aplicando cada uno de los WRITE.)
\item Si no hay checkpoints hay que revisar todo el archivo, puede haber quedado una transaccion desde el comienzo sin terminar.
\end{itemize}


\newpage
\subsubsection*{UNDO/REDO}

Los pasos:
\begin{figure}[!htb]
    \centering
    \includegraphics[width=0.8\textwidth]{img/PasosUndoRedo.PNG}
    \caption{Escribimos ambos valores, el viejo y el nuevo. Después de eso podemos mandar el valor de X a disco cuando queramos (antes o después del commit, tenemos ambos datos).}
\end{figure}

En el reinicio:
\begin{figure}[!htb]
    \centering
    \includegraphics[width=0.8\textwidth]{img/PasosUndoRedoReinicio.PNG}
    \caption{Deshago lo que no tiene commit, rehago lo que tiene commit}
\end{figure}

\begin{itemize}
    \item Undo o Redo me obligan a trabajar a nivel de item, no de bloque/fila.
    \item Es mucho mas flexible que los otros dos.
    \item El log se vuelve mas grande, el doble
    \item Me puedo despreocupar del uso del buffer del disco con este metodo
\end{itemize}

\newpage
\subsection*{Puntos de control}

\begin{itemize}
\item Al reiniciar el sistema no sabemos hasta donde hay que retroceder en el archivo de log. Para esto, se utilizan checkpoints. (Indica que todo hasta ese punto ha sido almacenado en disco)
\item Hay checkpoints inactivos y activos.
\item Los inactivos implican que no se toman mas transacciones mientras que se esta ejecutando el copiado a disco. Se usa el registro CKPT
\item Los activos toman transacciones usando dos registros para el checkpoint, BEGIN CKPT y END CKPT
\end{itemize}


\subsubsection*{Undo}

\textbf{Inactivo}

\begin{enumerate}
\item Cuando se quiere hacer, se dejan de aceptar nuevas transacciones
\item Espera a que terminen todas las transacciones comenzadas, hagan su commit
\item Escribe CKPT en el log y lo vuelca a disco
\end{enumerate}

Si el sistema se cae después del CKPT los items ya están guardados, durante la recuperación solo debemos deshacer las transacciones que no hayan hecho commit hasta encontrar el CKPT. Se podría borrar todo lo que esta atrás del CKPT en el log.

\bigskip

\textbf{Activo}

\begin{enumerate}
\item Escribimos un registro (BEGIN CKPT tactivas) con el listado de todas las transacciones  activas hasta el momento. (Seguimos recibiendo transacciones)
\item Esperamos a que todas esas transacciones activas hagan su COMMIT
\item Una vez que hicieron su commit, terminamos el checkpoint. No nos importa si aparecieron mas transacciones en el medio (END CKPT)
\end{enumerate}

Si ocurre un problema y queremos hacer un rollback:
\begin{itemize}
\item Si encontramos un END CKPT, retrocedemos hasta el BEGIN CKPT. Ninguna transaccion incompleta puede haber comenzado antes.
\item Si encontramos un BEGIN CKPT solamente podemos deshacer las que están en la lista o que no terminaron. 
\end{itemize}


\subsubsection*{REDO}

\textbf{Activo}

\begin{enumerate}
\item Escribimos un registro BEGIN CKPT con el listado de todas las transacciones activas hasta el momento
\item Hacer el volcado a disco de todos los ítems que hayan sido modificados por transacciones que ya commitearon. (las que ya habian terminado)
\item Escribir (END CKPT) en el log y volcarlo a disco.
\end{enumerate}

Si ocurre un problema y queremos hacer un rollback:
\begin{itemize}
\item Que encontremos primero un registro (END CKPT).En ese caso,
deberemos retroceder hasta el (BEGIN , Tx ) más antiguo del
listado que figure en el (BEGIN CKPT) para rehacer todas las
transacciones que commitearon. Escribir (ABORT, Ty ) para
aquellas que no hayan commiteado.

\item Que encontremos primero un registro (BEGIN CKPT).Si el
checkpoint llego sólo hasta este punto no nos sirve, y entonces
deberemos ir a buscar un checkpoint anterior en el log
\end{itemize}


\subsubsection*{UNDO/REDO}

\textbf{Activo}

\begin{enumerate}
\item Escribimos un registro con el listado de todas las transacciones activas hasta el momento
\item Volcamos a disco todos \textbf{los items} modificados antes del BEGIN CKPT
\item Escribimos END CKPT y lo llevamos al disco.
\end{enumerate}

En la recuperación es posible que debamos retroceder hasta el
inicio de la transacción más antigua en el listado de
transacciones, para deshacerla en caso de que no haya
commiteado.

Puedo rehacer cosas que estén después del BEGIN CKPT si se llego al END. Lo de antes de llevo a disco. 




\newpage

\section*{Clase 21}
\section{Seguridad}

Seguridad de la información: Es el conjunto de procedimientos y medidas para proteger a los componentes de los sistemas de información

\medskip
Implica:

\begin{itemize}
\item \textbf{Confidencialidad}: La información no sea ofrecida a individuos que no están autorizados a verla.
\item \textbf{Integridad}: Asegurar la correctitud durante el ciclo de vida de la información. Por ejemplo alterar las notas
\item \textbf{Disponibilidad}: Asegurar que la información este disponible cuando las personas autorizadas la requieran.
\item \textbf{No repudio}: Alguien que accedió a la información no pueda negar haberlo hecho. Que quede un registro del acceso.
\end{itemize}

\medskip
Hay que cubrir la información en distintos niveles (Personas, aplicaciones, red, OS, SGBD, archivos). Ahora nos focalizamos en lo que puede hacer el SGBD.

\subsection*{Control de acceso basado en roles (RBAC)}
Esta basado en definir roles para las distintas actividades y funciones desarrolladas por los miembros de una organización, con el objetivo de regular el acceso de los usuarios a los recursos disponibles

\medskip
Los elementos/entidades son:
\begin{itemize}
\item Usuarios: son las personas
\item Roles:  son conjuntos de funciones y responsabilidades
\item Objetos: son aquello a ser protegido
\item Operaciones: son las acciones que pueden realizarse sobre objetos
\item Permisos: Acciones concedidas o revocadas a un usuario o rol sobre un objeto determinado.
\end{itemize}

\newpage
\begin{figure}[!htb]
    \centering
    \includegraphics[width=0.8\textwidth]{img/ModeloRBAC.png}
\end{figure}


Soporta tres principios:

\begin{itemize}
\item Criterio del menor privilegio posible, si un usuario no va a realizar una operación, no debería de tener los permisos para realizarla.
\item División de responsabilidades, nadie debe de tener suficiente privilegios para usar el sistema en beneficio propio. Necesita de roles excluyentes. \textit{Por ejemplo, Que el pago de una factura a un proveedor requiera la aprobación de un empleado de Pagos y la carga de los datos de los proveedores deba realizarla un empleado de Compras.}
\item Abstracción de datos: los permisos son abstractos, dependen del objeto en cuestión.
\end{itemize}

\newpage
\subsection*{Autenticación y permisos en SQL}

En Postgres el esquema queda de la siguiente forma:

\begin{figure}[!htb]
    \centering
    \includegraphics[width=0.8\textwidth]{img/ModeloRBACPostgres.png}
\end{figure}


\begin{itemize}
\item Postgres permite definir políticas a nivel de fila. Solo puede haber algunas filas según mis permisos al hacer un SELECT.
\item Solo veo hasta donde tengo permisos al ver la tabla.
\item Solo el dueño puede crear o borrar la tabla. Se puede cambiar el dueño. Lo mismo pasa con las databases y los esquemas. DROP y ALTER no son privilegios que pueden darse.
\item El dueño de un objeto tiene todos los privilegios sobre el objeto, y puede extenderlos a los usuarios que desee.
\item Los superusuarios tienen todos los privilegios sobre todos los objetos, y son irrevocables.
\end{itemize}

\subsection*{SQL Injection}
Son una de las falla de seguridad más frecuentes en sistemas de bases de datos relacionales. Se manipula a nivel aplicación las variables de entrada de una consulta SQL. Sus consecuencias pueden ser graves.


\medskip
La solución es hacer todo lo que se pueda para evitarlo.

\begin{itemize}
\item Una solución es usar una función de escape sobre los parámetros.
\item Otra es aprovechar los mecanismos de control de acceso. Desde afuera se conectan con algún usuario especifico al que controlamos los permisos.
\item Validar los parámetros, castear cada dato al tipo correspondiente antes de anexarlo a la consulta.
\item Usar consultas parametrizadas.
\end{itemize}

\newpage

\section*{Clase 22}
\section{Practica: Neo4J}

\begin{itemize}
\item Orientada a grafos
\item Forma distinta de modelar.
\item Aplica a una gran cantidad de problemas, relaciones, distancias
\item Se tienen 2 elementos, nodos y arcos
\item Las ventajas son
    \begin{itemize}
    \item Que se pueden ver patrones de nodos conectados entre si
    \item Encontrar caminos entre nodos
    \item La ruta mas corta
    \item Medidas asociadas al grafo (centralidad,proximidad,periferia)
    \end{itemize}
\end{itemize}



\subsection*{Arcos}
\begin{itemize}
\item Tienen un único tipo
\item Pueden tener propiedades
\item Tienen una dirección si o si
\item Vinculan 2 nodos o con si mismo
\end{itemize}


\subsection*{Consultas}

\begin{verbatim}
MATCH patron
[WHERE filtros]
RETURN respuestas
[ORDER BY expresiones]
[SKIP cant LIMIT cant]  
\end{verbatim}


\begin{itemize}
\item patrón:estructura que quiero
\item filtros, como sql
\end{itemize}


\subsection*{Patrón de nodos}
\begin{itemize}
\item (alias:etiqueta:{filtros})
\item Es case-sensitive
\item Manejo de labels: No tengo que definir esquemas
\end{itemize}


\subsection*{Taller}
\subsubsection*{1 Muestre en orden alfabetico, las 10 primeras pelıculas del genero (genre) Science Fiction}
 
\begin{verbatim}
MATCH (n:Movie)
WHERE n.genre = "Science Fiction"
RETURN n
ORDER BY n.title
LIMIT 10
\end{verbatim}

\subsubsection*{2 Muestre 20 peliculas del genero “Drama” y cuyo estudio contenga la palabra “Pictures”, devolviendo el titulo de la pelicula y el nombre del estudio}

\begin{verbatim}
MATCH (n:Movie)
WHERE n.genre = "Drama" AND n.studio=~".*Pictures.*"
RETURN n.title, n.studio
LIMIT 20
\end{verbatim}

Otra opción es con n.studio CONTAINS "Picture"

\subsubsection*{3 Muestre el nombre y fecha de nacimiento de 20 personas que sean actores Y directores, y que tengan fecha de nacimiento registrada}

\begin{verbatim}
MATCH (n:Actor:Director)
WHERE n.birthdate IS NOT NULL
RETURN n.name, n.birthdate
LIMIT 20
\end{verbatim}

\subsubsection*{4 Muestre el grafo de las películas filmadas por Oliver Stone}

\begin{verbatim}
MATCH (os:Director{name: 'Oliver Stone'})-[d:DIRECTED]-(pelis:Movie)
RETURN os, pelis LIMIT 20
\end{verbatim}

\begin{verbatim}
MATCH (p:Director{name=”Oliver Stone”}) - [d:DIRECTED] -> (m:Movie)
RETURN p,d,m
\end{verbatim}

\subsubsection*{5 Muestre el grafo de todos los co-actores de Tom Hanks}

\begin{verbatim}
MATCH (th:Actor{name=”Tom Hanks”}) -[a: ACTS_IN]-> (m: Movie),
	  (m) <- [a2: ACTS_IN]- (ca:Actor)
RETURN  th, a, a2, ca
\end{verbatim}


\subsubsection*{6 Muestre 10 actores que están a distancia 4 de Kevin Bacon}

\begin{verbatim}
MATCH (n:Actor)-[*4]-(kb:Actor{name:'Kevin Bacon'})
RETURN n
LIMIT 10
\end{verbatim}

\subsection*{Alias de subgrafos}

\begin{itemize}
\item Para usar funciones sobre los subgrafos
    \begin{itemize}
    \item Length
    \item Relationships
    \item Nodes
    \end{itemize}
\item Se pueden usar como booleanos
\item Se pueden poner varios subgrafos en el MATCH
\item Se pueden tener OPTIONAL MATCH que es similar a un outer join
\item Caminos mas cortos, son funciones que reciben un subgrafo: shortestPath, allShortestPaths
\end{itemize}


\subsubsection*{7 Muestre las películas que Clint Eastwood dirigió y en las que no actuó}

\begin{verbatim}
MATCH (d:Actor:Director{name:'Clint Eastwood'})-[:DIRECTED]->(m:Movie)
WHERE NOT (d)-[:ACTS_IN]->(m) 
RETURN d,m
\end{verbatim}



\subsubsection*{8 Muestre un camino más corto de Meg Ryan a Kevin Bacon.}

\begin{verbatim}
MATCH r = shortestPath( (n{name:'Meg Ryan'})-[*]-(m{name:'Kevin Bacon'})  )
RETURN r
\end{verbatim}

\subsubsection*{9 Muestre los actores que trabajaron en la trilogía completa de The Matrix ("The Matrix", "The Matrix Reloaded" y "The Matrix Revolutions")}

\begin{verbatim}
MATCH (a:Actor)-[:ACTS_IN]-(m:Movie{title:'The Matrix'}),
      (a)-[:ACTS_IN]-(m2:Movie{title:'The Matrix Reloaded'}),
      (a)-[:ACTS_IN]-(m3:Movie{title:'The Matrix Revolutions'})
RETURN a
\end{verbatim}

Otra opción

\begin{verbatim}
MATCH (a:Actor)-[:ACTS_IN]-(m:Movie{title:"The Matrix"})
WHERE (:Movie{title:"The Matrix Revolutions"})<-[:ACTS_IN]-(a)-[:ACTS_IN]->(:Movie{title:"The Matrix Reloaded"})
RETURN a
\end{verbatim}

\subsection*{Funciones de agregación}
\begin{itemize}
\item Ya incluirlas en el RETURN nos devuelve lo que pedimos
\item Si devuelvo algo sin una función de agregación, se agrupa por esta propiedad.
\item WITH reemplaza al return (Seria como el HAVING en sql)
\item Permite manipular resultados antes de pasar a la siguiente parte de la consulta
\item Se puede usar para filtrar agregación
\item Con WITH tengo que definir alias a lo que voy a estar trabajando con AS.
\end{itemize}



\subsubsection*{10 Obtenga el nombre de los 10 directores que han trabajado con más actores (y la cantidad de actores)}

\begin{verbatim}
MATCH (d:Director)-[:DIRECTED]-(m:Movie), (a:Actor)-[:ACTS_IN]-(m)
RETURN d.name, COUNT(DISTINCT a) AS cantidad
ORDER BY cantidad DESC
LIMIT 10
\end{verbatim}


\subsubsection*{11 ¿Quien/es es/son el/los actor/es más joven/es de la base de datos?}
\begin{verbatim}
MATCH (a:Actor) 
WITH MAX(a.birthday) AS mas_joven
MATCH (j:Actor) 
WHERE j.birthday=mas_joven
RETURN j
\end{verbatim}

\subsubsection*{12 Devuelva el nombre de personas que hayan actuado en al menos 10 películas y hayan dirigido al menos 5}

\begin{verbatim}
MATCH (person:Person)-[acts:ACTS_IN]->(:Movie)
WITH COUNT(acts) as cant_actuaciones, person
MATCH (person:Person)-[directs:DIRECTED]->(:Movie)
WITH COUNT(directs) as cant_direcciones, cant_actuaciones, person
WHERE cant_direcciones >= 5 AND cant_actuaciones >= 10
RETURN person.name, cant_direcciones, cant_actuaciones
\end{verbatim}


\subsubsection*{13 ¿A qué distancia se encuentra el actor más viejo de Kevin Bacon?}
\begin{verbatim}
MATCH (a:Actor) WITH MIN(a.birthday) AS mas_viejo
MATCH (j:Actor) WHERE j.birthday=mas_viejo
MATH g=shortestPath((j)-[*]-(k:Actor{name:'Kevin Bacon'})
RETURN length( g )
\end{verbatim}


\subsection*{ABM}
\begin{itemize}
\item No es lo mismo crear nodos que arcos
\item Para nodos es CREATE y el patron del nodo
\item Para arcos es con MATCH para vincular los nodos y un CREATE para crear el arco, por cada arco
\end{itemize}


\newpage

\section*{Clase 23}
\section{Procesamiento y Optimización de consultas}


\begin{figure}[!htb]
    \centering
    \includegraphics[width=0.8\textwidth]{img/ProcesamientoSQL.PNG}
    \caption{Representación ideal}
\end{figure}

\begin{itemize}
\item Parser y traductor: la parsea a la consulta, se fija que sea sintácticamente correcta y compile, la traduce a una expresión de álgebra relacional
\item El optimizador analiza la expresión para llegar a un plan de ejecución: Determina con que métodos de acceso y algoritmos ejecuta lo determinado. Utiliza estadísticas para analizar esto.
\item Después queda solo evaluar para obtener el resultado de la consulta.
\end{itemize}



\subsection*{Información de catalogo}
Es utilizada para estimar costos y optimizar las consultas.

\begin{itemize}
\item n(R): Cantidad de filas que tiene una relación R.
\item B(R): Cantidad de bloques en disco que ocupa la tabla R
\item V(A,R): Cuantos valores distintos toma el atributo A en la relación R. La variabilidad no la mantiene siempre actualizada, n y B si.
\item F(R): Cantidad de tuplas de R que entran en un bloque (factor de bloque) F(R) = n / B . (Esta también  1/F(R) : Que fracción de un bloque ocupa una tupla.)
\end{itemize}

El SGBD corre un proceso que realiza un análisis de estas estadísticas. También almacena información de los índices. En este caso que altura y que largo tienen.


\begin{itemize}
\item Heigth(I(A,R)): Altura del índice de búsqueda I por el atributo A de la relación R.
\item Length(I(A,R)): Cantidad de bloques que ocupan las hojas del índice I.
\end{itemize}

Mantener actualizadas las estadísticas en cada operación ABM puede ser costoso, por lo que se hacen con cierta periodicidad.

\medskip


\begin{itemize}
\item La expresión se optimiza a través de una heurística y utilizando reglas de equivalencia, obteniendo un plan de consulta.
\item Cada plan de consulta lógico se materializa para obtener un plan de ejecución en el que se indica el procedimiento físico. Estructuras de datos a usar, índices, algoritmos, etc
\item Para comparar distintos planes de ejecución necesitamos estimar el costo. Hay varios costos posibles que se pueden tomar, pero el que mas afecta en las bases de datos relacionales centralizadas termina siendo el del acceso a disco.
\end{itemize}

\subsection*{Índices}

\begin{itemize}
\item Los índices son estructuras de búsqueda almacenadas y actualizadas por el SGBD que agilizan la búsqueda de registros a partir del valor de un atributo o conjunto de atributos.
\item Puede implementarse con distintas estructuras de datos. (Arboles, tablas de hash)
\item Se clasifican en 3 tipos:
\begin{itemize}
\item Primarios, el índice es sobre el campo de ordenamiento clave de un archivo ordenado de registros.
\item De clustering, un índice por atributo no clave que ordena al archivo
\item Los secundarios, son sobre campos que no son de ordenamiento del archivo.
\item Solo se puede tener un único índice primario o de clustering.
\end{itemize}
\item No se tiene un SQL estándar para la creación de los índices, pero en general es CREATE [UNIQUE] INDEX $nombreIndice$ ON $tabla$
\end{itemize}


\subsection*{Costos}
\subsubsection*{Selección}

\begin{itemize}
\item Partimos de una selección básica (condición simple)
\item Las estrategias que se tienen son:
\item \textbf{File scan}: se recorre o escanea el archivo buscando aquellos que cumplan la condición. El costo es B(R) (Se recorrieron todos los bloques)
\item \textbf{Index scan}: usan un índice para la búsqueda.
\begin{itemize}
\item Búsqueda con \textbf{índice primario}: solamente una tupla puede satisfacer la condicion, por lo que el costo va a ser (si se usa un arbol) $Height(I(A,R)) + 1$ o (si se usa un hash) $1$
\item Búsqueda con \textbf{índice de clustering}: Cuando Ai no es clave pero se tiene un índice de ordenamiento por el (clustering). Las tuplas están contiguas en los bloques, los cuales estarán disjuntos. El costo resulta entonces $Height(I(A,R)) + \lceil \frac{n(R)}{V(A_i,R) \cdot F(R)} \rceil = Height(I(A,R)) + \lceil \frac{B(R)}{V(A_i,R)} \rceil $
\item Búsqueda con \textbf{índice secundario}: cuando Ai no tiene un índice de clustering pero existe un índice secundario asociado a el. El costo es $Height(I(A,R)) + \lceil \frac{n(R)}{V(A_i,R)} \rceil $
\end{itemize}
\item Estos cálculos pueden extenderse a otras formas de comparaciones. (Mayor, menor, distinto, etc)
\item Si la selección involucra la conjunción de varias condiciones simples, pueden adoptarse distintas estrategias.
\item Si uno de los atributos tiene un índice asociado se aplica primero esta condición y luego se selecciona del resultado aquellas tuplas que cumplen las demás condiciones. (El costo es solo el índice)
\item Si hay un índice compuesto que tiene a varios atributos se utiliza este índice y se selecciona las que cumplan.
\item Si hay índices simples para varios atributos se pueden utilizar los índices y después interesar los resultados.
\item Si la selección tiene una disyunción de condiciones simples se aplican las mismas por separado y luego se unen los resultados. Si no tiene un índice alguna se hace por fuerza bruta.
\end{itemize}


\subsubsection*{Proyección}

\begin{itemize}
\item X es el atributo de la proyección. ($X'$ para referirnos con los duplicados)
    \begin{itemize}
    \item Si X es superclave no hay que eliminar duplicados, por lo que el costo es B(columna)
    \end{itemize}
\item Si X no es superclave, hay que eliminar duplicados. Para esto se puede ordenar la tabla o usar una estructura de hash. Eliminar duplicados depende mucho de la memoria
    \begin{itemize}
    \item Para ordenar, depende de la cantidad de memoria que tengamos. Si tenemos $B(\pi_{X'}(R)) \leq Mem$ se puede hacer directo en memoria, solo tenemos de costo leerlo todo. Si no nos alcanza la memoria, hay que hacer un ordenamiento externo (en disco) cuyo costo es: $ 2 \cdot B(R) \cdot \lceil log_{Mem-1}(B(R)) \rceil - B(R) $ El logaritmo representa la cantidad de etapas del sort. (Cargamos una cantidad de bloques a memoria, los ordenamos y los guardamos en otro lugar en disco. Cargo los siguientes bloques, ordeno y guardo. Así sucesivamente hasta recorrer todo. Hecho esto se hace una etapa de merge, me fijo cual es el mas chico de todos los primeros de los recorridos y avanzo en esa partición cargando el siguiente. Se hace esto hasta recorrer todo y que nos quede el archivo ordenado. ) La formula cuenta cuantas veces moví cada dato de disco a memoria. (La resta es sin el almacenamiento final en disco, si lo quiero contar se saca)
    \item Usando un hash, si entra en memoria se puede hacer el hashing en memoria directo. Si no, se usa el hashing externo (en disco), se hacen particiones en disco y se eliminan duplicados ahí. Dos elementos van a la misma partición. El costo es $B(R) + 2 B(\pi_{X'}(R))$ (Aplicamos la función de hash a cada bloque, al resultado cuando se llene el bloque con un valor lo movemos devuelta a disco dependiendo de lo que de. Cuando se termino cargamos a memoria estos bloques y tomamos los valores unicos (deberian de estar en el mismo bloque)) (En memoria tenemos una cantidad de bloques mas uno para traer el bloque de disco y aplicar la función de hash a las filas.)
    \end{itemize}
\item Si la consulta SQL no tiene DISTINCT el resultado va a ser siempre B(R)
\end{itemize}


%clase 24
\subsubsection*{Operaciones de conjuntos}

\begin{itemize}
\item \textbf{Unión}: Se hace una tabla temporal que tenga los datos ordenados (en memoria o externo) y sin duplicados de cada parte de la unión y después se unen ambas tablas (merge) de datos ordenados. (Algoritmo dado en Algoritmos 1) Vienen ordenados por como lo hace el gestor, no es algo que le dijimos que haga. Como el estándar no obliga la forma de salida, lo da ordenado. El costo es $cost(Ordenar_M(R)) + cost(Ordenar_M(S)) + 2 B(R) + 2 B(S) $
\item \textbf{Intersección}: La intersección es similar, si hay cosas iguales, avanzo ambas y devuelvo una. Si hay destinos avanzo la menor.
\item \textbf{Diferencia}: Se devuelven las que están en R pero no en S. Se aplica el algoritmo.
\end{itemize}

\subsubsection*{Junta}
Existen distintos métodos para calcularla.

\begin{itemize}
\item \textbf{Loops anidados}: Lo que hace es probar todos los bloques contra todos los bloques. Carga a memoria 2 bloques (uno de cada lado) y compara todos contra todos. Una vez hecho eso, saca uno (el de la derecha), carga otro y vuelve a comparar. (Así hasta que pase por toda la relación de la derecha). Nos conviene usar como pivote el mas chico. El hecho de comparar todos con todos es lo que hace la junta muy costosa, escala mal. El costo es de $min( B(R) + B(R) B(S), B(S) + B(R) B(S) )$ en el peor caso, en el mejor caso se cargan ambas relaciones a memoria y ese es el único costo. 
\item \textbf{Único loop}: Vamos a tener un índice para acceder a los datos. Si alguna tabla tiene un índice para el atributo de junta, voy a poder recorrer a través de el. Se agarra la primera fila de S, voy al indice de R y encuentro el puntero al bloque. Junto las filas. Hay que recorrer cada fila de S. Desaparece el termino cuadrático. La junta por índice no mejora porque tenga mas memoria. El costo si el indice es primario es de $ B(S) + n(S) (Height(I(A,R)) + 1) $, si es de clustering $B(S) + n(S) (Height(I(A,R)) + \lceil \frac{B(R)}{V(A_i,R)} \rceil )$, y si es secundario $B(S) + n(S) (Height(I(A,R)) + \lceil \frac{n(R)}{V(A_i,R)} \rceil )$
\item \textbf{Sort-merge}: Consiste en ordenar los archivos de cada tabla por el/los atributos de junta, si entra en memoria el costo es solo cargarlo, si no, se usa un algoritmo de sort externo. Se ordena R y se vuelve a guardar en disco. El costo es $ B(R) + 2 B(R) \cdot \lceil log_{Mem-1}(B(R)) \rceil + B(S) + 2 B(S) \cdot \lceil log_{Mem-1}(B(S)) \rceil $
\item \textbf{Método de junta hash}: La idea es particionar las tablas R y S en m grupos uysando una funcion de hash aplicada sobre los atributos de junta X. Si m fue escogido de manera que para cada par de grupos (Ri, Si) al menos uno entre en memoria y sobre un bloque de memoria para hacer desfilar al otro grupo, el costo total es de $3 ( B(R) + B(S) )$
\end{itemize}


\subsubsection*{Pipelining}

Sucede cuando el resultado de un operador puede ser procesado por el operador siguiente en forma parcial sin la necesidad de esperar a que haya terminado. Se suele usar siempre que sea posible.

Al calcular el costo de dos operadores anidados $O_2(O_1(R))$
debemos considerar que en caso de utilizar pipelining no será
necesario tener todos los bloques de la salida de $O_1$ para
comenzar a calcular $O_2$. En particular, no tendremos que
materializar toda la salida de $O_1$ por falta de espacio en memoria.



\subsection*{Estimación de la cardinalidad}
\begin{itemize}
\item Queremos estimar el tamaño que tiene algún resultado intermedio. Debe de ser precisa, fácil de calcular y no depender de la forma en que se calculo la relación intermedia.
\item Para la \textbf{selección} se usa la variabilidad. La estimación es de $\frac{n(R)}{V(A_i),R}$. (Se llama selectividad de $A_1$ en R a $\frac{1}{V(A_i),R}$) Otra opción es usar un histograma, es útil para valores concretos.
\item Para la \textbf{junta} la estimación es de $\frac{n(R) \cdot n(S)}{max(V(R,B), V(S,B))}$.
\end{itemize}

\newpage
\subsection*{Reglas de equivalencia}

\begin{figure}[!htb]
    \centering
    \includegraphics[width=0.8\textwidth]{img/ReglasEquivalencia.PNG}
\end{figure}

\begin{figure}[!htb]
    \centering
    \includegraphics[width=0.8\textwidth]{img/ReglasEquivalencia2.PNG}
\end{figure}


\subsection*{Heurísticas de optimización}
\begin{itemize}
    \item Se realizan las selecciones lo mas temprano posible
    \item Productos cartesianos por juntas
    \item Proyectar para descartar atributos no usados lo antes posible (entre selección y proyección priorizar la selección)
    \item Si hay varias juntas realizar la mas restrictiva primero
\end{itemize}

\newpage

\section*{Clase 24}
\section{Practica: Costos}

\begin{itemize}
\item Medimos en unidades de I/O a disco
\item Cualquier otro proceso que se haga en comparación tiene un costo ínfimo.
\item Cuando tenemos varias tablas en un join, empezamos por la que nos va a dejar menos resultados.
\item Tabla mas pequeña es la que usamos en el ciclo externo (algoritmo para la junta)
\item Es para disminuir el costo
\item Analizar arboles que estén ramificados a izquierda o derecha (left-deep o right-deep) para acotar las posibilidades.
\item \textbf{R-trees}: Indexan objetos geométricos a través de su minimum boungind rectangle (MBR) Pongo en las hojas del R-tree los MBR de las partes que contienen cada hoja.
\item \textbf{Quad-trees}: El espacio se descompone en cuadrantes disjuntos. Cada cuadrante lo subdividimos si tiene un mínimo de puntos. Se usa para indexar puntos geométricos o datos rater en dos dimensiones.
\end{itemize}

\subsection*{Índices}

\begin{itemize}
\item Su objetivo es acceder de forma mas rápida a la búsqueda de datos. También es hacer cumplir las reglas de negocio (Ejemplo no hay padrones duplicados. La estructura sirve para hacerla cumplir)
\item Cluster: El orden del índice con el orden de los datos coinciden. El orden en que vienen los datos del índice es como vienen los datos en el archivo
\item No cluster: Es cuando no coinciden los ordenes, se cruza del índice a los datos con el camino que toma otra entrada del índice.
\item La misma consulta se comporta de forma distinta dependiendo del tipo de índice.
\item Simples: Están creados sobre un único atributo. I(A), I(B)
\item Compuestos: Están creados por 2 o mas atributos. I(A,B) != I(B,A). El orden de los atributos dentro del índice es importante para la optimización de la consulta. Conviene poner en el primer lugar el atributo mas discriminante.
\item Pueden tener duplicados o no, con duplicados seria que nos manda a un lugar donde hay varias entradas. (Ejemplo índice por comuna, una comuna nos manda a una entra con muchos alumnos.)
\item Podemos indexar parcialmente las tablas. (De forma completa o densa significa que cada registro de la tabla tiene su representación en el índice)
\item El índice sirve al momento de acceder a la estructura física, si proyectamos lo perdimos, ya no nos sirve.
\end{itemize}









\newpage

\section*{Clase 25}
\section{Data warehousing}

Se tienen datos estaticos y dinamicos en las empresas, el volumen de datos se relaciona con el tamaño de la organizacion. Debe de ser previsto para dimensionar la base de datos correctamente. Esta capacidad se conoce capacidad transaccional. (Dar a basto para procesar los datos)


\subsection*{OLTP}
\begin{itemize}
\item On-line transaction processing
\item Datos que se generan dinámicamente
\item Capacidad transaccional es la capacidad para procesar el volumen de datos
\end{itemize}


\subsubsection*{Arquitectura de 3 capas}
\begin{itemize}
\item Presentación, la interfaz en la que el usuario carga sus datos.
\item Lógica, es la capa intermedia, hace de servidor. Recibe la consulta y la ejecuta.
\item Capa de datos, los nodos de almacenamiento.
\end{itemize}


\subsection*{Olap}
\begin{itemize}
\item On-line analitical processing
\item Se empezó a hacer relevante extraer datos de información ya almacenada.
\item Se tiene registro de los datos y la capacidad de procesarlos, surgió la idea de aprovecharlos para tomar decisiones.
\item Para esto hace falta reducir la cantidad de datos y poder expresar consultas mas complejas
\item Codd propone 12 reglas, se muestran 6:
    \begin{itemize}
    \item Vista conceptual multidimensional, mantener los datos en una matriz, cada dimension representa un atributo.
    \item Manipulación intuitiva de datos, se debe de poder diseñar la vista conceptual con una interfaz amigable.
    \item Accesibilidad, combinar datos que vienen de distintos lugares (mediarlos)
    \item Extracción batch e interpretativa, se debe de poder almacenar el resultado del procesamiento batch. Debe de poder mostrarse también.
    \item Modelos de análisis, poder responder consultas de tipo estadístico o predicativo.
    \item Arquitectura cliente-servidor, debo de poder conectarme y ver el data warehouse
    \end{itemize}
\end{itemize}


Las aplicaciones OLAP generalmente se ejecutan con una copia paralela de la base de datos que se conoce como data warehouse (integran datos provenientes de fuentes de datos heterogéneas).



\subsection*{Modelado conceptual}
\begin{itemize}
\item Nuestro objetivo es definir una serie de medidas numéricas sobre un conjunto de atributos a los que denominaremos dimensiones
\item Debemos definir cuales serán las dimensiones del data warehouse
\item A la medida numérica asociada a un valor concreto de cada una de las dimensiones la llamamos hecho.
\item El diagrama de estrella permite comunicar la estructura de hechos y dimensiones de un data warehouse.
\end{itemize}

\begin{figure}[!htb]
    \centering
    \includegraphics[width=0.7\textwidth]{img/EjemploDiagramaEstrella.PNG}
    \caption{Ejemplo diagrama estrella}
\end{figure}

\subsection*{Modelo lógico}
\begin{itemize}
\item La tabla de hechos guardará información sumarizada, de acuerdo a las dimensiones que nos interesará explorar.
\item La forma de almacenamiento depende de las implementaciones. Ej MOLAP, ROLAP, HOLAP
\end{itemize}

\begin{figure}[!htb]
    \centering
    \includegraphics[width=0.7\textwidth]{img/EjemploMOLAP.PNG}
    \caption{Ejemplo MOLAP}
\end{figure}


\subsection*{Operaciones}
\begin{itemize}
\item Operación roll-up consiste en agregar los datos de una dimensión subiendo un nivel en su jerarquía.  Por ejemplo, si tenemos el total de ventas por ciudad y producto, podríamos hacer un roll-up de la ciudad para obtener un total por provincia y producto.
\item Operación drill-down es la contraria a roll-up.
\item La operación de pivoteo consiste en producir una tabla agregada por un subconjunto del conjunto de dimensiones en cierto orden deseado.
\item La operación de slicing y dicing permiten realizar una selección en una dimensión o mas.
\end{itemize}



\newpage

\section*{Clase 26}
\section{Practica: Resolución de consultas}
\begin{itemize}
\item Una consulta SQL se transforma en un árbol de consultas. Buscamos ver que árbol utilizar.
\item Para calcular los costos nos interesa conocer los meta datos
    \begin{itemize}
    \item n: cuantas filas tiene
    \item B: cuantos bloques tiene
    \item F: factor de bloque, cuantas filas entran en un bloque
    \item V: variabilidad de un atributo en una tabla (valores distintos)
    \item Height: altura del índice
    \end{itemize}
\item Buscamos realizar las selecciones lo antes posible, reemplazar productos cartesianos y selecciones por juntas, proyectar atributos lo antes posible priorizando la seleccion y realizar las juntas más restrictivas primero.
\item El objetivo es reducir el tamaño de relaciones intermedias
\end{itemize}


\subsection*{Selección}

\begin{itemize}
\item File Scan: Costo B(R), se recorre toda la tabla.
\item Index Scan: height(idx) + variante dependiendo del índice
\end{itemize}

Siempre acordarse de redondear hacia un valor entero.

\subsection*{Junta}
\begin{itemize}
\item Loops anidados: La cantidad de memoria ayuda mucho en los loops anidados
\item Único loop: Es cuando tenemos un índice que podemos aprovechar. Arranca por el que no tiene el índice como para usar.
\item Junta Hash: Surge de leer 3 veces ambas tablas 3 (B(R) + B(S))
\end{itemize}


Si es la misma tabla y se utiliza el mismo atributo de junta, solo carga la tabla una vez. No tendría sentido cargar devuelta la tabla.


\subsection*{Pipelining}
\begin{itemize}
\item Se concatena la salida de un operador con la entrada de otro. El resultado de un operador será la entrada de otro operador.
\item Puedo llegar a hacer un ahorro en el costo (puede aumentar el uso de memoria). Por ejemplo no tener que hacer una proyección extra.
\item En vez de procesar completamente un operador y luego el siguiente, puede irse procesando parcialmente el resultado a medida que se arma
\end{itemize}



\subsection*{Estimación de cardinalidad}
\begin{itemize}
\item Para poder usar las fórmulas de operadores que trabajan con resultados de otros operadores, precisamos conocer la información que nos dan los metadatos
\item Precisamos estimar la cantidad de filas n(R) 
\item Algunas operaciones modifican F(R), se recalcula B(R)
\end{itemize}



\subsection*{Histogramas}
\begin{itemize}
\item Guardan la frecuencia de ciertos valores.
\item Mantener actualizado el histograma es costoso.
\item En general tiene una buena performance
\end{itemize}






Se trata de no utilizar operaciones en las dos tablas a la vez, se trabaja en una, se hace el join y se sigue trabajando.


\newpage

\end{document}
