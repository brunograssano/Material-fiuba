\documentclass[titlepage,a4paper]{article}

\usepackage{a4wide}
\usepackage[colorlinks=true,linkcolor=black,urlcolor=blue,bookmarksopen=true]{hyperref}
\usepackage{bookmark}
\usepackage{fancyhdr}
\usepackage[spanish]{babel}
\usepackage[utf8]{inputenc}
\usepackage[T1]{fontenc}
\usepackage{graphicx}
\usepackage{float}

\pagestyle{fancy} % Encabezado y pie de página
\fancyhf{}
\fancyhead[L]{Apuntes Economía de las Org. - BG}
\fancyhead[R]{2C 2021}
\renewcommand{\headrulewidth}{0.4pt}
\fancyfoot[C]{\thepage}
\renewcommand{\footrulewidth}{0.4pt}

\begin{document}
\begin{titlepage} % Carátula
	\hfill\includegraphics[width=6cm]{logofiuba.jpg}
    \centering
    \vfill
    \Huge \textbf{Apuntes}
    \vskip2cm
    \Large [71.46/71.55/91.41] \\ Economía de las Organizaciones\\
    2C 2021 
    \vfill
    \begin{tabular}{ | l | } % Datos del alumno
      \hline
      Grassano, Bruno \\ \hline
      bgrassano@fi.uba.ar \\ \hline
  	\end{tabular}
    \vfill
    \vfill
\end{titlepage}

\tableofcontents % Índice general

\newpage

\section*{Introducción}\label{sec:intro}
El presente archivo contiene los apuntes que fueron tomados a lo largo de la cursada de la materia Economía de las Organizaciones (91.41). 

La cursada consiste de un parcial donde se evalúan (generalmente) los temas que se vieron hasta costos (practico y teórico). En formato virtual se podía aprobar por tema. Lo recomendable es ir llevando la materia medianamente al día, ya que si se deja medio atrás probablemente no se llegue con todos los temas. Después del parcial dan los demás temas de la materia (ver índice) y se evalúa un mini trabajo practico al final sobre evaluación de proyectos de inversión (mas que nada esta para tener practica del tema para el final).

La materia a mi personalmente me gusto mucho, por ahí a alguien que no le interesen los temas o ya los conozca no.

\newpage
\section*{Clase 1}
\section{Contabilidad}

\begin{center}
$Utilidad = Ventas - Costos$
\end{center}

\begin{itemize}
\item Persona física responde con sus bienes
\end{itemize}

\begin{center}
\begin{math}
Bienes + Derechos = Patrimonio + Obligaciones
\end{math}
\end{center}

\begin{itemize}
\item Derechos es lo que nos deben, y las obligaciones son deudas con terceros.
\item Persona jurídica es una empresa, organización con fines de lucro. Es diferente si es S.R.L. o S.A. Sigue la ecuación también.
\end{itemize}
\begin{itemize}
\item Ordenamos el activo según la liquidez, es decir la rapidez con la que se hace dinero, y los pasivos por la exigibilidad (corto plazo primero)
\item Se tiene que cumplir la ecuación $A = P + PN$, es valida al momento inicial (cero) de cada ejercicio contable. (no lo es en todo momento)
\item Lo mas liquido es la disponibilidad, después lo que hemos prestado (si me lo devuelven enseguida)
\end{itemize}

\begin{figure}[!htb]
    \centering
    \includegraphics[width=0.4\textwidth]{img/BalanceActivoPasivoCorrienteYNo.PNG}
\end{figure}

\begin{itemize}
\item Si uno opera bien, el balance final luego de un periodo crece.
\end{itemize}

\begin{figure}[!htb]
    \centering
    \includegraphics[width=0.7\textwidth]{img/CicloOperativo.PNG}
\end{figure}
 
\begin{itemize}
\item Los proveedores proveen lo que necesito
\end{itemize}






\subsection*{En las empresas}
Una empresa es una organización que realiza alguna actividad con fines económicos o comerciales para ser sustentable en el tiempo. Interactúa con un montón de actores (proveedores, estado, mercados) de diferentes formas.

A la contabilidad le interesa ver los bienes y derechos que tiene la empresa, y las deudas y obligaciones, a quien se lo debe (terceros, o los propios socios). Estos de forma contable, se ve como activo, pasivo, y patrimonio neto.

\begin{itemize}
\item El activo son todos los bienes (materiales e inmateriales) y derechos que una empresa posee.
\item El pasivo son todas las obligaciones con terceros
\item El patrimonio neto son todas las obligaciones que una empresa posee con sus accionistas. Es el valor total de una empresa una vez descontadas las deudas
\end{itemize}

La ecuación estática de la contabilidad quiere decir que todo lo que la empresa tiene, se lo debe a alguien.

Las cuentas mas comunes del activo son:
\begin{itemize}
\item Disponibilidades (dinero)
\item Créditos por ventas
\item Bienes de cambio (mercaderías)
\item Bienes de uso (activos fijos)
\item Cargos diferidos (que se paga por adelantado, ejemplo un alquiler pagado por adelantado)
\item Bienes inmateriales/intangibles
\item Inversiones
\item Moneda Extranjera
\end{itemize}

Las cuentas mas comunes del pasivo son:
\begin{itemize}
\item Proveedores
\item Impuestos a pagar
\item Previsiones/provisiones
\item Deudas bancarias
\end{itemize}

Las cuentas del patrimonio neto son las siguientes, cada una tiene ciertas restricciones (se puede sacar la riqueza de algunas cuentas mas fácilmente que otras
\begin{itemize}
\item Capital, aportes de los accionistas
\item Reservas, las pone la propia empresa (ya sea por cuenta propia o por obligación del estado)
\item Utilidad del ejercicio
\item Utilidades de ejercicios anteriores
\end{itemize}

Dividimos activo y pasivo en corriente y no corriente.
\begin{itemize}
\item Una activo corriente, es aquel que se puede convertir en dinero liquido durante el ejercicio contable (1 año generalmente). Por naturaleza son las disponibilidades (caja), también las cuentas corrientes en el banco (saco la plata y listo). Los stocks son activo corriente, los podemos vender durante el ejercicio.
\item No corriente son las maquinarias, no vendería algo que necesito para producir.
\item Pasivo corriente, son aquellas obligaciones que vencen dentro del ejercicio contable (1 año). El no corriente vencen después.
\end{itemize}


\textbf{Ecuación dinámica de la contabilidad (fundamental)}

\medskip

Activo + Egresos (Perdidas o R- (resultado negativo)) = Pasivo + Patrimonio neto + Ingresos (Ganancias o R+)

\medskip

La cuenta de resultados permite calcular y explicar los beneficios o perdidas de una empresa. Estos surgen del intercambio de valores no equivalentes. Permite calcular la utilidad de la empresa a los fines fiscales y de gestión. (pagar impuestos y tomar decisiones)

Las cuentas de egreso o resultado negativo son los pagos de sueldos, impuestos, amortizaciones, gastos generales, alquileres, costos de ventas.
Las cuentas de ingreso o resultado positivo son ventas de bienes de cambio, ventas de bienes de uso, intereses ganados, descuentos obtenidos.

\subsection{Contabilidad}
La contabilidad la técnica de cuantificar, medir y analizar la realidad económica de las organizaciones con el fin de facilitar la dirección.


\begin{figure}[!htb]
    \centering
    \includegraphics[width=0.7\textwidth]{img/ProcesoContable.PNG}
\end{figure}

\begin{figure}[!htb]
    \centering
    \includegraphics[width=0.7\textwidth]{img/LineaTiempoContable.PNG}
\end{figure}

El libro diario y mayor se hacen durante todo el ejercicio contable. El balance y el cuadro de resultados se hace al final.

\newpage

\subsection{Plan de cuentas}
Es una especie de diccionario para usar en los libros contables. Se indica que lenguaje vamos a utiliza, \textit{por ejemplo, disponibilidades es el dinero en efectivo}. 

Tiene el termino, su definición y un código indicando a que cuenta pertenece.
Contribuye al cumplimiento de los objetivos de la contabilidad, brindan información, ayudan al control, y facilitan la imputación de los registros contables. Es flexible para introducir nuevas cuentas.

\begin{figure}[!htb]
    \centering
    \includegraphics[width=0.8\textwidth]{img/PlanDeCuentasEjemplo.PNG}
\end{figure}

El plan de cuentas tiene que ser lo mas granular posible, tratar de identificar lo mas posible una cuenta con su representación. Esto es para no perder información.

\subsection{Partida Doble}
\begin{itemize}
\item Define las reglas sobre como se comportan las cuentas.
\item Tenemos una partida del Debe y otra del Haber.
\item Afecta mínimo a dos cuentas, no podemos tener cero cuentas en alguna de las opciones (Debe/Haber)
\item La sumatoria del Debe, debe de dar lo mismo del Haber
\item Para la registracion los movimientos se utiliza lo que se conoce con la Cuenta 'T'. Compuesta de Titulo de Cuenta,  Debe y Haber.
\item La idea de todo lo que tengo se lo debo a alguien convierte a la convertibilidad en un sistema espejo.
\item Las cuentas de activo crecen por el debe (\textit{Ej. Tengo mas caja}), y decrece por el haber.
\end{itemize}


Algunas características son:
\begin{itemize}
\item Quien recibe es deudor y quien entrega es acreedor.
\item No hay deudor sin acreedor sin deudor
\item Quien recibe debe a quien da
\item Todo valor que 'entre' debe ser equivalente al que 'sale'
\item Todo lo que entra por una cuenta debe salir por la misma
\item Las perdidas se debitan, y las ganancias se acreditan.
\end{itemize}

\begin{figure}[!htb]
    \centering
    \includegraphics[width=0.7\textwidth]{img/DebeHaberConsecuencias.PNG}
\end{figure}

Cuando la cuenta del activo crece, se agrega en el debe, si decrece va en el haber. \textit{Una venta entra por el debe, un gasto se va a ir por haber}
Cuando crece el pasivo patrimonio crece por el haber, me endeude mas, pero también crece por el debe del activo (por el efectivo que me dieron) (cierra la cuenta, esta en ambos lados)

\subsection{Asiento contable}
Es cada una de las anotaciones o registros que se hacen en el libro diario de contabilidad. Modelan cada una de las acciones, \textit{ejemplo el pago de impuestos.}

Tiene:
\begin{itemize}
\item Fecha de la anotación
\item Cuentas que intervienen (con código y denominación)
\item Importes asociados a cada cuenta
\item Detalle de los documentos que avalan la operación.
\end{itemize}


Hay que tratar de ser lo mas granular en el detalle de cuenta.
Primero las que tienen Debe, después las del Haber.
Si es del Haber, tiene una 'a' al comienzo.
Incluir la variación de cada parte \textit{Ejemplo, A+ activo crece, A- activo decrece, P+, P-, RN+ , etc}


\begin{figure}[!htb]
    \centering
    \includegraphics[width=0.8\textwidth]{img/AsientosContablesEjemplos.PNG}
\end{figure}

\subsection{Modelo Hidráulico}
Es un modelo que establece una analogía del modelo contable con el modelo de la ingeniería hidráulica.

Forma de facilitar la contabilidad a quienes no estén formadas con técnica contable

Se compone de tres actores
\begin{itemize}
\item El agua y su flujo (flecha)
\item Los medidores de agua (rectángulos)
\item Los tanques de agua (círculos)
\end{itemize}


Tenemos tanques donde se deposita el agua, y medidores por donde entra o pasa el agua. Las cuentas patrimoniales (A, P y PN) usan la simbología grande, el activo usa el tanque grande, y el pasivo y patrimonio neto usa los medidores grandes. Las cuentas de resultados negativos y positivos usan tanques y medidores chicos (por ser transitorios). 

Los tanques van depositando la riqueza de la empresa, la van guardando y los medidores llevan registro. Cuando ingresa activo, se deposita en los tanques, y se lleva registro de que ingreso.

Si hay que pagar alguna deuda, sale de esos tanques y pasa por los medidores.

\begin{itemize}
\item En los asientos contables, el agua va del haber al debe. 
\item Cuando crece el activo o los egresos, se acumula agua en los tanques. 
\item Si decrece el activo o los egresos, se va agua de los tanques.
\item Cuando crece el pasivo, PN o ingresos entra agua al sistema que pasa por los medidores que registran el ingreso de agua hacia los tanques. 
\item Cuando decrece el pasivo, PN o ingresos, sale agua del sistema que pasan por los medidores que registran el egreso de agua desde los tanques.
\end{itemize}

\begin{figure}[!htb]
    \centering
    \includegraphics[width=0.7\textwidth]{img/ModeloHidraulicoEjemplo.PNG}
\end{figure}

\subsection{Libro diario}
\begin{itemize}
\item Recopilación de los asientos ordenados de forma cronológica
\item Debe de registrarse todo
\item No se pueden eliminar, sacar o tachar hojas.
\end{itemize}

\begin{figure}[!htb]
    \centering
    \includegraphics[width=0.7\textwidth]{img/LibroDiaripEjemplo.PNG}
\end{figure}

\subsection{Libro Mayor}
\begin{itemize}
\item Se hace a la par del libro diario, recoge las operaciones (asientos contables) en función de las cuentas que se han visto afectadas. Es un libro mas de soporte, no es obligatorio.
\item Luego de realizar el asiento en el libro diario, se pasa el mismo a la ficha individual de cada cuenta.
\item Permite construir el balance
\item En Imputación se pone la cuenta de contrapartida. (de asiento contraria) Si son varias se pone VARIOS.
\item Las cuentas de Activo tienen saldo deudor o cero. No puedo retirar mas de lo que debo.
\item Las cuentas de Pasivo y Patrimonio Neto tienen saldo acreedor o cero. No puedo pagar mas de lo que debo.
\item Todas las cuentas de Resultados (Ingreso y Egreso) deben terminar con saldo cero, luego de los ejercicios de cierre.
\item El primer movimiento es Saldo Inicial. Va el valor del ultimo balance o cero si la empresa es nueva.
\item La fecha inicial es la del comienzo del nuevo ejercicio.
\end{itemize}

\begin{figure}[!htb]
    \centering
    \includegraphics[width=0.7\textwidth]{img/LibroMayorEjemplo.PNG}
\end{figure}

\begin{figure}[!htb]
    \centering
    \includegraphics[width=0.7\textwidth]{img/RelacionesDeLosLibrosContables.PNG}
\end{figure}

\newpage

\subsection{Balance}
Permite conocer la situación financiera y económica de una empresa en un momento del tiempo. Tiene Activo, Pasivo, y Patrimonio Neto. Se construye con los saldos de las ficha del Libro Mayor. Tiene el resultado de las cuentas patrimoniales.


Asientos de cierre, para ver si tuvimos utilidad o perdida. Esta para hacer que se cumpla la ecuación fundamental de la contabilidad (estática). Aparece la cuenta Ganancias y Perdidas que es transitoria (se comporta como RN o RP) Se realizan la 'sumatoria' de ingresos y egresos.
Cancelo las cuentas de resultados negativos contra ganancias y perdidas (RN), se imputa por el lado del Haber (a x cuenta) y Ganancias y perdidas en Debe.
(Cancelamos todas las cuentas de egresos (Resultado Negativo) Todo lo que esta en Haber, lo llevamos a la cuenta de Ganancias y Perdidas en Debe. Por lo tanto, todas las cuentas que llevamos ahí quedan en 0)
Todas las cuentas de Ingresos (Resultado Positivo) Similar, pero tomamos todo lo del Debe, y las llevamos a Ganancias y Perdidas en Haber poniendo en 0 las cuentas movidas. (RP)
Los resultados posibles son:
\begin{itemize}
\item Si es cero, no hubo variación. No se realiza asiento contable.
\item Si el resultado es positivo (saldo acreedor), el resultado le pertenece a los dueños de la empresa (se agrega a patrimonio neto). Se le puede hacer el calculo del impuesto a ganancias y hacer el movimiento. Se lleva a Utilidades del Ejercicio (a Haber)
\item Si el resultado es negativo, se resta al patrimonio neto. Se perdió den el ejercicio. Lo llevamos a la cuenta Perdida del Ejercicio (en Debe)
\item Ganancias y Perdidas después de haber hecho esto queda en 0.
\end{itemize}

\begin{figure}[!htb]
    \centering
    \includegraphics[width=0.7\textwidth]{img/BalancesEnElTiempo.PNG}
\end{figure}


\subsection{Cuadro de resultados}
\begin{itemize}
\item Es un documento que muestra ordenada y detalladamente la forma de como se obtuvo el resultado del ejercicio durante un periodo determinado. Llegamos a la utilidad antes y después del impuesto a las ganancias. Por lo que nos permite evaluar la gestión de la empresa en lo referente a ventas, los costos de ventas, los gastos de estructuras. Esto se construye antes del cierre,
\item Tratar de discernir situaciones excepcionales (\textit{La venta de las joyas de la abuela})
\item A.R.E.A: ajuste de resultado de ejercicios anteriores
\item Las utilidades del cuadro de resultados deben coincidir con valores de los asientos de cierre del libro diario. La única diferencia puede estar con el tratamiento del impuesto a las ganancias.
\end{itemize}

\begin{figure}[!htb]
    \centering
    \includegraphics[width=0.5\textwidth]{img/CuadroDeResultados.PNG}
\end{figure}

\section*{Clase 2}

El contrato de un alquiler se considera como 'cargos diferido' en el activo no corriente, y cada vez que se usa cada mes va a gastos, y se agrega una 'amortización' llamada 'cargo diferido devengado'.


\section*{Ejercicios}
\begin{enumerate}
\item Identificar todas las cuentas intervinientes. (algunas dependen de los montos, por lo que las identificamos después)
\item Determinar el tipo de cuenta y variación (Activo, pasivo, PN, variación positiva o negativa). Indicarlo!
\item Armar el asiento/partida doble teniendo en cuenta 'las reglas de aumentos y disminuciones'.
\item Verificar: Tener al menos una cuenta en el debe y una en el haber y que la sumatoria de ambos sea igual!
\end{enumerate}

\begin{itemize}
\item Para las cuentas del libro mayor, si es una cuenta nueva ponemos en 0 el saldo.
\item Si la contrapartida tiene varias cuentas en el asiento, poner a varios (en el libro mayor). 
\item Si es una sola cuenta, va la cuenta contra la que jugamos (Debe - Haber) en el asiento.
\item Detallar cada parte. \textit{Ej. En vez de mercadería, poner mercaderías PCs}.
\item Servicio de luz, cuenta de resultado negativo, algo que consumí y ya no tengo.
\item La venta de bienes de cambio (mercaderías) involucra 2 asientos. Cobro y venta (en el momento de la venta), y Costo de Venta y Baja de mercaderías (depende del inventario).
\item Ventas es una cuenta de resultado positivo.
\item Costo de venta es una cuenta de resultado negativo. (por el valor al que compramos la mercadería, surge del asiento realizado de la compra)
\item Para el cierre del ejercicio, se hacen 3 asientos. Cancelamos las cuentas de resultado positivo/ingreso contra la cuenta de ganancias y perdidas. Se cancelan las cuentas de resultado negativo/egreso contra la cuenta de ganancias y perdidas. Si TP $>$ RN va a utilidades del ejercicio (PN+), si no a perdida del ejercicio (RegPN+)
\item La cuenta ganancias y perdidas es la única que puede tener saldo positivo o negativo (deudor o acreedor)
\item Una cuenta se cancela imputándola por la contrapartida (debe por el haber, haber por el debe)
\item Para el balance es con el saldo final de las cuentas del libro mayor.
\item El cuadro de resultados se construye a partir de los saldos antes del cierre de las cuentas transitorias (resultado positivo/negativo)
\end{itemize}


\subsection{Cuentas del Patrimonio neto}
\subsection*{Capital}
\begin{itemize}
\item Es el saldo que representa el capital suscrito por los dueños de la empresa. Es lo que los accionistas comprometieron a integrar.
\item Los aumentos de capital se deciden por asamblea de accionistas
\item Las reducciones de capital deben ser resueltas por una asamblea extraordinaria.
\item Las empresas pueden decidir aumentar el capital mediante la emisión de acciones. Puede darse para el ingreso de nuevos accionistas, o para los accionistas existentes. su objetivo es conseguir una rentabilidad mayor.
\item Hay ratios para determinar cuando conviene tomar estas acciones. (Se ve en finanzas)
\end{itemize}


\subsection*{Utilidades del ejercicio}

\begin{itemize}
\item Es el resultado positivo del ultimo ejercicio cerrado, pendiente de aplicación. La distribución es decidida por los socios o accionistas. Generalmente se hace luego de conocer el balance.
\item El pago de las utilidades a los accionistas o socios depende de las disponibilidades de activos líquidos. Si no se poseen, se puede construir una deuda con los socios/accionistas.
\item Se puede usar para cancelar perdidas de ejercicios anteriores también.
\end{itemize}


\subsection*{Utilidades de ejercicios anteriores}
Son las utilidades o beneficios no repartidos ni aplicados específicamente a ninguna otra cuenta tras la aprobación de las cuentas anuales y de distribución de resultados.

\subsection*{Reservas de utilidades}
\begin{itemize}
\item Representan una parte de las utilidades no distribuidas y destinadas a dar cumplimiento de alguna obligación.
\item Hay varios tipos (legales, estatutaria, facultativa)
\end{itemize}


\subsection*{Pago de dividendos}
\begin{itemize}
\item Forma en que se reduce el patrimonio neto.
\item Puede decidirse el reparto de utilidades.
\item Se reduce la cuenta de utilidad del ejercicio (o anteriores), y aparece una cuenta de pasivo de dividendos (deuda contraída) a pagar que se cubre con disponibilidades (pago de la deuda).
\end{itemize}


\subsection*{Perdidas del ejercicio}
\begin{itemize}
\item Es el resultado negativo del ultimo ejercicio cerrado pendiente de aplicación.
\item Para hacer frente a las perdidas, los accionistas/socios pueden reducir el capital, reducir reservas, aportar activos, saldarlas con utilidades de ejercicios anteriores.
\end{itemize}
 

\subsection{Mercaderías}
\subsection*{Venta de bienes de cambio}
\begin{itemize}
\item Los asientos de cobro y venta se realiza en el momento en que se efectúa la venta.
\item Un asiento de costo de ventas y baja de mercaderías depende del sistema de inventario. (RN+)(Debe)
\begin{itemize}
\item Si es permanente se realiza en el mismo momento que cobro y venta. Si son cosas sencillas de reventa este conviene mas. Permite tener una sincronización entre el inventario real y el contable.
\item Si es esporádico se realiza una cantidad fija de veces en intervalos. (por bimestre/trimestre/...) Si es un sistema complejo conviene mas este sistema.
\end{itemize}
\end{itemize}

\subsection*{I.V.A.}
\begin{itemize}
\item Es el impuesto al valor agregado, es indirecto (el fisco - AFIP - no lo percibe directamente del tributario) sobre el consumo. Es financiado por el consumidor.
\item \textbf{Débito Fiscal}: la empresa se lo cobra al \textbf{consumidor}. Es considerado una cuenta de \textbf{Pasivo}. Es deuda que la empresa tiene con el fisco - AFIP.
\item \textbf{Crédito Fiscal}: la empresa paga a otras \textbf{empresas} como consumidor de bienes o servicios (comprar maquinas o mercadería). Se considera un \textbf{Activo}. Son las deducciones que la empresa realiza en el pago del impuesto con el fisco - AFIP. (Pagamos mas de lo que deberíamos en estos productos)
\item La liquidación es de manera mensual.
\item Se grava el facturado, importe neto.
\item La alícuota depende de la actividad. La mas frecuente actualmente es 21\%.
\item Dependiendo de la diferencia del débito y crédito se determina cuanto se paga a la AFIP. (lo que cobre menos lo que pague)
\end{itemize}



\subsection*{Devolución de mercaderías}
\begin{itemize}
\item Las empresas pueden recibir devoluciones por productos defectuosos o equivocaciones. (Los defectuosos no se ven en el curso)
\item El tratamiento contable depende del motivo, del tipo de empresa, y del momento de venta y devolución.
\item Se hace un contra asiento de la venta si el pedido no correspondía con lo pedido originalmente. (todas las cuentas se imputan al revés)
\item Si el inventario es permanente, el costo de venta o C.M.V. se calcula cada vez que se realiza el asiento.
\item Si es esporádico, no corresponde realizar asiento ya que no se realizo el asiento de costo de ventas todavía.
\end{itemize}



\subsection*{A.R.E.A.}
\begin{itemize}
\item Cuenta transitoria de ingreso o egreso en la cual se imputan los ajustes de resultados ya calculados en ejercicios anteriores por no corresponder la imputación.
\item Ajuste de resultados de ejercicios anteriores
\item Se imputan los ajustes de resultados ya calculados en ejercicios anteriores. Se utiliza para castigas la utilidad del ejercicio, ya que la venta no correspondía.
\item Tiene las ventas menos el costo de la mercadería vendida(Costo Mercaderia Vendida)
\item AREA = Ventas (RP+) - C.M.V. (RN-) (devolución en diferente periodo)
\item Se castiga la utilidad del nuevo ejercicio con la diferencia.
\end{itemize}

\subsection*{Previsiones y Provisiones}
Funcionan imputando un\textbf{ falso gasto} de la obligación o eventual contra una \textbf{falsa deuda}. Cuando el gasto se vuelve exigible, se cancela la falsa deuda contra la cuenta de pago. Si no alcanza, el remanente se trata como cuenta de egreso. (Reservar riqueza para cuando pase)

Esto permite mostrar previsibilidad y poder analizar cuanto se necesito realmente.

También evita tener que estar haciendo ajustes por la cuenta A.R.E.A

\subsubsection*{Previsiones}

\begin{itemize}
\item Son las obligaciones eventuales y estimativas que van a incidir en la cuenta de resultados del ejercicio. No son exigibles al momento que se constituyen
\item Se calcula en base a un porcentaje que dice la probabilidad de ocurrencia.
\item No sabemos si van a ocurrir, y no sabemos el importe.
\item \textit{Ej. Despidos, siniestros (inundaciones/incendios), deudores incobrables}
\end{itemize}

\subsubsection*{Provisiones}
\begin{itemize}
\item Obligaciones ciertas que inciden en la cuenta de resultados del ejercicio. No son exigibles.
\item Sabemos cuando ocurren, y tenemos idea de cuanto.
\item Se imputan como si fueran un falso gasto de la obligación o eventual contra una falsa deuda. 
\item Al momento que se vuelve exigible se cancela la falsa deuda contra la cuenta de pago.
\item Si no alcanza el restante se trata como cuenta de egreso.
\item \textit{Ej. Servicios, impuestos, reparaciones}
\end{itemize}


\subsection*{Deudores incobrables}
\begin{itemize}
\item Deudores por ventas en activo
\item Los deudores morosos se consideran como activo.
\item Deudores en litigio judicial en activo
\item Deudores incobrables van a resultado negativo.
\end{itemize}



\subsection*{Activos fijo}
\begin{itemize}
\item Aquellos que no varían durante la explotación de la empresa (maquinaria, útiles, ...)
\item Obligatoriamente involucra el alta del activo(A+), el pago del fijo (A- o P+), IVA crédito fiscal (A+)
\item Opcionalmente están los pagos de impuestos o intereses (RN) y descuentos obtenidos (RP)
\end{itemize}



\subsubsection*{Amortización}
\begin{itemize}
\item Es la representación contable de la perdida de valor o depreciación de caracter irreversible que experimenta un activo fijo.
\item A cada activo fijo se le asigna una vida útil(periodo de amortización) y un valor residual (valor del activo al finalizar el periodo de amortización)
\item Conviene separarlos, no juntarlos todos. Va a permitir sacar cuanto es el valor actual de cada uno.
\item Vemos las lineales, se amortiza lo mismo cada año.
\end{itemize}

\begin{center}
\begin{math}
Amortizacion = \frac{Valor Origen - Valor residual}{Vida util}
\end{math}
\end{center}

\begin{itemize}
\item Intervienen las cuentas de amortización activo fijo, es una de resultado negativo
\item Amortización acumulada activo fijo, es de pasivo, a efecto contable se pone en el activo con signo negativo. Se la denomina regularizadora del activo
\item Se amortiza para disminuir la utilidad del ejercicio evidenciando la perdida de valor del activo fijo. Se genera una \textbf{falsa deuda} que permite renovar el activo fijo, también permite ver el valor actual de forma mas directa.
\end{itemize}


\subsubsection*{Venta}
\begin{itemize}
    \item Se puede vender el activo fijo, intervienen las cuentas de activo fijo, amortización acumulada (se da de baja), medio de pago, y la cuenta de resultado (positivo/negativo). No genera I.V.A.
\end{itemize}


\section*{Clase 3}

Empresa industrial tiene mas activo no corriente que una empresa comercial generalmente.
Distribución de utilidades

\subsection{Moneda extranjera}
\begin{itemize}
    \item La moneda extranjera es la cuenta de activo que refleja la tenencia de moneda-billete de otros países.
    \item Se adquiere como inversión o medio de cambio.
    \item Ingresa al libro diario, mayor, balance convertida.
    \item Aplica a crypto también.
    \item Suele variar la cotización. Cuando es negativa, las perdidas son reconocidas en el momento que \textbf{ocurren}. Cuando es positiva, las ganancias deben ser reconocidas en el momento en que se \textbf{evidencian} (al momento de la venta de la moneda extranjera o después de un tiempo prudencial, considerado de carácter irreversible) (se puede terminar pagando impuesto a las ganancias cuando se pasa a resultado positivo). (Principio de la cautela)
    \item Son asientos de ajuste, interviene Ajuste Negativo o Positivo Moneda Extranjera (RN+ o RP+) y Moneda Extranjera.
\end{itemize}

\subsection{Cargos diferidos}
\begin{itemize}
\item Son los gastos pagados por adelantado. 
\item Es una cuenta de activo.
\item Mes a mes se realiza el devengamiento (consumo), imputación del gasto. Se realiza hasta que tenga saldo nulo.
\item Se ocasionan con el pago de alquileres, intereses, impuestos.
\item Buscan evidenciar la existencia de un activo que incrementa la estructura patrimonial de la empresa.
\end{itemize}

\subsection{Cuentas de producción} %contabilidad segun actividad
Las empresas pueden clasificarse en 3 tipos de actividad:
\begin{itemize}
\item Comerciales: intermediarias entre productor y consumidor, son compra/venta de productos.
\item Industriales: producen bienes mediante la transformación o extracción de materias primas.
\item Servicios: brindan un servicio a la comunidad.
\end{itemize}

\subsection*{Costos}
El Coste es el consumo valorado en dinero de bienes y servicios necesarios para la fabricación o servicio que constituye el objetivo principal de la empresa. Existen los directos (ligados a la producción de un solo producto) y los indirectos (afectan el proceso de uno o mas productos, no se pueden asignar sin usar algún criterio).

Los tipos de costos se clasifican en costo de ventas (materia prima, mano de obra, amortizaciones de activos fijos) y costo de administración, ventas y finanzas (sueldos, amortizaciones de activos fijos no productivos).

\subsection*{Inventario periódico}
Consiste en no registrar todas y cada una de las salidas que se producen en los diferentes almacenes, sino que cuando quiere conocer las unidades consumidas lo que hace es un recuento de las existencias restantes. (Supuesto que todo lo que no esta se consumió)

Consumo = Existencia inicial + Compras o ingresos - Existencia final


\subsection*{Inventario permanente}
Registra todas y cada una de las salidas de los almacenes. (Supuesto que todo lo no consumido esta en el almacén)

Existencia final = Existencia inicial + Compras o ingresos - Consumo


\subsection*{Valoración de inventarios}
La valorización de los inventarios a valores de entrada se hace en base al costo de las unidades que ingresan al almacén. Es necesario un registro de las entradas y de las salidas.

La registracion se hace apoyado en una política de salida (FIFO o PEPS y LIFO UEPS) (no cambia durante el ejercicio contable, solo puede cambiar para el otro ejercicio)


\subsubsection*{Ficha de stock}
Es una ficha en la que se observan las entradas y salidas de bienes con sus valores. Su objetivo es ayudar a determinar el costo de mercadería vendida (C.M.V.)

\begin{figure}[!htb]
    \centering
    \includegraphics[width=0.8\textwidth]{img/ficha.PNG}
\end{figure}


\subsubsection*{Costeo estándar}
Se aplica en empresas industriales, permite obtener una valoración predeterminada de los factores que concurren a la fabricación de un producto y del producto mismo.

\subsection*{Cuentas contables}
\subsubsection*{Cuentas de producción patrimoniales de activo}
Se contabiliza la tenencia de los productos en sus diferentes fases. Desde materia prima hasta producto terminado.
\begin{itemize}
\item Materias primas y suministros
\item Producción en proceso
\item Producción Terminada
\end{itemize}


\subsubsection*{Producción transitoria }
Se comporta como transitorias de egresos (Resultado negativo). Hay varias cuentas.
\begin{itemize}
\item Materia prima en proceso
\item Mano de obra directa en proceso
\item Sueldos de fabrica en proceso
\item Cargas sociales en proceso
\item Amortizaciones en proceso
\item Otros gastos generales de fabricación en proceso
\end{itemize}
Se cancelan contra una cuenta de producción patrimonial del activo, Producción en proceso

\begin{figure}[!htb]
    \centering
    \includegraphics[width=0.8\textwidth]{img/diagramaProduccionAsientos.PNG}
\end{figure}

\subsubsection*{Asientos de cierre}
El momento depende de que tipo de evaluación de cantidades decida la empresa (permanente o esporádico)

\begin{enumerate}
\item Prod. en Proc. Inicial + Prod. en Proc. Durante = Prod. en Proc. Final + \textit{Prod. Terminada Durante}
\item Prod. Terminada Inicial + \textit{Prod. Terminada Durante} = Prod. Terminada Final + \textbf{Costo de Mercadería Vendida}
\end{enumerate}

\begin{itemize}
\item Prod. en Proc. Durante hay que determinarlo en el libro diario.
\end{itemize}


\newpage
\section*{Clase 4}
\section{Finanzas}
\begin{itemize}
\item Esta mas relacionado con la obtención y gestión del dinero, recursos o capital. 
\item Esta para tomar decisiones, tener criterio al momento de tomarlas.
\item Intenta responder preguntas: ¿Es buena la rentabilidad de la empresa? ¿Es conveniente este dinero en caja? ¿Es conveniente producir mas? ¿Es conveniente vender la empresa? ¿Es necesaria la financiación?
\item El objetivo es usar estar herramientas, y apoyarlas con lo visto en contabilidad.
\item La información financiera mira mas la cuenta de disponibilidades, los tipos de activos. Esto es sacado del Libro Diario, Libro Mayor, y Balance
\item Cuando estamos hablando de información financiera estamos hablando del activo liquido. (Contabilidad mira mas la utilidad/perdida del ejercicio en el patrimonio neto)
\item El nivel de disponibilidades es distinto del resultado financiero.
\item Tenemos diferentes herramientas que se pueden dividir en Pasado/Presente, y en Futuro. Ninguna por si sola permite tomar una decisión.
\item La variable fundamental en finanzas es el tiempo.
\item Finanzas, analizamos en un momento y una decisión puede ser correcta. Pasa un tiempo y cambia el contexto, la decisión9 puede no ser correcta ahora.
\item Las herramientas nos sirven para diagnosticas problemas/oportunidades, corregir problemas, y establecer metas/objetivos.
\item A futuro buscamos mejorar el funcionamiento integrado (todas las areas tienen que tener objetivos compatibles)
\end{itemize}


\begin{figure}[!htb]
    \centering
    \includegraphics[width=0.8\textwidth]{img/herramientasTiempoFinanzas.PNG}
\end{figure}

\subsection*{Herramientas a futuro}
\begin{enumerate}
\item El primer paso es establecer metas y objetivos para el próximo ejercicio contable.
\item Se construye la proyección de balance, cuadro de resultados, flujo de fondos
\item Se validan las proyecciones, se ven los baches y excedentes financieros, y se ven distintos ratios de finanzas
\item Se hace una retroalimentación
\end{enumerate}

Es un ida y vuelta.


\begin{itemize}
\item La gerencia general se encarga de los objetivos estratégicos: aumentar rentabilidad, aumentar ventas, disminuir costos, disminución de cargas financieras.
\item Las gerencias/subgerencias se encargan de los objetivos tácticos o operativos: necesidad de personal, necesidad de activos fijos, de proveedores, de bienes de cambio, necesidad financiera.
\end{itemize}

Las metas de ambos se tienen que integrar entre si con el presupuesto y la proyección.

Son diferentes dependiendo de cuando surge la empresa. 
\begin{itemize}
\item Las empresas nuevas tienen menor exactitud en el presupuesto (las existentes tienen mayor exactitud). 
\item La existente por ahí se mueve lento, no es tan ágil para tomar decisiones. \textit{Ej. Puede ser IBM, empresas mas nuevas que se adaptan mas rápido}
\end{itemize}

\subsection*{Presupuesto integrado}
Primero necesito armar el libro diario proyectado (o presupuesto integrado). Si lo tengo a nivel macro, quiere decir que tengo todo agrupado. Como máximo me da un flujo de fondos proyectado. (a nivel macro)
\begin{itemize}
\item Cuadro de resultados proyectado (presupuesto económico)
\item Balance proyectado, necesito tener la proyección día a día.
\item Flujo de fondos proyectado (presupuesto financiero)
\end{itemize}



\subsection{Baches financieros}
\begin{itemize}
\item Son los faltantes de dinero.
\item Nos permite detectar si lo que hacemos es viable o no.
\item Para empresas chicas generalmente resulta mejor realizar el presupuesto financiero o flujo de fondos.
\item Para empresas grandes un bache puede no afectarles, les va a importar la rentabilidad a largo plazo. Sabe que puede pagar los baches en el medio.
\item La visión macro mira periodos grandes (años), la micro mira periodos mas chicos (semanales, mensual)
\end{itemize}

\begin{figure}[!htb]
    \centering
    \includegraphics[width=0.8\textwidth]{img/bacheFinanciero.PNG}
\end{figure}

\subsection{Flujo de fondos}
\begin{itemize}
\item Es la proyección/estimación de todos los ingresos y egresos de dinero de la caja en el tiempo.
\item Permite detectar los excedentes y faltantes de dinero.
\item Esto logra optimizar la inversión de los excedentes, conocer la necesidad de financiación (prestamos), optimizar otras variables relacionadas para evitar la necesidad de financiación(plazo de  cobreo, stock, etc)
\item Para armarlo:
\begin{enumerate}
    \item Lo primero que se hace es establecer un tiempo para la proyección, normalmente un año o ejercicio contable. Esto se subdivide en plazos adecuados para que no escondan baches o excedentes.
    \item Se incluyen todos los movimientos de caja, indicando los \textbf{egresos e ingresos}. Se ubican en la linea de tiempo según corresponda.
    \item Se incluyen las hipótesis consideradas (\textit{Ej. inflación sobre ingresos y egresos})
\end{enumerate}
\end{itemize}

\subsubsection*{Formula de flujo de caja}
Caja = Proveedores + Deudas Bancarias + Capital + Utilidades + Reservas - Créditos - Stock - Activo Fijo

\begin{itemize}
\item Se conocen los valores proyectados de todas las cuentas menos Caja.
\item Se despeja de la ecuación fundamental de la contabilidad la cuenta Caja/Disponibilidades. Se arranca de A = P + PN (Es pasar todo lo de activo para el otro lado menos caja)
\end{itemize}


\subsubsection*{Otra forma}
\begin{itemize}
    \item El bache financiero se puede ver como Ingresos menos Egresos también.
    \item Si se tienen los movimientos sumariados o macro no puedo obtener el flujo de fondos con mayor detalle.
    \item Para determinarlo:
    \begin{enumerate}
        \item Buscamos movimientos contables que involucren movimiento de efectivo (caja)
        \item Determinar si son ingresos o egresos
        \item Hacer ingresos - egresos
    \end{enumerate}
\end{itemize}


\newpage
\subsection{Excedentes financieros}

\begin{figure}[!htb]
    \centering
    \includegraphics[width=0.8\textwidth]{img/excedentesFinancieros.PNG}
\end{figure}

No es razonable tener mucho excedente de caja, no es productiva. Lo mismo para los otros valores. 
Tener demasiada caja no es sano, se lo puede aprovechar mejor (para vida persona y empresas).
Lo importante es que lo decidamos nosotros, no la realidad. Tomar decisiones para la empresa, que este sana.

\begin{figure}[!htb]
    \centering
    \includegraphics[width=0.8\textwidth]{img/excedentesFinancieros2.PNG}
\end{figure}


El stock queremos que sea lo mas chico posible. No produce.
El Just In Time es tener lo mínimo posible para no incumplir pedidos. - manejar eficientemente los niveles de stock.
Lo mismo para la producción en proceso y materia prima.

Buscamos optimizar el funcionamiento de la empresa.
Minimizar caja, tener lo suficiente y \textit{un poco mas por las dudas}. Lo mismo para stock y los demás (signo menos)
Para los proveedores lo máximo que se pueda, siempre y cuando no cobren interés

\newpage
\begin{figure}[!htb]
    \centering
    \includegraphics[width=0.4\textwidth]{img/excedentesFinancieros3.PNG}
\end{figure}

Cuidado con los ratios, ya que son fotos solamente (nos podemos llevar alguna apreciación errónea, pueden estar distorsionadas). Hay que evaluarlos mas continuamente \textit{semana a semana}. Los valores trabajan mejor si se esta estabilizado. (sobretodo para caja)
Cuidado también con esto, las empresas pueden \textit{maquillarlos un poco}.


\subsubsection*{¿Donde aplicar los excedentes?}
\begin{itemize}
\item Comprar mercadería o insumos, prevé inflación y reduce el costo de mercadería vendida.
\item Comprar o modernizar activos
\item Cancelar pasivos
\item Comprar acciones, bonos, o moneda extranjera
\item Inversiones a largo plazo
\item Recompra de acciones de la empresa
\end{itemize}


\subsection{Ratios de liquidez}
% Imagen con otros indices, ratios de liquidez
% disponibilidades se entiede todo lo que se puede convertir en sentido inmediato a efectivo - es en sentido amplio -
% buscan que entendamos el concepto, cuando aplicarlas
\begin{figure}[!htb]
    \centering
    \includegraphics[width=0.8\textwidth]{img/ratiosLiquidez.PNG}
\end{figure}

\newpage
\subsection{Ratios de rendimiento} % indices importantes

\begin{figure}[!htb]
    \centering
    \includegraphics[width=0.8\textwidth]{img/ratiosRendimiento.PNG}
\end{figure}

\subsection{Ratios de endeudamiento}
Buscamos una relación sana entre el pasivo y el patrimonio neto (50-50)

\begin{figure}[!htb]
    \centering
    \includegraphics[width=0.8\textwidth]{img/ratiosEndeudamiento.PNG}
\end{figure}

\subsection{Técnicas de valuación de empresas}
Hay que ver la parte contable de la empresa y como le esta yendo. Depende de distintos factores.
\begin{itemize}
\item El tamaño
\item El tipo de empresa
\item El pasado y presente
\item Contexto y expectativas
\item Si cotiza en bolsa de valores
\end{itemize}

\begin{figure}[!htb]
    \centering
    \includegraphics[width=0.8\textwidth]{img/valuacionEmpresas.PNG}
\end{figure}

\begin{figure}[!htb]
    \centering
    \includegraphics[width=0.8\textwidth]{img/valuacionEmpresas2.PNG}
\end{figure}

\newpage
\section*{Clase 5}

\subsection{Herramientas}
Ver la empresa con las herramientas como un todo. Analizar desde varios puntos.

\subsubsection*{Origen y aplicación de fondos}
Es una herramienta que nos permite comparar los dos últimos balances (la diferencia). Aplica a pasado y presente debido a que estamos tomando los balances.


A efectos prácticos se hace año a año. Seria extraño analizar con balances de hace mucho con el de ahora solamente. (a menos que sea mas macro, quizás es mejor comparar con mas)
\begin{itemize}
    \item Origen, de donde salieron los fondos.
    \item Aplicación, a donde fueron los fondos.
\end{itemize}

\textit{Ej. Compramos maquinaria con efectivo. El origen es la caja, la aplicación es la maquinaria.}

\begin{figure}[!htb]
    \centering
    \includegraphics[width=0.9\textwidth]{img/armadoOrigenAplicacion.PNG}
\end{figure}

Sirve para ver que es lo que se hizo, y como se hizo. Podemos inferir con una precisión razonable (tener en cuenta que es una foto, en el medio no sabemos que paso)
Esto nos permite tomar decisiones y poder orientar la estrategia de la empresa.


\begin{figure}[!htb]
    \centering
    \includegraphics[width=0.9\textwidth]{img/variacionOrigenAplicacion.PNG}
\end{figure}

Pensar como si todo el delta se hubiera destinado a hacer algo. (En la realidad son muchos registros) Estamos viéndolo como un neto. Puede pasar que de 0, en este caso no es origen ni aplicación, es una cuenta que se mantuvo al margen. (pero pueden haber habido movimientos durante el medio)


\begin{figure}[!htb]
    \centering
    \includegraphics[width=0.9\textwidth]{img/ejemploOrigenAplicacion.PNG}
\end{figure}

Una vez averiguados los valores en el armado (la sumatoria de los orígenes tiene que dar lo mismo que la de las aplicaciones), lo ordenamos por monto (origen en un lado, aplicaciones en otro) y armamos el gráfico que se ve a continuación (color para identificar):

\begin{figure}[!htb]
    \centering
    \includegraphics[width=0.9\textwidth]{img/graficoOrigenAplicacion.PNG}
\end{figure}

\begin{itemize}
    \item La relación no es lineal, no estamos viendo un montón de decisiones en el medio. No sabemos en que orden sucedieron las cosas, y cual fue el motivo real.
    \item Esto es para ver de forma clara y poder tomar acciones y decisiones de mejor forma. Podemos ver un pantallazo, si estuvo bien o no. ¿Nuestros créditos por ventas tienen interés? ¿Por que? ¿Se tomo deuda a corto plazo para cubrir deuda a largo plazo? La idea es hacerse preguntas. Estaría mal decir se tomaron cargos diferidos para hacer créditos por ventas. 
    \item Un balance real puede tener muchas cuentas, en el gráfico solamente se incluyen las que tuvieron variaciones mas significativas,
    \item Que este el gráfico no quiere decir que hay que estar explicando cada movimiento.
    \item El gráfico en un contexto puede tener sentido, en otro contexto no.
\end{itemize}

\medskip
\textbf{Optimizaciones posibles}

\medskip
Son lineamientos generales
\begin{itemize}
    \item Según los principios de conformidad financiera, el activo corriente se financia con pasivo corriente, y el activo no corriente con pasivo no corriente y patrimonio neto.
    \item Hay que evitar la toma o aumento de deudas y/o stock en momentos de recesión.
    \item Evitar otorgar créditos si se prevé quiebras o incobrables.
    \item Evitar vender moneda extranjera en procesos de devaluación.
\end{itemize}

\medskip
\textbf{Limites y criticas}

\medskip
\begin{itemize}
    \item Muestra la evolución entre 2 momentos, se puede aplicar a periodos mas cortos (trimestral), o apoyarse en otros documentos
    \item Requiere correcciones que eliminen efectos de la contabilidad (moneda extranjera, inversiones)
    \item Requiere conocer la distribución de utilidades.
\end{itemize}

\subsubsection*{Capital de trabajo}
Es la parte del activo corriente que resulta de la diferencia con el pasivo corriente. Es la parte del activo corriente financiada con recursos de carácter no corriente (permanente) (Llamado Fondo de maniobra también)

\medskip
Con enfoque de activo circulante:
\begin{center}
    KT = Activo Corriente - Pasivo Corriente
\end{center}

\medskip
Con enfoque de capitales permanente:
\begin{center}
    KT = Pasivo No Corriente + Patrimonio Neto - Activo No Corriente
\end{center}

\medskip
Queremos que sea lo mas chico posible. (positivo tendiendo a cero) (no hay un valor exacto)
\begin{itemize}
    \item Si es negativo implica que el pasivo corriente (exigible) es mayor que el activo corriente (liquido o convertible a liquido). Esto lleva a un bache financiero.
    \item Si es positivo cercano a cero puede ser riesgoso debido a problemas en las ventas, créditos por incobrables, etc Puede llevar a bache financiero.
    \item Si es positivo y grande se tiene un excedente financiero
\end{itemize}

\medskip
\textbf{Optimizaciones posibles}

\medskip
Se puede reducir los días de maduración.
\begin{center}
Días de maduración = Días de Caja + Días de compras + Días de stock + Días de Crédito - Días de proveedores
\end{center}

\medskip
El limite de la reducción esta dado por características de producto-mercado y el costo del pasivo corriente versus el pasivo no corriente y capital de trabajo.

Se busca llegar al ideal, un KT tendiente a cero.

\medskip
\textbf{Estructura Deudora y Acreedora}
\medskip
\begin{itemize}
    \item En una estructura deudora el pasivo corriente es mucho mayor a los créditos, por lo que se tiene un plazo de pago a proveedores mayor que el cobro a clientes. La ventaja es que permite que se aceleren procesos expansivos, pero el problema es que en momentos de recesión aumenta la posibilidad de baches financieros. Compramos a largo plazo y vendemos a corto plazo. \textit{Ej. Los supermercados, empiezan a generar excedentes financieros y todavia no tienen que pagarle a nadie. Esto permite ampliar las sucursales (por el excedente). En momentos de recesion, la gente deja de consumir y se reducen los ingresos y en algun momento hay que pagar a los proveedores.} 
    \item En una estructura acreedora los créditos son mayores a el pasivo corriente, por lo que el plazo de pago a proveedores es menor que el plazo de cobro a clientes. La ventaja es que en recesión disminuyen los baches financieros, pero que si aumentan las ventas no se tiene capital para pagar en efectivo. \textit{Ej. Autos de lujo, desarrollo de software, hago el gasto pero los ingresos vienen demorados}
\end{itemize}

Lo mejor es tratar de no maximizar ninguna de las dos.

\subsection{Índices \& ratios }


\begin{itemize}
\item La deuda no es mala, permite apalancar. \textit{Ejemplos dados en clase, como afecta al rendimiento operativo y sobre el patrimonio neto (lo que tuvimos que poner) } Si tomo deuda y financia activos productivas es una deuda sana.
\item Los indicadores solamente mencionan un punto. Tenemos que saber que estamos midiendo y tomar una decisión. Hay algunos autores que no toman todas las variables en algunos índices, siempre hay que ir al concepto y ver de que se trata.
\item Los elegimos mirando origen y aplicación de fondos y la estructura de la empresa.
\end{itemize}



\begin{figure}[!htb]
    \centering
    \includegraphics[width=0.9\textwidth]{img/indices1.PNG}
\end{figure}

\begin{itemize}
\item Rentabilidad de los activos, en cierta forma me dice que tan rentable es una maquina. Rentabilidad del patrimonio neto nos importa mas.
\end{itemize}
\begin{figure}[!htb]
    \centering
    \includegraphics[width=0.9\textwidth]{img/indices2.PNG}
\end{figure}

\begin{itemize}
\item Utilidad Operativa me dice como funciona la empresa en cuanto a maquinarias, empleados. Rendimiento sobre el patrimonio neto le va a importar a los accionistas y dueños.
\end{itemize}
\begin{figure}[!htb]
    \centering
    \includegraphics[width=0.9\textwidth]{img/indices3.PNG}
\end{figure}
\begin{itemize}
\item Evolución de ventas no tiene en cuenta la inflación. (Los indices con N y N-1 no lo tienen en cuenta. Si queremos usarlos hay que ajustar los valores.)
\item Rotación de Activos es una foto. El ratio de rotación de activos vamos a querer que sea lo mas alto posible. Generar mas ventas con una empresa mas chica es mejor, optimizar financieramente a una empresa tiene que ser lo mas chica posible.
\end{itemize}

\begin{figure}[!htb]
    \centering
    \includegraphics[width=0.9\textwidth]{img/indices4.PNG}
\end{figure}

\begin{figure}[!htb]
    \centering
    \includegraphics[width=0.9\textwidth]{img/indices5.PNG}
\end{figure}

\begin{figure}[!htb]
    \centering
    \includegraphics[width=0.9\textwidth]{img/indices6.PNG}
\end{figure}

\begin{figure}[!htb]
    \centering
    \includegraphics[width=0.9\textwidth]{img/indices7.PNG}
\end{figure}

\begin{figure}[!htb]
    \centering
    \includegraphics[width=0.9\textwidth]{img/indices8.PNG}
\end{figure}
\begin{itemize}
\item Liquidez: Es en cierta forma la capacidad de afrontar las deudas a corto plazo. Queremos que sea un valor cercano a uno. \textit{Por ejemplo, si da 2, significa que tengo dos veces activo corriente como para cancelar el pasivo corriente}. 
\item La liquidez es independiente de los beneficios. Podemos vender mucho y no generar liquidez.
\item Liquidez inmediata: Incluyo solamente lo que verdaderamente es liquido. (inversiones financieras temporales y disponibilidades) 
\end{itemize}

\begin{figure}[!htb]
    \centering
    \includegraphics[width=0.9\textwidth]{img/indices9.PNG}
\end{figure}

\begin{figure}[!htb]
    \centering
    \includegraphics[width=0.9\textwidth]{img/indices10.PNG}
\end{figure}


\begin{figure}[!htb]
    \centering
    \includegraphics[width=0.9\textwidth]{img/indices11.PNG}
\end{figure}

\begin{figure}[!htb]
    \centering
    \includegraphics[width=0.9\textwidth]{img/indices12.PNG}
\end{figure}



\begin{itemize}
\item Índice de productividad: en una forma me dice cuanta plata genera cada persona. Si es bajo, se tienen que aumentar ventas, o despedir gente, o puede pasar que se deje de operar en el país. Costo medio de personal, me da una especie de sueldo promedio.
\item Cash Flow Explotación: Vemos cuanto genera la empresa sin tener en cuenta los resultados financieros o extraordinarios
\item Índices de endeudamiento: Miden como esta la empresa frente a las deudas, si las puede afrontar. Incluye las de largo plazo
\item Apalancamiento: Es una forma de mostrar como se controlaría una empresa, como se controla en base a activo/pasivo/patrimonio. \textit{Ej. empresa con activo \$10M, pasivo \$9.999.999 y patrimonio neto \$1, en cierta forma con\$1 manejo una empresa de \$10.000.000}
\item Tasa Costo Deuda: Muestra a cuanto estoy endeudado, ayuda a ver cuando conviene tomar deuda.
\item Días de Gestión: Tenemos que ver si la cantidad de días es razonable. Todo excedente es malo (no es que es positivo, si no que es muy alto. Es malo para el negocio). Los días de proveedores buscamos maximizarlos, siempre y cuando no tenga intereses.
\end{itemize}






\subsubsection*{Recomendaciones}
\begin{itemize}
    \item Tener un numero limitado de ratios significativos y bien seleccionados. \textit{Si casi no hay pasivo, no me va a interesar calcular algunos índices}
    \item Mientras mas tengo es mas difícil ver que esta pasando.
    \item No existen óptimos absolutos, dependen de la actividades del negocio. Todo es cuestionable, puede ser que se justifique x cosa.
    \item Nos permiten realizar comparaciones entre empresas/sectores/periodos. 
\end{itemize}




\subsection{Herramientas 2}
\subsubsection*{Dupont}
Herramienta para optimizar el rendimiento sobre activos. (multiplico y dividió por las ventas para separar en dos partes)

\begin{figure}[!htb]
    \centering
    \includegraphics[width=0.9\textwidth]{img/dupont.PNG}
\end{figure}


\begin{itemize}
    \item Analizamos dos ramas, después tenemos que ver que sean compatibles entre si. Hay que verlo de forma integral al diagrama
    \item Puede pasar que la misma compañía tenga la misma rentabilidad sobre activos totales con estrategias diferentes. \textit{Una con alto beneficio y baja rotación con otra con bajo beneficio y alta rotación.}
    \item Los objetivos estratégicos están a la izquierda, y cuando baja a las gerencias incide en alguna de las variables de la derecha.
\item Ayuda a plantear objetivos macro.
\end{itemize}


\subsection{Crecimiento Autosostenido}
\begin{itemize}
    \item Self-Sustainable Growth
\item Es el máximo crecimiento que puede tener una empresa sin deformar el ratio de deuda sobre patrimonio neto.
    \item Esto se puede deformar con empresas que crecen muy rápido, es un problema.
    \item CAS = Rendimiento del PN x \%Ganancias NO distribuidas
    \item Se consideran ganancias no distribuidas a lo destinado a capital, reservas, y utilidad de ejercicios anteriores
    \item Esto es importante para la financiación con 3eros (proveedores y entidades financieras). Si disminuye la relación patrimonio neto/activos, disminuye la probabilidad de conseguir crédito.
\end{itemize}

\newpage
\section*{Clase 6}
\textit{Fue repaso de diferentes temas de finanzas.}
\medskip

\begin{itemize}
\item Devengado es derecho a percibir, es una venta
\item Percibido
\item Pagado (mis deudas) y cobre (mis ventas)
\item Lo gastado es lo que va al cuadro de resultados
\item Hay dos amortizaciones, por tiempo y por uso
\item Capital de trabajo es la parte de activo financiado por el pasivo no corriente y el patrimonio.
\item Lo único mensual es el cierre de la liquidación del I.V.A.
\item El diario es todos los días y mayor debería de tener una actualización a la par.
\item La construcción del balance sale LUEGO del cierre del ejercicio de la ficha del libro mayor.
\item Anticipos de sueldo seria un cargo diferido. (puede pensarse como un crédito que estamos otorgando)
\item Todo en finanzas es posibles X cosa.
\item Días de maduración lo minimizamos, es el tiempo hasta que obtenemos una ganancia en una inversión. Cuanto tiempo esta tardando mi dinero en volverme a mi.
\item Rendimiento sobre activos mira mas la operación sobre la empresa, no tiene en cuenta cargas financieras (intereses). Es lo que gano sobre lo que se invirtió.
\item Calidad de deuda, puedo transformar parte de mi pasivo corriente en pasivo no corriente. Es importante que las deudas traten de ser a largo plazo. (PC/(PC+PNC))
\item Apalancamiento financiero tiene en juego si me endeudo o no y que gastos financieros se tienen en cuenta. Evalúa la relación entre la deuda y los capitales propios y como juega el endeudamiento en función de los gastos financieros. 
\item Se disminuyen los gastos financieros obteniendo pasivos gratuitos (créditos a muy largo plazo) y/o mejorando la calidad de la deuda.
\item Financiar los activos con pasivo corriente y pasivo no corriente respecto del patrimonio neto (endeudarse) si el rendimiento sobre activos esta mejor que la tasa de interés que nos cobran.
\item Al aumentar la deuda aumentamos el rendimiento sobre el patrimonio neto. (Recordar la formula) Esto es siempre y cuando nos presten y no se pase el gasto financiero.
\end{itemize}

\newpage
\section*{Clase 7}
\section{Costos}

\subsubsection*{¿Para que sirven?}
\begin{itemize}
\item De alguna manera miden la eficiencia, que resultado tengo respecto de lo que consumo.
\item El objetivo principal es conocer el costo. Si quiero menor costo debo aumentar la eficiencia.
\end{itemize}


\subsubsection*{¿Con que frecuencia se realiza?}
Es mensual o bimestral el análisis de costos

\subsubsection*{¿Que contabilidad de costos necesitamos?}
\begin{itemize}
\item La que nos brinde un valor de inventarios con la rapidez y desglose requerido.
\item La que tenga un plan de cuentas lo mas amplio posible que indique cuanto, en que, y en donde se gasto.
\item La que brinde una adecuada departamentalizacion.
\end{itemize}


\subsubsection*{¿Que elementos tiene?}
Los elementos del costo están conformados por materia prima, la mano de obra directa, gastos generales de fabricación (todas las erogaciones necesarias para realizar la producción, energía, seguros, amortizaciones, impuestos, que no pueden imputarse de forma directa.), y costos de distribución (comercial, comisiones, logística). Todos estos deben de asignarse al producto terminado.

\subsection{Clasificación de costos}

\begin{figure}[!htb]
    \centering
    \includegraphics[width=0.8\textwidth]{img/TiposDeCostos.PNG}
\end{figure}

Como se acumula
\begin{itemize}
\item Costeo por orden de trabajo, es claro y definido cuanto se pide. \textit{Ej. Quiero 10 mil unidades, quiero un edificio, etc - Están definidos}. Cada pedido se trata como un output especial. Se tiene amplia variedad e productos o servicios. Calculo el costo en función al pedido, no tanto del periodo de tiempo. \textit{Ej. Imprentas, astilleros, fabricantes de muebles, constructores, asesorías, productos especiales}
\item El costeo por proceso es continuo. Es de producción masiva de productos o servicios homogéneos. El costo unitario es el costo total del periodo dividido las unidades producidas. Se puede usar también cuando un conjunto de procesos de fabricación producen un amplio numero de productos homogéneos. Los costos se acumulan para un periodo de tiempo. \textit{Ej. Plantas químicas, procesadores de alimentos, textiles}
\end{itemize}


Según cuando se determina
\begin{itemize}
\item Histórico, obviamente tiene un costo estimado, pero la idea es terminar el trabajo y calcular el costo.
\item Estándar, se tiene una estimación mas científica o técnica del costo. Sabiendo cuanto se gasta para cada uno de materia prima, MOD, y gastos generales.
\item Histórico con cuota \textit{(no se evalúa)}.
\end{itemize}


Según como tratamos a los gastos fijos.
\begin{itemize}
\item Absorción, todos los gastos van al centro productivo. 
\item Directo, no considero los gastos fijo en el producto. Considero solo los variables, materia prima y MOD. Mano de obra indirecta es aquella en la cual si no esta aquella persona el producto se hace igual. \textit{Ej un supervisor, seguridad} pertenecen al gasto general.
\end{itemize}



\subsection*{Practica}
\begin{itemize}
\item Buscamos responder diferentes preguntas. Por ejemplo, ¿Es correcto sumar el costo fijo? ¿Y si lo quiero ver por separado? ¿No compre muy caro? ¿Prepare con las cantidades adecuadas?
\item De forma casera pensamos solo en la materia prima (\textit{Cuando vamos a un asado se divide lo que salio la carne, no se incluye el costo del alquiler de quien puso la casa}). Las empresas empiezan a ver mas variables que solo la materia prima, ven costos de fabricación, costos generales.
\end{itemize}


\subsubsection*{¿Para que sirven los costos?}
\begin{itemize}
\item En el concepto tradicional es para valorizar inventarios y fijar precios de venta (costos + utilidad = precio de venta).
\item En el concepto moderno es para tomar decisiones (¿me conviene seguir fabricando este producto?) y medir eficiencia (precio venta - costos = utilidad).
\item Es complejo calcular los costos, hay todo un tema temporal. ¿Cuando compramos la materia prima? Por ahí se valorizo. Generalmente compramos un poco mas de ciertos materiales (tiramos un poco mas de harina a la mezcla, se cae algo al piso). Todo esto va afectando. ¿Que pasa cuando introduzco un nuevo producto? ¿La puedo fabricar con la estructura actual? Hay un montón de costos asociados al servicio que se brinda.
\item Cuando hablamos de costo en la materia, no hablamos del precio de venta, hablamos de lo que \textit{costo} fabricar el producto (lo que sale para la empresa). 
\item \textit{El precio es lo que se paga, y el valor es lo que se obtiene.}
\end{itemize}

\newpage
\subsubsection*{¿Donde es mas caro/barato?}

\begin{figure}[!htb]
    \centering
    \includegraphics[width=0.8\textwidth]{img/lugaresCostos.PNG}
    \caption{Por ejemplo, en la izquierda una pizza sale 901 y en el de la derecha 900.}
\end{figure}

\subsection*{Elementos del costo}
\begin{itemize}
\item \textbf{Materia prima}: Son los elementos físicos que se consumen y transforman.
\item \textbf{Mano de obra}: representa el valor del trabajo del personal necesario para realizar el producto final.
\item \textbf{Gastos generales de fabricación (GGF)}: son las erogaciones necesarias para realizar la producción - energía, seguros, amortizaciones, impuestos.
\item \textbf{Costos de distribución}: comercial, comisiones, logística
\end{itemize}


\subsection*{Objetivo}
El objetivo es conocer cuanto me cuesta producir y vender un producto o servicio. Tener un mejor escenario para definir el futuro de la empresa.

Si quiero minimizar el costo debo mejorar la eficiencia (\textit{Ej la maquinaria, el tiempo de producción})

\subsection*{Contabilidad de costos}
Es importante tenerla para entender que es lo que esta pasando en la empresa.
El análisis de costos se hace mensual/bimestral.
Se tendrán costos razonables si tenemos una información confiaba brindada por la Contabilidad de costos.

\newpage
\subsection*{Clasificación de costos II}

\begin{figure}[!htb]
    \centering
    \includegraphics[width=0.8\textwidth]{img/TiposDeCostos.PNG}
\end{figure}

Absorción y directo no tiene tanto que ver con los distintos tipos de producto.

\medskip
Costeo estándar es el que más trabajo lleva. ¿Cuanto tiempo lleva amasar una pizza? ¿Y en el horno? Si bien el costeo estándar es una forma que nos dice cuanto cuesta cada parte, con muchos productos y etapas se complican los cálculos.

\begin{figure}[!htb]
    \centering
    \includegraphics[width=0.6\textwidth]{img/DistribucionCostos.PNG}
    \caption{El flujo que se tiene.}
\end{figure}

Directo y por absorción no se superponen. Tiene que ver con como se distribuyen los gastos fijos en cada costeo. 


\begin{figure}[!htb]
    \centering
    \includegraphics[width=0.6\textwidth]{img/DistribucionAbsorcionDirectoCostos.PNG}
\end{figure}


Histórico y estándar no se superpone. El histórico dice algo que ya paso (calculamos en base a algo que ya paso) mientras que el estándar es más mirando a futuro. Calcula anticipadamente cual debería ser el costo de un producto. Ver cuanta materia prima, cuanto tiempo lleva fabricar el producto.

En el histórico es más difícil darnos cuenta si lo estamos produciendo de forma óptima.

\subsubsection*{Absorción y directo}
Las ventas no dependen de como haga el sistema de costeo.
Si depende cuando haga el cuadro de resultado. Depende también del sistema de inventario. La diferencia que surge de la utilidad de explotación en el cuadro de resultados esta en el inventario. (Cuanto vale segun las cuentas que hicimos)

\begin{itemize}
    \item En el absorción cuando no se vende algo, estoy dejándolo en el activo, no se mete en el cuadro de resultados.
    \item En absorción el costo fijo se va diluyendo a medida que aumenta la producción (y sus costos) por lo que el costo unitario disminuye. En el directo se mantiene constante (es proporcional)
    \item La utilidad en el costeo por absorción es función de la cantidad vendida y producida. Esto pasa por como distribuimos los gastos fijos. (si producimos más aumenta la utilidad de explotación, nuevamente la diferencia esta en el inventario.)
    \item Da la sensación de que se distribuye mejor los costos fijos, pero si no se traduce en ventas, estamos aumentando el activo con un inventario que si no vendo se convierte en un problema.
\end{itemize}

\begin{figure}[!htb]
    \centering
    \includegraphics[width=0.8\textwidth]{img/AbsorcionDirecto.PNG}
\end{figure}

\begin{figure}[!htb]
    \centering
    \includegraphics[width=0.8\textwidth]{img/cuadroConDiferenciasCostos.PNG}
\end{figure}

\newpage
\section*{Clase 8}

\subsection{Materia prima}
\begin{itemize}
\item Es variable y proporcional a la producción.
\item El costo de la materia prima aparece cuando se retira del almacén (sino se considera como una inversión en materia prima) (pasa a materia prima en proceso)
\item Nos abastecemos de ella según como negociamos la compra y que rotación tenemos. Esto depende también de como la financio, a corto o largo. Lo ideal es comprar lo óptimo, pero dependiendo del contexto (inflación o trabas de importación) puede ser bueno prevenir.
\end{itemize}



\subsubsection*{Stock del almacén}
\begin{itemize}
\item Periódico, cada periodo se observa que esta registrado y que aparece en el almacén. MPinicial-MPfinal me indica lo que se consume.
\item Permanente, se hace un control de cada movimiento (vale almacén, código, fecha, hora, sector al que va)
\end{itemize}




\subsubsection*{Rendimiento MP}
Siempre hay algún desperdicio debido a la manipulación y aptitud de la mano de obra, y al proceso debido a problemas físicos y químicos.

\subsection{Elementos de mano de obra directa}
\begin{itemize}
\item Directa es cuando la persona tiene que estar si o si para que se haga el producto.
\item Pueden tener remuneración quincenal o mensual. 
\item La mano de obra conlleva cargas sociales asociadas.
\item Se tiene que presupuestar la hora normal y las extra, si son en horas normales o nocturnas.
\end{itemize}



\subsection{Gastos indirectos}
\begin{itemize}
\item Son los mas difíciles de calcular
\item Todo lo que no es MP y MOD es gasto general de fabricación. Engloba muchos conceptos.
\item Hay que analizar como se lleva a los centros productivos.
\item Mano de obra indirecta, no esta vinculado a la producción.
\item Combustibles, lubricantes, su consumo es función de las horas trabajadas y en general es variable.
\item Repuestos y reparaciones
\item Depreciación de bien de uso. Es un gasto directo si lo hago en función de la cantidad de unidades que se hacen, pero en general se considera en forma cronológica. Esto se analiza de forma periódica. Es función del driver cost que mejor represente la perdida de valor del bien de uso.
\item Impuestos, IVA, inmobiliario, patentes, financiero, ganancias, otros gastos (seguros, seguridad, higiene industrial, limpieza)
\end{itemize}


\subsection{Producción Equivalente}
Buscamos calcular la cantidad fabricada ($Q_{fab}$) y valorizarla con la mayor exactitud posible. Se aplica en procesos que se extienden en el tiempo y en ordenes de fabricación donde los stocks en producción en proceso toma valores importantes. 

Tenemos $Q_{fab}$, $Q_{terminada}$, $Q_{vendida}$

\medskip
\begin{math}
CostoUnitario= \frac{costo de la producción fabricada equivalente}{Q_{FabricadaEquivalente}}
\end{math}

\begin{figure}[!htb]
    \centering
    \includegraphics[width=0.8\textwidth]{img/EjemploProduccionEquivalente.jpg}
    \caption{Ejemplo de las cuentas}
\end{figure}

\subsection*{Practica}

\begin{figure}[!htb]
    \centering
    \includegraphics[width=0.8\textwidth]{img/CalculoASTD.jpg}
\end{figure}

\begin{itemize}
\item STD de tiempo, el tiempo a fabricar el producto.
\item STD de jornal seria lo que se cobra por hora.
\item Cuando hacemos los costos STD los hacemos a futuro, pero a medida que se fabrique se van generando variaciones. Si en el medio el sector de compras logro comprar a 3,8 y estimábamos a 4, el STD sigue siendo 4, es una variación. Los datos reales no cambian el costo STD. 
\end{itemize}


\begin{enumerate}
\item \textbf{Distribución de gastos generales de fabricación}: Lo primero que hacemos es una distribución secundaria de los centros de servicios a los centros productivos (Si tenemos uno es trivial, con más hay que aplicarlo según el porcentaje de utilización,\textit{Ej. el centro de servicios 1 asiste en un 60\% al centro productivo 1, y al otro en un 40\%.}
\item Sumamos los costos estándar (fijo y variable) del centro productivo con lo distribuido de servicios (fijo y variable) y obtenemos los gastos generales de fabricación (Fijo más variable)
\item Obtenemos el nivel esperado de actividad del centro productivo (la capacidad normal por la producción de horas hombre que se tiene) y las sumamos.
\item Obtenemos la cuota beta dividiendo los gastos generales de fabricación con el nivel esperado de fabricación.
\end{enumerate}


\begin{figure}[!htb]
    \centering
    \includegraphics[width=0.8\textwidth]{img/ProcesoCalculoASTD.jpg}
    \caption{En este caso es STD, con absorción}
\end{figure}

\begin{itemize}
\item La cuota alfa solo tiene la parte variable (STD, directo) (los del centro productivo más los del centro de servicios). Esta cuota siempre va a ser más chica que la beta.
\item La cuota alfa es lo variable dividido el nivel de actividad, la beta solo cambia que es con todo.
\item Mientras mas grande la cuota beta más 'pesa' en el bolsillo (significa que se tiene bajo nivel de actividad).
\item La cuota beta y alfa  no es por producto, es para todo el centro.
\item Puede pasar que no todos los productos tengan que pasar por todos los centros productivos.
\item Tener en cuenta las unidades al momento de hacer las cuentas.
\item En los gatos de fabricación generales cuando calculamos para cada producto, solo incluimos los centros productivos por los que pasa el producto.
\item Puede pasar que un centro de servicios asista a otro centro de servicios.
\item Se empieza distribuyendo lo que esta 'más lejano'
\end{itemize}





\newpage
\section*{Clase 9}

\section{Análisis marginal}


\begin{itemize}
\item El análisis marginal estudia la generación de utilidades en la empresa en función de los aportes individuales de cada tipo de bien (productos) o servicios producidos.
\item Se concentra en la dinámica más que en la determinación. Como son generadas estas utilidades y que pasaría en diferentes escenarios.
\end{itemize}

\subsection*{¿Que hace el análisis marginal?}
\begin{itemize}
\item Mira la construcción de la utilidad o perdida de la empresa a partir de los diferentes productos.
\item Relaciona los costos con las ventas y su precio. Se basa en el costeo directo.
\item Pone el foco en costos propios (directizables) que varían con el nivel de actividad (vender mas o menos cambia los costos) y existen mientras el producto exista (fijos propios).
\item Pone el foco en el marginal de producir una unidad mas.
\item Es una herramienta de análisis para el pasado o para el futuro (estimado)
\end{itemize}


\subsection*{¿Que nos permite?}
\begin{itemize}
\item Saber que producto es mas rentable
\item Saber cuando discontinuar una sucursal o producto
\item Cuando sustituir un producto
\item Cuando agregar un nuevo producto
\item Cuando conviene comprar un producto para revender
\item Saber donde aplicar los recursos escasos.
\end{itemize}


\subsection*{Definiciones}
Suponemos n-productos (P1, ... Pn). Para cada uno se definen 4 variables. Se consideran independientes entre si.

\begin{itemize}
\item pi: precio de venta (promedio)
\item wi: costo variable unitario (MP y MOD, GGF, GGAVF), varian con el nivel de actividad.
\item Qi: volumen físico de ventas
\item Fi: gasto fijo propio que no varían con el nivel de actividad durante el periodo. (Alquileres, publicidad, GGACF)
\end{itemize}

\medskip
Estas se definen para un periodo dado.

\begin{itemize}
\item Presupuestado (Futuro) \textit{Ej. el próximo año}
\item Histórico (Pasado) \textit{Ej. El año pasado}
\end{itemize}


\subsubsection*{Ventas de un producto}
Vi = pi Qi

\subsubsection*{Ventas para toda la empresa}
V = $\sum_{i=1}^n$ Vi

\subsubsection*{Costo total}
C = F + $\sum_{i=1}^n$ w'i Qi + $\sum_{i=1}^n$ $si$ Vi

\begin{itemize}
\item w'i coeficiente de costo proporcional a los volúmenes de ventas.
\item si coeficiente de costo proporcional a la venta de producto
\end{itemize}


\subsubsection*{Utilidad Operativa}
U = V - C

\subsubsection*{Ecuación Fundamental}
\begin{itemize}
\item U = $\sum_{i=1}^n$ (pi - wi) Qi - F
\item wi = w'i + si pi
\end{itemize}



Solo se puede aplicar si los precios se mantienen constantes.
Se estudia los volúmenes físicos principalmente ya que este varia más.
Los precios, los costos variables unitarios y los gastos fijos tienden a ser constantes.

\subsubsection*{Utilidad Marginal (ei)}
\begin{itemize}
\item ei = pi - wi
\item U = $\sum_{i=1}^n$ ei Qi - F
\end{itemize}


Derivando la utilidad respecto de la cantidad obtenemos que se tiene un incremento de las utilidades cuando se vende una unidad adicional.(Los fijos al ser marginal SON SOLO LOS PROPIOS, no los de estructura)

\subsubsection*{Utilidad marginal (Ei)}
Definición complementaria de la utilidad marginal de un producto.
\begin{itemize}
\item Ei = ei Qi = pi Qi - wi Qi = Vi - Wi
\end{itemize}



Reescribimos
\begin{itemize}
\item U = $\sum_{i=1}^n$ Ei - F
\end{itemize}


\subsubsection*{Tasa de Utilidad Marginal del producto (mi)}

mi = $\frac{ei}{pi}$ %!!!!!!!!!!!!!!!!!!!!!!!!!!!!!!!!!!!!!!!!!!!!!!!!!!!!!!!!!!

\begin{itemize}
\item Nos indica el porcentaje de utilidad marginal respecto al precio de venta.
\item Se considera adimensional.
\end{itemize}



\begin{figure}[!htb]
    \centering
    \includegraphics[width=0.8\textwidth]{img/FormulaEquivalenteTasaUtilidadMarginal.jpg}
\end{figure}


\subsubsection*{Tasa de utilidad marginal de la empresa (m)}

\begin{itemize}
\item m = $\frac{E}{V}$
\item U = E - F
\item U = m V - F
\end{itemize}


Nos da la idea de canto aumenta la utilidad por cada peso adicional vendido, siempre que esa venta sea a precio constante. (Sea por variar el volumen)

Cuando la proporción relativa de los distintos productos en unidades físicas se mantiene constante se dice que las ventas mantienen una mezcla constante. (Es ponderado por el volumen.)

\begin{itemize}
\item $m_{empresa}$ = m1 a1 +... mn an
\item a1 + ... an = 1
\end{itemize}



\subsection*{Ecuación de la empresa - Modelo analítico}
\begin{figure}[!htb]
    \centering
    \includegraphics[width=0.8\textwidth]{img/EcuacionDeLaEmpresa.jpg}
\end{figure}

\begin{itemize}
\item Nos permite responder preguntas muy rápidamente.
\item Solo nos concentramos en los términos que nos interesan.
\item Mientras mas productos la formula se vuelve mas potente, ya que solo analizamos los términos que nos interesan.
\end{itemize}


\subsection*{Diagramas}
El análisis marginal introduce la variable de la cantidad vendida y el precio de venta. (Los diagramas de costos solo se enfocan en los costos)

Podemos ver el punto de equilibrio.

\begin{figure}[!htb]
    \centering
    \includegraphics[width=0.8\textwidth]{img/Knoppel.jpg}
\end{figure}

\begin{figure}[!htb]
    \centering
    \includegraphics[width=0.8\textwidth]{img/UtilidadEnFuncionDeVentas.jpg}
\end{figure}

\begin{figure}[!htb]
    \centering
    \includegraphics[width=0.8\textwidth]{img/EquilibrioUnitario.jpg}
\end{figure}

\subsubsection*{Punto de equilibrio}
\begin{itemize}
\item Nos indica la cantidad de unidades que debo vender para no ganar ni perder dinero.
\item Siempre se desea alcanzar, pero la meta es sobrepasarlo.
\item Se puede obtener en la ecuación fundamental igualando la utilidad a cero.
\item Se debe considerar pi, wi, Fpropios como constantes, así como la eficiencia.
\end{itemize}


\subsection*{Calidad de ventas}
U = m V - F

\medskip

Si el monto de ventas mide la cantidad de ventas, las tasa m nos define la calidad de ventas. Cuanto mayor es la utilidad marginal aportada por cada peso de venta tanto mejor es la calidad de ventas. Si mejora la calidad de sus ventas, en promedio cada peso vendido aporta mayor utilidad marginal.

Se ve con el $m_{original}$, $m_{desplazamiento}$ :

\begin{itemize}
\item $m_{empresa}$ = m1 a1 +... mn an
\item a1 + ... an = 1
\end{itemize}



\subsection*{Costeo marginal}

\begin{itemize}
\item Un producto puede ser vendido por debajo del costo unitario y sin embargo estar aportando positivamente a la utilidad marginal. Hay que realizar el análisis correctamente.
\item Siempre que pi > wi es conveniente que las ventas de ese producto aumenten, porque lejos de traer perdidas, traen ganancias adicionales. Se llama Costeo Marginal porque costea el producto a su costo variable.
\item Una vez que calcule la utilidad marginal del producto, lo comparo contra los fijos propios. Si es mayor es un producto rentable. 
\item Lo importante de un producto es que contribuya positivamente a formar E, utilidad marginal total de la empresa.
\end{itemize}


\subsection*{Variabilidad de los gastos fijos}

\begin{itemize}
\item Tienen un comportamiento independiente de los niveles de actividad, pero siempre dentro de un rango de variación.
\item Son constantes para una estructura y dimensión dada de la empresa y variaciones relativamente pequeñas del nivel de actividad. Tienden a crecer al crecer la dimensión de la empresa.
\end{itemize}

\subsection*{Precio Óptimo}
Es un precio para la cual Ei es máxima. Ese precio maximiza la utilidad marginal aportada por el producto.


\subsection*{Introducción de un nuevo producto}

\begin{itemize}
\item Realizamos el análisis marginal de la introducción de ese nuevo producto. Este tiene que incrementar las utilidades.
\item El precio de venta tiene que cubrir los costos variables unitarios y los fijos propios del producto. Si eso pasa va a ser conveniente introducir el producto.
\item Si el producto es sustitutivo se debe restar a la contribución en la utilidad del producto eliminado/sustituido.
\end{itemize}


\subsection*{Eliminación de producto}
Siempre que la utilidad marginal aportada por el producto sea menor que los gastos fijos propios del producto.

\medskip
$Ei < Fi$ $\rightarrow$  $(pi - wi) Qi < Fi$


\begin{itemize}
\item Cuando absorba algún insumo de MP o MOD que sea critico.
\item Cuando se remplaza por otro producto de mayor E.
\item Cuando la inversión que se necesite para producir sea elevada o utilice activos que con otro producto tengan mayor rendimiento.
\item Solo casos excepcionales de comercialización pueden inferir el mantenimiento del producto de utilidad negativa.
\end{itemize}

\subsection*{Comprar o fabricar}
Analizo el precio de compra si es inferior al costo variable de fabricación.

\medskip
pci < w'i


\begin{itemize}
\item Se puede decidir continuar con la fabricación si estuviera ocupando mano de obra calificada que de otra manera tuviera que despedirse.
\item Se decide comprarla afuera con el objetivo de liberar tiempo de maquina o tiempo hombre que es necesario para otras operaciones de mayor rentabilidad.
\end{itemize}

\subsection*{Eliminar sucursal}
Similar a los anteriores, tenemos que ver que productos trabajan esa sucursal.


\subsection*{Recurso escaso}
Para ver donde aplicar el recurso escaso necesito ver la utilidad marginal sobre el recurso escaso.


\begin{itemize}
\item $ganancia = \frac{e}{uso\ de\ recurso\ escaso}$
\end{itemize}

% LA DISTRIBUCION SE HACE EN HORAS HOMBRE, NO EN UNIDADES, SE TIENE QUE USAR UNA UNIDAD COMUN!!!!!!!!!!!!!!!!!!!!!!!!! (centro de servicio a productivos)



\newpage
\section*{Décima clase}

\section{Costos ABC}

\begin{itemize}
\item El precio de venta deja de ser variable y comienza a ser un techo.
\item Actividades: cual es la forma óptima en la que puedo mejorar la aplicación de los gastos generales sobre los productos.
\item Se consideran otras áreas (Ej Marketing)
\item Prioriza al cliente y su satisfacción y la calidad del servicio. No tanto a la mano de obra y la materia prima.
\item Se cuestionan los costos directos, como se asignan. Hacemos foco en evaluar los costos y su asignación. gestionar con mayor ventaja.
\item Los costos indirectos se asignan en base a variables en función del volumen (hh, hmaq, kg de MP)
\item El enfoque ABC nos dice que el costo no esta solo dado por el volumen, sino por factores más complejos (cadena de valor)
\item Puede ocurrir que se tengan grandes costos para productos con bajo volumen con procesos complejos de fabricación. El producto genera costos indirectos aunque sea de bajo volumen.
\item Esta forma de analizar las actividades que realiza la empresa y medir su costo y porque se realizan hace que la empresa se vea en forma mas horizontal 
\item Tiene importancia el uso de ABC cuando los costos indirectos tienen alto valor respecto de los otros costos (MP - MOD). Suele pasar en empresas de servicios o grandes empresas. Impacta mas a largo plazo que corto.
\item El costeo por absorción castiga al de mayor volumen.
\item El costeo ABC costea mejor al más complejo.
\end{itemize}


\subsection*{Principio}

\begin{itemize}
\item Los productos no consumen recursos, sino actividades y son estas las que consumen recursos. 
\item Se asignan los costos a los productos a través de las actividades necesarias para la producción de los mismos, es decir, se calcula previamente el coste de las actividades que realiza la empresa.
\item Los costos directos van directo a los productos, los indirectos van en función de las actividades, y estas a los productos.
\item Se tiene una tasa (estimada) por actividad. (Antes era solo la alfa o la beta)
\end{itemize}
 





\section*{Clase 11}
Parcial
% duracion 3hs y media del parcial
% 3 hs resolucion, 15 min+ de entrega
% Entrega por mail
% leer campus https://campus.fi.uba.ar/mod/page/view.php?id=112978
% practicar en el formato pensado a entregar
% unidades se aprueban independientemente
% si se usa excel aprovechar las formulas
% justificar todo lo que se hace, colocar hipotesis si es necesario
% balancear entre lo pedido y la justificacion, si piden balance final no es la idea realizar todas las fichas de libro mayor
% realizando la sumatoria de x valores puede ser suficiente
% 
% CONTABILIAD 
% Siempre colocar el tipo de cuenta y la variacion, nos va a ayudar revisar el asiento contable
% En la medida de lo posible revisar la sumatoria del debe y el haber, este error puede desaprobar el ej.
% Validar el balance A = P + PN
% Si piden balance pero no ficha del libro mayor, hacer intermedio, ej Disponibilidades = 175 + 56 - 130 = 110 ---- asi para las cuentas del balance
% 
% ANALISIS MARGINAL
% integrado a ejercico de costeo directo, basado en costeo directo, independiente del ejercicio de costeo directo - tiene preguntas teoricas
% generalmente se pregunta punto de equilibrio, mi, ei, utilizacion de recurso escaso, elimiar producto, eliminar sucursal, sustituir produccion
% variaciones de U wn funcion de las condiciones iniciales
% UN ERROR COMMUN es incluir los gastos fijos de estructura (aquellos que viven mas alla de la existencia de un producto o grupo de productos) (salvo que se pida en la utilidad operatva)
% TIP: manejar la cantidad de formulas acotada correctamente, necesarias fundamental, modelo analitico, ei, mi
% 
% FINANZAS
% origen y aplicacion, armar el grafico antes, de forma generica, poner indices significativos
% justificar las respuestas de los indices
% no poner hipotesis que no sean logicas (grados de venta que no tengan sentido, si estamos en argentina decir inflacion 2%)
% 
%
% mandar 2 formatos distintos? por las dudas

\newpage
\section*{Clase 12}
\section{Matemática financiera}
\begin{itemize}
\item Es un tema que nos toca a todos en mayor o menor medida. Por ejemplo ya sea pedir un préstamo o otorgarlos.
\item Hay que tener en cuenta diferentes cuestiones para el préstamo de dinero.
    \begin{itemize}
    \item Inflación
    \item Capital
    \item Tiempo: Prestando hoy nos perdemos de utilizarlo
    \item Oferta/Demanda: Cuanto se presta y cuanto se pide
    \item Riesgo: Si es alguien que conocemos, es un desconocido. Hay riesgos de solvencia, de que desaparezca.
    \end{itemize}
\item En este paso del tiempo esta lo que se conoce como función del interés (f(interés))
\item Nos interesa obtener lo prestado más algo más.
\item No es lo mismo el dinero de hoy de lo que va a valer a futuro.
\end{itemize}

\medskip
Algunas definiciones:
\begin{itemize}
\item \textbf{Tasa de interés}: Es la cantidad que se abona en una unidad de tiempo por cada unidad de capital invertido.
\item \textbf{Tasa pasiva}: La pagan las entidades financieras por los depósitos que captan. \textbf{Nos la pagan a nosotros}. Funciona en un modo como el costo de mercadería vendida.
\item \textbf{Tasa activa}: Es la \textbf{tasa que se cobra a quienes vienen a pedir el dinero}.
\item \textbf{Spread o Diferencial}: Es la diferencia de las tasas. Entre la activa y la pasiva
\item \textbf{Interés Simple}: Es el interés más simple que hay en matemática financiera. 
\end{itemize}

\begin{figure}[!htb]
    \centering
    \includegraphics[width=0.8\textwidth]{img/MatematicaFinanciera/Formulacion.PNG}
\end{figure}

\begin{figure}[!htb]
    \centering
    \includegraphics[width=0.8\textwidth]{img/MatematicaFinanciera/CuentasInteresSimple.jpg}
    \caption{Interés simple}
\end{figure}

Hay una tasa de interés que es la que nos van a pagar o tenemos que pagar con la particularidad que el interés se calcula sobre el capital inicial. \textbf{La tasa y la cantidad de periodos deben de ser compatibles.} (interés mensual, el tiempo debería de ser mensual)


\newpage
\begin{itemize}
\item \textbf{Tasa de interés nominal}: Es la rentabilidad obtenida en una operación financiera que se capitaliza de forma simple. \textbf{Solo se tiene en cuenta el capital principal} (TIN)
\end{itemize}

\begin{figure}[!htb]
    \centering
    \includegraphics[width=0.8\textwidth]{img/MatematicaFinanciera/TIN.jpg}
    \caption{Tasa interes nominal}
\end{figure}



\begin{itemize}
\item \textbf{Tasa Nominal Anual (TNA)}: Es la tasa nominal que se paga por periodo de un año considerando que el interés es simple. Se utiliza con fines informativos. Hay que tener en cuidado cuando nos vienen a ofrecer algo. Ponen el foco en alguna de las tasas para que parezca mas atractivo de lo que es. Busca dar transparencia a quienes dan servicios financieros.
\end{itemize}

\begin{figure}[!htb]
    \centering
    \includegraphics[width=0.8\textwidth]{img/MatematicaFinanciera/TNA.PNG}
    \caption{Tasa Nominal Anual}
\end{figure}


\newpage
\begin{itemize}
\item \textbf{Tasa proporcional}: Como el interés simple es lineal, si queremos calcular el capital con los intereses que obtenemos después de un periodo usamos esto si la tasa y el periodo no están en la misma unidad de tiempo. Se ve con interés compuesto y TEA.
\end{itemize}

\begin{figure}[!htb]
    \centering
    \includegraphics[width=0.8\textwidth]{img/MatematicaFinanciera/tasaProporcional.PNG}
    \caption{Tasa proporcional}
\end{figure}


\begin{itemize}
\item \textbf{Interés Compuesto}: Los intereses pasan a formar parte del 'capital' (bola de nieve). Es la acumulación de intereses que se han generado en un periodo determinado por un capital inicial a una tasa de interés durante cierta cantidad de periodos que dure. Los intereses se reinvierten conjuntamente al capital inicial.
\end{itemize}

\begin{figure}[!htb]
    \centering
    \includegraphics[width=0.8\textwidth]{img/MatematicaFinanciera/interesCompuestoFormula.PNG}
\end{figure}

\begin{figure}[!htb]
    \centering
    \includegraphics[width=0.8\textwidth]{img/MatematicaFinanciera/InteresCompuesto.PNG}
\end{figure}


\newpage
\subsection{Valor actual y valor futuro}

\begin{figure}[!htb]
    \centering
    \includegraphics[width=0.8\textwidth]{img/MatematicaFinanciera/ValorActualValorFuturo.PNG}
\end{figure}


\begin{itemize}
\item \textbf{Tasa equivalente}: Son tasas equivalentes aquellas que siendo distintas capitalizan en distintos plazos y en un mismo plazo total a un mismo capital para producir un mismo monto total. Nos dicen que obtenemos el mismo monto final y el plazo es el mismo pero capitalizan de forma distinta. Nos va a permitir hacer ciertas conversiones.
\end{itemize}

\begin{figure}[!htb]
    \centering
    \includegraphics[width=0.8\textwidth]{img/MatematicaFinanciera/TasaEquivalente.PNG}
    \caption{Tasa equivalente}
\end{figure}



\newpage
\begin{itemize}
\item \textbf{Tasa efectiva}: Es aquella que capitaliza a un solo periodo. Si es a un año, se denomina Tasa Efectiva Anual (TEA), indica cuanto se debería de obtener a un año o cuanto deberíamos de estar pagando. Dada la TNA se puede obtener la TEA. Generalmente tiene mas importancia la TEA por estar hablando de interés compuesto. Nos permite obtener la tasa de un periodo de manera muy rápida.
\end{itemize}

\begin{figure}[!htb]
    \centering
    \includegraphics[width=0.8\textwidth]{img/MatematicaFinanciera/tasaEfectivaAnual.PNG}
    \caption{Tasa efectiva anual}
\end{figure}




\subsubsection*{Relación de TEA y TNA}
Están relacionadas, se utiliza la formula de tasa equivalente para obtener la que queramos.

\begin{figure}[!htb]
    \centering
    \includegraphics[width=0.8\textwidth]{img/MatematicaFinanciera/RelacionTEAyTNA.PNG}
\end{figure}

En la gran mayoría de los casos la TEA va a ser mas grande que la TNA. A medida que la TNA es mayor, se observa una diferencia mas marcada entre la capitalización por interés simple vs compuesto.  Al calcular la TEA el m es generalmente 365/30.

\subsection{Costo financiero total}
\begin{itemize}
\item El CFT es el costo final o real de un préstamo, crédito o de la financiación de una tarjeta. Se incluye la tasa de interés y todos los costos involucrados.
\item En algunos lugares se llama costo financiero total efectivo anual (CFTEA)
\item Es básicamente la tasa que sirve para calcular el costo de un crédito. La gran bola de nieve que vamos a terminar pagando (ej IVA (puede o no incluirlo, por lo general si), seguros, gastos de evaluación y otorgamiento, enchufan los cargos acá).
\end{itemize}


El orden a ver para poder decidir es: CFT, TEA, TNA (si queremos hacer alguna conversión)

\newpage
\subsection{Tasa Real}
Es el tipo de interés real. El rendimiento neto que obtenemos en la cesión de capital o dinero, una vez hemos tenido en cuenta los efectos y las correcciones en la inflación.

\begin{figure}[!htb]
    \centering
    \includegraphics[width=0.8\textwidth]{img/MatematicaFinanciera/TasaReal.PNG}
\end{figure}

Por ejemplo, aumentan sueldos en 40\% y la inflación es de 45\%. Se perdió frente a la inflación, la i real es negativa.

\newpage
\subsection{Renta}
Es una serie de pagos iguales y realizados a intervalos constantes. El pago puede ser a periodo vencido o anticipado. 

\begin{figure}[!htb]
    \centering
    \includegraphics[width=0.8\textwidth]{img/MatematicaFinanciera/APeriodoVencidoValorActual.PNG}
    \caption{A periodo vencido, R son las cuotas}
\end{figure}

\begin{figure}[!htb]
    \centering
    \includegraphics[width=0.8\textwidth]{img/MatematicaFinanciera/RentaAPeriodoAnticipado.PNG}
    \caption{A periodo anticipado, R son las cuotas}
\end{figure}

Llevamos a valor actual los diferentes valores futuros.

\begin{figure}[!htb]
    \centering
    \includegraphics[width=0.8\textwidth]{img/MatematicaFinanciera/APeriodoVencidoValorActual.PNG}
    \caption{Valor futuro}
\end{figure}

\begin{figure}[!htb]
    \centering
    \includegraphics[width=0.8\textwidth]{img/MatematicaFinanciera/Relaciones.PNG}
    \caption{Relaciones entre Valor actual vencido y anticipado}
\end{figure}

Para valor futuro es parecido.

\newpage
\subsection{Prestamos}

Dinero que nos prestan y después tenemos que devolver en cuotas. Tienen un componente de interés.

\subsubsection{Sistema Francés}

Muy emparentado con renta debido a que las cuotas son todas iguales.

\begin{figure}[!htb]
    \centering
    \includegraphics[width=0.8\textwidth]{img/MatematicaFinanciera/SistemaFrances.PNG}
\end{figure}


\begin{figure}[!htb]
    \centering
    \includegraphics[width=0.8\textwidth]{img/MatematicaFinanciera/SistemaFrances2.PNG}
\end{figure}

\newpage
\subsubsection{Sistema Alemán}
\begin{itemize}
\item Busca que la amortización de capital sea constante en cada cuota.Es fija la amortización de capital.
\item La sumatoria de amortizaciones es igual al monto del préstamo (Cancelamos toda la deuda tomada)
\item La parte amarilla corresponde a egresos.
\item Cuando cambiamos de préstamo, nos llevamos el monto no devuelto del anterior préstamo.% problemas, se plantean con las condiciones dadas, por mas que despues cambien -
\end{itemize}


\begin{figure}[!htb]
    \centering
    \includegraphics[width=0.8\textwidth]{img/MatematicaFinanciera/SistemaAleman.PNG}
\end{figure}

\begin{figure}[!htb]
    \centering
    \includegraphics[width=0.8\textwidth]{img/MatematicaFinanciera/SistemaAleman2.PNG}
\end{figure}


\newpage
\subsection{Inflación: IPC}

\begin{itemize}
\item IPC se construye sobre una canasta de productos. Dependiendo que se pone en la canasta, puede ser representativa o no para nosotros.
\item El INDEC hace un relevamiento de hogares y pregunta que se consume para formar una lista de productos. Mensualmente los encuestadores visitan los comercios y hacen un relevamiento para calcular la variación.
\item Si no consumimos estos productos no nos es muy representativa.
\item Hay que tomar con pinzas los productos que están, por ejemplo productos con precios cuidado, este valor de inflación seria tendencioso.
\item También hay que tener en cuenta si esos productos se encuentran en las góndolas, no tiene sentido si solamente se encuentra en un lugar.
\end{itemize}


\subsection{¿Como calcular la inflación personal?}

\begin{itemize}
\item Operar de la misma manera, relevar lo que consumimos y llevar un registro para obtener el valor.
\end{itemize}

\subsection{¿Como se calcula el riesgo país?}
Es la sobre tasa que pagan los bonos argentinos para que sean aceptados en el mercado respecto de los bonos del tesoro del USA que se consideran libres de riesgo.

\begin{itemize}
\item Cada punto son 100 puntos básicos
\item No es cualquier bono, tienen que cumplir ciertos requisitos.
\item Ej 2\% paga USA, ARG 18\%, 16\% de sobre tasa. Son 1600 puntos básicos
\item Tiene mas sentido para países emergentes.
\item Tiene en cuenta principalmente el riesgo.
\end{itemize}


% Excel, poniendo $celda letra$numero la dejamos fija
% si uso la TEA para los meses, fijarse de llevarla a su equivalente mensual - usamos la conversion de tasas equivalentes







\newpage
\section*{Clase 13, 15, 16}
\section{Evaluación de proyectos de inversión}
\begin{itemize}
\item Se basa en como convertir una idea en algo concreto. Se debe evaluar desde la racionalidad económica, si aumenta o disminuye la riqueza.
\item Pueden ser inversiones a a diferente plazo. Algunos ejemplos de inversiones
\item Activos fijos, espero obtener retornos en mediano y largo plazo
\item Consecuencias mas inmediatas, ingresos incrementales, estudiar los costos, calcular lotes de compra
\item Se pueden comparar ingresos y egresos en periodos menores de 1 año.
\item Hay que ver cuanto perdemos o ganamos en cada alternativa que tengamos.
\end{itemize}


\subsection{Decisiones de inversiones}
\begin{itemize}
\item La inversión es una erogación de capital con la intención de obtener un retorno en el futuro que pague la inversión original y genere una utilidad adicional.
\item Se hace un flujo de fondos durante la vida útil del proyecto para aplicar diferentes criterios de evaluación. (Ej VAN, TIR, periodo recuperación con actualización de fondos, o IVA (índice de valor actual) )
\item La evaluación de proyectos tiene un objetivo del proyecto. Se hace un análisis del mercado, un análisis técnico operativo, económico financiero, y social ambiental. En base a esto tomamos una decisión
\begin{itemize}
\item En el de mercado hay que ver que oferta hay, que demanda, como están los precios, y de comercialización (costos de comercialización para lo que queremos hacer). Con esto sacamos una conclusión del mercado.
\item El análisis técnico operativo tiene que ver con el tamaño que va a tener el proyecto, donde se va a ubicar (ejemplo reducción de impuesto), análisis de costos y disponibilidad de recursos necesarios, determinación de la organización humana.
\item El análisis económico financiero consiste en realizar el cuadro de resultados proyectado, ingresos y egresos operativos, el valor del pago de impuestos a las ganancias.
\item El flujo de fondos proyectado consiste en determinar el monto de inversión que requiere la puesta en marcha, estimar la secuencia temporal en el que se integrara la inversión, aplicar los índices y metodologías de evaluación y comparación de proyectos en base al flujo de fondos. Con esto se determina se evalúa la bondad del proyecto
\end{itemize}
\end{itemize}


Hay que tener en cuenta que:

\begin{itemize}
\item Hay que saber diferenciar los hechos económicos (involucran cuentas de resultados, RP, RN) de los financieros (involucran disponibilidades) y sus implícanos.
\item Hay que tener la capacidad de discernir como incrementar los beneficios de la empresa. (Utilidad = Ventas - Costos)
\item Mejorar el nivel de caja. (Ingresos - Egresos de caja) La formula de flujo de caja, el delta de disponibilidad
\item Discernir la conveniencia de realizar una inversión.
\item Entender la rentabilidad económica y financiera
\end{itemize}

\newpage
\subsection{Gráfico de análisis marginal aplicado a finanzas}
\begin{itemize}
\item Las ventas se cobran y los gastos se pagan en distintos momentos que las ventas y costos que aparecen en el cuadro de resultados. La diferencia entre lo cobrado y lo pagado nos da el nivel de disponibilidades.
\item Hay gastos que no pagamos, por ejemplo las amortizaciones, ya se pago su inversión correspondiente.
\end{itemize}


\begin{figure}[!htb]
    \centering
    \includegraphics[width=0.8\textwidth]{img/EvaluacionProyectos/KnoppelEvaluacionProyecto.PNG}
\end{figure}



\subsection{Cuadro de resultados proyectado}

Es ingresos menos egresos.

\begin{figure}[!htb]
    \centering
    \includegraphics[width=0.8\textwidth]{img/EvaluacionProyectos/FlujoDeFondosProyectado.PNG}
\end{figure}

\begin{itemize}
\item Cada periodo tiene ventas y costos.
\item Si tengo mas ingresos que egresos voy a tener un valor que lo debo afectar por los impuestos. Se lleva el pago de impuestos al flujo de fondos. Nos sirve para calcular el impuesto a las ganancias.
\end{itemize}


\subsection{Fondo de maniobra o capital de trabajo}
Es importante para incorporarlo al flujo de fondos

\begin{figure}[!htb]
    \centering
    \includegraphics[width=0.8\textwidth]{img/EvaluacionProyectos/CapitalDeTrabajo.PNG}
\end{figure}

\newpage
\subsection{Armado del flujo de fondos}
Tenemos egresos e ingresos de caja.

\begin{figure}[!htb]
    \centering
    \includegraphics[width=0.8\textwidth]{img/EvaluacionProyectos/CuadroDeREsultadosGrafico.PNG}
\end{figure}

\begin{itemize}
\item Capital de trabajo lo ponemos en el flujo de fondos, es una inversión, un egreso al comienzo del proyecto que no se amortiza.
\item El pago de impuestos lo tengo que llevar al flujo de fondos, aparece el impuesto a las ganancias. (Si tuve mas ingresos que egresos)
\item Si vendo/cobro o pago algo a varios días (ej 30, 60 dias) lo tengo que tener en cuenta. En el análisis por lo general no podemos desglosarlo debido a que son periodos mas grandes (ej 1 año)
\item Los ingresos son por cobranzas del pronostico de ventas. Recuperación al vender el inventario/maquinarias (Valor rezago) se recupera el capital de trabajo y la inversión
\item Lo que vendí en el año lo cobrare en ese año y lo que compre en ese periodo lo pagare en ese año.
\item Al finalizar el proyecto recupero el capital de trabajo, debido a que es activo corriente menos pasivo corriente. Por ejemplo recupero el stock. No hay amortización por el capital de trabajo. Si hay por ejemplo con maquinas.
\item Al final esta también el recupero de la inversión, por ejemplo la amortización total de una maquina, al venderla recupero algo. Por mas que valga menos que lo inicial lo manifestamos.
\item El flujo neto es considerando los impuestos.
\end{itemize}



\newpage
\subsection{Valor actual neto o presente neto}

\begin{itemize}
\item Valor actual neto, se aplica la tasa de actualización para ver cuanto se esta recuperando en el tiempo. Llevamos lo obtenido al valor actual (el del comienzo, el año 0)
\item Una vez hecho esto podemos hacer una suma para ver como nos fue. Usamos el interés compuesto.
\item Depende del costo de capital y el riesgo empresario
\item Tasa de actualización se considera como 10\%, lo definimos nosotros
\item Es el remanente (ingreso o egreso al final) al llevarlo al año 0.
\item Da una idea de montos
\item No de idea de rendimiento, 100 puede venir de algo previsto de 10 o de 10000
\item Su comprensión no es inmediata, hay que conocer el sistema y los conceptos que involucra. (interés compuesto y valor actual de una suma futura)
\item Es propia del inversionista, sesga el resultado. Cambia dependiendo de cada inversionista.
\item La F es 1 dividido $(1+i)^n$ con i la tasa de actualización (descuento)
\item Asume que los flujos netos que se van obteniendo son reinvertidos a la tasa requerida de rendimiento.
\end{itemize}

\begin{figure}[!htb]
    \centering
    \includegraphics[width=0.8\textwidth]{img/EvaluacionProyectos/ValorActualNeto.PNG}
\end{figure}


\begin{figure}[!htb]
    \centering
    \includegraphics[width=0.8\textwidth]{img/EvaluacionProyectos/ValorActualNetoEjemplo.PNG}
\end{figure}

\newpage
\subsection{Tasa de descuento - actualización}

\begin{itemize}
\item La tasa de descuento apropiada para usar en la determinación de VAN en el costo promedio ponderado del capital.
\item Este costo es el promedio ponderado de los rendimientos que esperan las partes que contribuyen fondos. Estos pesos de ponderación son determinados por las proporciones de fondos que provee cada fuente. (si lo pedimos o no, si ponemos una parte)
\item Estamos analizando si el proyecto sirve o no. Si da positivo es aceptado.
\item Es el costo del dinero.
\item En el caso de la inflación podemos llevar los valores a alguna moneda constante o tenerla en cuenta. Viene de la parte de la tasa real de matemática financiera también. (Si nos dan la tasa con el componente inflacionario usamos esa como nos la dan, si nos dan una sin esto averiguamos la i real)
\end{itemize}


\newpage
\subsection{Tasa de corte}

\begin{itemize}
\item Es la tasa que la empresa define para la aprobación de sus proyectos
\item Es decir que calculando sus costos de capital define una tasa de actualización
\item En general también se le agrega el riesgo empresario (4\% en nuestro caso) (Ejemplo de este riesgo, poner dinero en plazo fijo tiene riesgo nulo)
\item Por ejemplo, si nos da un rendimiento mayor a la tasa de corte es un proyecto aceptable.
\item Ejemplo de tasa de corte =  tasa anterior promedio + riesgo empresario
\end{itemize}

\subsection{Proyecto + financiación}

\begin{itemize}
\item Se trata de analizar el proyecto con su tasa de actualización estándar. Se trata de conseguir el proyecto base. Vemos si podemos conseguir mejor financiación.
\item Se consideran distintos tipos de financiación con el sistema francés o alemán. Se ve si convienen los prestamos a una tasa baja, el tiempo de devolución, como es su impacto sobre la evolución del proyecto, la incidencia de la financiación en el flujo del capital propio.
\item Elegimos la financiación más conveniente.
\end{itemize}


\subsection{Ejemplo de Flujo de fondos}

\begin{figure}[!htb]
    \centering
    \includegraphics[width=0.8\textwidth]{img/EvaluacionProyectos/EjemploFlujoDeFondos.PNG}
    \caption{Considera que ya están los impuestos}
\end{figure}

\newpage
\subsection{Tasa interna de retorno - TIR}

\begin{itemize}
\item Es aquella que llevada al año 0 todos los ingresos a esa tasa iguala los egresos
\item Actualizando los valores de ingresos equipara al de los egresos
\item Es directamente comparable con el costo del capital de la empresa o con el interés del dinero vigente en el mercado.
\item La comprensión no es inmediata
\item No da idea del monto invertido
\item Presupone que los fondos ingresados se reinvierten a la tasa del proyecto (a la misma TIR)
\item Es propia de la inversión, no tiene un resultado sesgado. No cambia, es de la inversión.
\item Es la tasa máxima que podría pagar un inversor si no ganara nada
\item Se hace efecto palanca si la tasa de financiación es menor a la TIR del proyecto ($i_{financicacion} < TIR_{proyecto}$), esto hace que la TIR aumente. Es por recibir dinero prestado a una tasa menor del proyecto original. Para hallar la TIR con la financiación se hace el VAN del nuevo flujo (el que tiene en cuenta la financiación). Si pedimos financiación con la misma tasa que la TIR el flujo del capital propio daria igual. (lo mismo que me cobran es lo mismo que obtengo)
\end{itemize}

\begin{figure}[!htb]
    \centering
    \includegraphics[width=0.8\textwidth]{img/EvaluacionProyectos/TIRCuenta.PNG}
\end{figure}

\begin{figure}[!htb]
    \centering
    \includegraphics[width=0.8\textwidth]{img/EvaluacionProyectos/EjemploTIR.PNG}
\end{figure}

\newpage
\subsection{Periodo de recuperación}

\begin{itemize}
\item Es un paso intermedio previo al VAN. Es el lapso en el cual el VA de los egresos es recuperado a través de VA de los ingresos a un interés determinado.
\end{itemize}

\begin{figure}[!htb]
    \centering
    \includegraphics[width=0.8\textwidth]{img/EvaluacionProyectos/PeriodoDeRecuperacion.PNG}
\end{figure}

\begin{figure}[!htb]
    \centering
    \includegraphics[width=0.8\textwidth]{img/EvaluacionProyectos/EjemploPeriodoRecuperacion.PNG}
    \caption{La tasa de actualización es la tasa de interés anual}
\end{figure}

\newpage
\subsection{Consideraciones}

\begin{figure}[!htb]
    \centering
    \includegraphics[width=0.8\textwidth]{img/EvaluacionProyectos/Consideraciones1.PNG}
\end{figure}

\begin{figure}[!htb]
    \centering
    \includegraphics[width=0.8\textwidth]{img/EvaluacionProyectos/Consideraciones2.PNG}
\end{figure}


\newpage
\subsection{Análisis de sensibilidad}

\begin{itemize}
\item Nos indica que variables de riesgo afectan mas, las seleccionamos.
\item Por ejemplo vemos que pasa si las ventas aumentan/disminuyen, o que pasa con los precios de venta/costos. Otra opción puede ser alquilar en vez de comprar un terreno.
\item Estimamos que pasa cambiando de a un parámetro. (las demás se mantienen constantes)
\item Una variable es importante dependiendo de su participación porcentual en los beneficios o costos y su rango de valores probables.
\item El análisis permite determinar la dirección del cambio en el VAN.
\item Algunas de las principales variables macroeconómicas son el nivel de pronostico del PBI, el tipo de cambio, la tasa de interés, y el riesgo país. Estas variables se interrelacionan
\item Algunos escenarios que se pueden plantear son devaluación (caen ventas, aumentan costos), nuevos competidores, bajas/altas tasas de interés, problemas de importación.
\end{itemize}

\begin{figure}[!htb]
    \centering
    \includegraphics[width=0.6\textwidth]{img/EvaluacionProyectos/graficoAnalisisSensibilidadEjemplo.PNG}
        \caption{Grafico de ejemplo para mostrar la variacion de las variables con su efecto en el VAN}
\end{figure}

\newpage
\subsection{Índice de Valor Actual Neto (I.V.A)}

\begin{enumerate}
\item Valor actual neto (VAN) es el resultado ya.
\item Indica cuanto ganamos por cada unidad arriesgada de la máxima inversión.
\item Mientras mayor sea el I.V.A. mejor.
\item Nos sirve para calcular la eficiencia de la inversión y es el cociente del valor actual neto y la máxima inversión actualizada.
\item Nos sirve para comparar alternativas de inversión cuando existen limitaciones presupuestarias.
\item Puede usarse para cuantificar el riesgo también.,
\end{enumerate}

\begin{figure}[!htb]
    \centering
    \includegraphics[width=0.5\textwidth]{img/EvaluacionProyectos/IVAEvaluacionProyectos.PNG}
\end{figure}

\newpage
\subsection{Resolución de ejercicios}
\begin{itemize}
    \item Determinamos los movimientos económicos para construir el cuadro de resultados. (RP, RN)
    \item Determinamos los movimientos financieros (Ingresos o egresos de dinero) para armar el flujo de fondos
    \item Ya con el flujo de fondos podemos obtener el VAN y la TIR (lo sacamos remplazando la i, nos vamos acercando, no es necesario dar el valor exacto)
    \item En los proyectos apalancados (\textit{gano plata con plata de los demás}) puede pasar que no consiga un valor que haga que el VAN sea 0. (si siempre tiene valores positivos)
    \item En tasa con moneda constante usamos la tasa real. No se tiene inflación, es para proyectos no muy largos (menos de 2 años) $ (1 + i_{real}) = \frac{1 + i_{costoDinero}}{1 + F_{inflacion}}  $
    \item En tasa con moneda corriente tenemos la inflación, la usamos tal cual la dan. (tasa completa)
    \item Si el proyecto paga impuestos a las ganancias hay que hacer el cuadro de resultados
    \item Si hay un préstamo calculamos el total a pagar en cada periodo para ver como nos afecta, los intereses van al cuadro de resultados (es RN) y el total de la cuota va al flujo de fondos (es egreso de dinero)
    \item Para el periodo de recupero simple no se tiene en cuenta el valor actual (sumo directo) (para el calculo asumimos que es lineal lo que se obtiene en un periodo)
\end{itemize}

% estado de resultados nos sirve para saber el IG, que se refleja en el flujo de fondos

% amortizaciones tener una reserva de dinero, para saber la utilidad real de la empresa
% si ganamos 10k, no ganamos realmente eso, por ahi es 7k, os otros 3k se usarian para mantener el nivel de vida, la heladera, auto, eventualmente se van a romper

% anual el analisis es para tener una visibilidad macro del proyecto
% mensual es mas trabajoso, el analisis es mas fino

% que es una amortizacion, que es el capital de trabajo, puede ser negativo? el costo por absorcion que diferencia tiene con el directo? en que cambian los graficos de marginal con los costos (costos no habla del precio de venta).
% costos ABC para cuando los costos de tamaño son > 10 veces el tamaño de los directos


\newpage
\section*{Clase 14}
\section{Marketing}
\begin{itemize}
\item Es el arte de explorar, crear y entregar valor para satisfacer las necesidades de un mercado. No es solamente hacer publicidad o promociones, es ver a quienes le vendemos, como nos diferenciamos, quienes son los competidores.
\item Es útil para los dueños, directivos, empleados. (Para todos)
\item Es lo más importante que tiene una empresa. Olvidarse la idea de marketing como publicidad.
\end{itemize}


\subsection{5C Clientes, compañía, competencia, colaboradores, contexto}

\begin{itemize}
\item El contexto esta arriba de todo. El marketing se analiza en función de determinado contexto.
\item Intenta responder a:
\begin{itemize}
\item \textbf{Cliente}: ¿Que necesidades satisfacemos? Tenemos que entender el mercado, si hay muchos players o son pocos, ¿a quienes apuntamos?, ¿por que?, ¿Como usan el producto y por que dejarían de usarlo? Hay diferentes tipos, Iniciador, Influenciador, Decisor, Comprador, Usuario. Cuanta más información pueda tener una empresa sobre lo que se nos ocurra para comprar, se va a tener una mejor estrategia de marketing. Necesitamos enfocarnos en necesidades.
\item \textbf{Compañía}: ¿Que fortalezas y debilidades tenemos? Tenemos que responder la misión (¿Para que estamos?) y la visión (¿A donde vamos?), las fortalezas, y las debilidades. \textit{Nike: traer inspiración e innovación para cada atleta en el mundo. Si tienes un cuerpo, eres un atleta}
\item \textbf{Competencia}: ¿Con quienes competimos? Necesitamos entenderla, la actual, la futura, y la sustitutiva.
\item \textbf{Colaboradores}: ¿Con quienes colaboramos? Tenemos que ver quienes son. Los competidores también colaboran con nosotros. (Por ejemplo cuando se unen para hablar con gobiernos o organismos mas grandes) Algunos colaboradores son los accionistas, inversores, empleados, sindicatos, proveedores, legisladores, clientes.
\item \textbf{Contexto}: ¿Cual es el contexto cultural, legal, limites tecnológicos? Hay diferentes factores, económicos, sociodemograficos, culturales, políticos, legales, tecnológicos. Se utiliza el FODA como una herramienta. (Fortalezas, Debilidades, Oportunidades, Amenazas) \textit{Ejemplo, Blockbusters no supo ver lo que el publico demandaba. }
\end{itemize}
\end{itemize}



\subsection{Segmentación del mercado o nichos}
\begin{itemize}
\item ¿A quien se lo queremos vender? Si le queremos vender a todos, lo mas probable es que a largo plazo terminemos por no venderle a nadie.
\item Hay que tratar de buscar nichos, lugares donde hay demanda insatisfecha. El mercado es demasiado amplio y esta formado por clientes con diferentes necesidades.
\item Un segmento del mercado es un grupo de consumidores que debe responder de forma similar a un conjunto determinado de esfuerzos de marketing. Estos segmentos deben ser homogéneos (fáciles de identificar), sustanciales (dimensión adecuada), medibles, y accesibles (llegar a ellos). Algunas posibles es por nivel de ingreso, región, conducta. 
\item Un ejemplo seria de segmento seria: hombres que compran zapatos. Un nicho, hombres mayores de 20 años que compran zapatos con ingresos mayores a \$30000. Un micro nicho seria igual al de nicho pero que calzan mas de 45. Se va volviendo cada vez mas especifico.
\end{itemize}


\subsection{7P - 4P}

\begin{itemize}
\item \textbf{Producto}, tiene diferentes etapas, desde que se diseña e introduce hasta que se retira. Aveces lo que falla no es el producto, sino el marketing (mercado chico, no esta bien definido, errores de estimación).
\item \textbf{Promoción}, son las actividades de comunicación para tratar de dar a conocer el producto y venderlo. Por ejemplo, se filtraron imágenes de X cosa. Muchas veces es parte de una campaña de marketing de la propia empresa para tratar de hacer creer al cliente que tiene información antes de que se haga publica.
\begin{itemize}
\item Above The Line (ATL), es una campaña masiva, radio, tv, carteles, etc
\item Below The Line (BTL), no es una campaña masiva, requiere seleccionar bien el segmento al que se apunta (ej mensajes personalizados). Utiliza canales no tradicionales y apuesta a la creatividad. Busca ser llamativo y viral. Hay que seleccionar bien el segmento al que se apunta.
\end{itemize}
\item \textbf{Plaza}, buscar donde se va a encontrar el producto. Desde el canal directo, productor consumidor, a toda la cadena de distribución que hay. Debe de tener en cuenta el grado de separación deseado con nuestros clientes, las características del producto, la necesidad de distribución y la elección del punto de venta. Nos da un valor agregado al momento de generar ingresos.
\item \textbf{Precio}, consiste del aspecto interno y el externo.
\begin{itemize}
\item El interno esta el objetivo, marketing, y costos.
\item El externo es análisis del mercado y la competencia.
\end{itemize}
\end{itemize}

En servicios hay diferencias. No se acumulan, no hay inventario, no se intercambian, agrega las otras P:
\begin{itemize}
\item \textbf{Procesos}, Homologado, como se vuelve estándar una marca.\textit{Ejemplo Starbucks}
\item \textbf{Personas}
\item \textbf{Physical}, el entorno influye en el servicio que voy a percibir. La gente se siente mas incentivada a ir al lugar. Cambia la experiencia total que recibe la persona. Lo mismo para el entorno de trabajo en las empresas (tecnologías por ejemplo).
\end{itemize}


\subsection{Influencias}

\begin{figure}[!htb]
    \centering
    \includegraphics[width=0.7\textwidth]{img/Influencias Marketing.PNG}
\end{figure}


\subsection{Marketing relacional}
\begin{itemize}
\item Tratar de entender a nuestro cliente. Saber quien es. Es mucho menos costoso retener un cliente que captar uno nuevo.
\item Se busca crear y mantener relaciones duraderas con nuestros clientes.
\item Hay muchísimas herramientas para esto, CRM
\item Una empresa no puede existir sin marketing. 
\item Una empresa sin marketing no piensa.
\end{itemize}


\section{Oferta y demanda}

\begin{itemize}
\item Intenta explicar la relación entre la demanda de un bien y la cantidad del mismo en base al precio.
\item \textbf{La cantidad demandada}: es la cantidad que los compradores quieren y pueden comprar.
\item \textbf{Ley de la demanda}: manteniendo todo lo demás constante, la demanda de un bien disminuye cuando sube su precio
\item \textbf{Cantidad ofertada}: es la cantidad de un bien que los productores quieren y pueden vender.
\item \textbf{Ley de la oferta}: manteniendo todo lo demás constante, la cantidad de un bien aumenta cuando sube su precio
\end{itemize}

\subsection*{Afecta a la demanda}

\begin{itemize}
\item Bien normal: si al aumentar los ingresos su demanda aumenta.
\item Bien inferior: si al aumentar ingresos la demanda disminuye
\item Moda
\item Bien sustituto: aumento del bien 1 produce un aumento de la demanda del bien 2.
\item Bien complementario: disminuir el precio del bien 1 hace que aumente la cantidad demandada del bien 2
\end{itemize}

\subsection*{Factores que afectan a la oferta}

\begin{itemize}
\item Mejora en la producción
\item Costos
\item Bien sustituto: aumentar el precio del bien X produce un aumento en la oferta del bien Y.
\item Bien complementario: la subida del precio del bien X produce una disminución en la oferta del bien Y.
\end{itemize}

\subsection*{Competencia perfecta}
Tomarlo como un modelo teórico. Siempre que se cumplan los supuestos.


\subsubsection*{Supuestos}

\begin{itemize}
\item Hay muchas empresas que venden productos idénticos a muchos compradores
\item No hay restricciones para entrar en la industria
\item Las empresas establecidas no tienen ventajas respecto de las nuevas.
\item Los vendedores y compradores estan bien informados acerca de los precios.
\end{itemize}

\subsubsection*{Como funciona}

\begin{itemize}
\item La demanda y la oferta de la industria determinan el precio del mercado y la producción de la industria. Si la demanda aumenta, la curva de demanda se desplaza hacia la derecha, el precio de equilibrio sube. Si la demanda disminuye, la curva de demanda se desplaza hacia la izquierda, entonces el precio de equilibrio baja.
\end{itemize}

Esto se ve afectado por diferentes comportamientos al agregar un impuesto o poner protecciones de precios.

\begin{figure}[!htb]
    \centering
    \includegraphics[width=0.7\textwidth]{img/OfertaDemanda.PNG}
\end{figure}


% TP
% hacer un bosquejo
% sacar ingresos, hacer el cuadro de resultados
% que podria pasar con el proyecto, a que es sensible
% si tengo un salto en los costos variables se refleja en el capital de trabajo
% 3 escenarios, hay 3 grupos
% devaluacion variamos el costo de venta para ver como afecta el costo de venta, que pasa  si auentamos el precio de venta. es analizar el problema
% tiene que ser algo que no lleve mas de unos minutos presentarlo
% 2 entrada competidor, precio maximo de 2,8, se reducen las utilidades en un 15%
% 3 baja de intese en los prestamos
% incluir prestamo dentro de la evaluacion del proyecto





\end{document}